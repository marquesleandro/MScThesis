\begin{center}
\textbf{ABSTRACT}
\end{center}

% JUST ONE paragraph.
% Number of sheets (NS) = PDF pages - 2 (less cover and 2 first pages, which are printed at the same sheet)

$\!$\\

\hspace{-1.3cm}\textbf{Marques}, Leandro \textit{An ALE-FE Method for Blood Flow Dynamics in Coronary Arteries Using the Vorticity-Streamfunction Formulation}. xxf. Master's Thesis~(Master of Mechanical Engineering) - Engineering Departament, State University of Rio de Janeiro~(UERJ), Rio de Janeiro, Brazil, 2020.

\vspace{.2cm}

\indent 
The present work aims at developing a 
computational framework to simulate blood flow 
in coronary artery with
drug-eluting stent placed for several Schmidt number using the Vorticity-Streamfunction 
Formulation in an Arbitrary Lagrangian-Eulerian (ALE) approach.
The blood was modeled as single-phase, incompressible 
and newtonian fluid. The Navier-Stokes equation is 
shown according to the vorticity-streamfunction 
formulation with species transport equation.
The Finite Element Method (FEM) is used to solve 
the governing equations where the Galerkin formulation was used
to discretized the governing equations in spatial domain 
and the semi-Lagrangian method was used to discretized 
the material derivative using first order backward difference scheme. 
Moreover, the mesh update procedure performs nodes regularisation
allowing to simulate moving boundary problems and
the linear systems was solved using Conjugate 
Gradient Iterative Method.
Therefore, the new formulation proposed in this work 
is suitable to simulate flows in coronary arteries with good accuracy. 

\vspace{1cm}

\hspace{-1.3cm}Keywords: Abritary Lagrangian-Eulerian; Vorticity-Streamfunction Formulation; Finite Element Method; semi-Lagrangian Method; Drug-Eluting Stent.
