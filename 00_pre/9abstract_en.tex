\begin{center}
\textbf{ABSTRACT}
\end{center}

% JUST ONE paragraph.
% Number of sheets (NS) = PDF pages - 2 (less cover and 2 first pages, which are printed at the same sheet)

$\!$\\

\hspace{-1.3cm}\textbf{Marques}, Leandro \textit{Finite Element Numerical Simulation of Flow in Coronary Artery}. xxf.  Undergraduate thesis~(Bachelor of Mechanical Engineering) - Engineering Departament, State University of Rio de Janeiro~(UERJ), Rio de Janeiro, 2018.

\vspace{.2cm}

\indent The present work aims at developing a computational framework to simulate coronary artery flows in cartesian
coordinates. The Finite Element Method (FEM) was used to solve the governing equations of the blood flow in coronary
artery with atherosclerosis and drug-eluting stent placed. The blood was modeled as single-phase, incompressible and newtonian fluid. The Navier-Stokes equation was shown according to the stream-vorticity formulation with chemical species transport equation. The Taylor-Galerkin scheme were used to decrease spurious oscillations as seen when the convective term is predominant.

\vspace{1cm}

\hspace{-1.3cm}Keywords: Stream-Vorticity Formulation; Finite Element Method; Taylor-Galerkin Scheme; Drug-Eluting Stent; Atherosclerosis.
