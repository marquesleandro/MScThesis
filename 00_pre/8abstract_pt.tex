\begin{center}
\textbf{RESUMO}
\end{center}

% JUST ONE paragraph.
% Number of sheets (NS) = PDF pages - 2 (less cover and 2 first pages, which are printed at the same sheet)

$\!$\\

\hspace{-1.3cm}\textbf{Marques}, Leandro \textit{Simulação Numérica em Elementos Finitos do Escoamento em Artéria Coronária}. xxf. Trabalho de conclusão de curso (Bacharelado em Engenharia Mecânica) - Faculdade de Engenharia, Universidade do Estado do Rio de Janeiro~(UERJ), Rio de Janeiro, 2018.

\vspace{.2cm}

\indent O presente trabalho tem como objetivo desenvolver uma estrutura computacional para simular o escoamento em uma artéria coronária em coordenadas cartesianas. O Método dos Elementos Finitos (MEF) foi usado para resolver as equações de governo do escoamento sanguíneo na artéria coronária com aterosclerose e stent farmacológico implantado. O sangue foi modelado como um fluido monofásico, incompressível e newtoniano. A equação de Navier-Stokes foi apresentada segundo a formulação corrente-vorticidade com o transporte de espécie química. O esquema Taylor-Galerkin foi usado a fim de reduzir as oscilações espúrias que podem ser observadas quando o termo convectivo é predominante.

\vspace{1cm}

\hspace{-1.3cm}Palavras-chave: Formulação Corrente-Vorticidade; Método dos Elementos Finitos; Esquema Taylor-Galerkin; Stent Farmacológico; Aterosclerose.
