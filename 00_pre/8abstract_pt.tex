\begin{center}
\textbf{RESUMO}
\end{center}

% JUST ONE paragraph.
% Number of sheets (NS) = PDF pages - 2 (less cover and 2 first pages, which are printed at the same sheet)

$\!$\\

\hspace{-1.3cm}\textbf{Marques}, Leandro \textit{Um Método FE-ALE para a Formulação Corrente-Vorticidade com a Equação de Transporte de Espécie Química}. xxf. Dissertação de Mestrado (Mestrado em Engenharia Mecânica) - Faculdade de Engenharia, Universidade do Estado do Rio de Janeiro~(UERJ), Rio de Janeiro, Brasil, 2020.

\vspace{.2cm}

\indent 
O presente trabalho tem como objetivo o desenvolvimento 
de uma estrutura computacional para simular o escoamento 
sanguíneo em uma artéria coronária com stent farmacológico 
implantado para diversos números de Schmidt usando a Formulação Corrente-Vorticidade em uma 
abordagem Lagrangeana-Euleriana Arbitrária (ALE).
O fluido sanguíneo foi modelado como um fluido monofásico, incompressível e newtoniano.
A equação de Navier-Stokes é apresentada segundo a formulação 
corrente-vorticidade com a equação de transporte de espécie química.
O Método dos Elementos Finitos (FEM) é usado para resolver as equações 
de governo, onde a formulação de Galerkin foi usada para 
discretização no espaço, enquanto o Método semi-Lagrangeano 
foi usado para discretizar a derivada material usando 
o \textit{backward difference scheme}. 
Além disso, o procedimento de remalhamento
realiza a regularização dos nós possibilitando
simular problemas com contornos móveis e os sistemas lineares foram resolvidos utilizando o 
Método Iterativo Gradientes Conjugados.
Portanto, a nova formulação proposta
nesse trabalho é apropriada para
simular escoamentos em artéria coronária com boa precisão.

\vspace{1cm}

\hspace{-1.3cm}Palavras-chave: Lagrangeana-Euleriana Arbitrária; Formulação Corrente-Vorticidade; Método dos Elementos Finitos; Método semi-Lagrangeano; Stent Farmacológico.
