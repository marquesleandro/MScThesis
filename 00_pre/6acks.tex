\begin{center}
\textbf{AGRADECIMENTOS}
\end{center}
$\!$\\

\medskip 
Este trabalho foi desenvolvido no Laboratório de Ensaios Numéricos (LEN)
do Grupo de Estudos e Simulações Ambientais em Reservatórios (GESAR) da
Universidade do Estado do Rio de Janeiro (UERJ).

\medskip 
Inicialmente gostaria de registrar minha profunda gratidão e admiração ao meu
orientador, Prof. Gustavo Rabello dos Anjos, pelos ensinamentos, instruções e atenção
na elaboração deste trabalho.
Agradecer, também, ao meu co-orientador Prof. José Pontes pelo suporte e desenvolvimento na 
minha formação acadêmica.
Ao Grupo de Pesquisas Gesar, em especial ao Prof. Norberto Mangiavacchi por disponibilizar toda a infra-estrutura
necessária para a elaboração desta pesquisa.

\medskip 
Aos amigos Livia Correa, Danillo Couto, Leonardo Cunha, Carlos Junior, Lucian Joshua, Alex Luiz,
Alan Luiz, Danilo Franco, Igor de Azevedo, Thiago Manhães,
Giovanno Victor e Caio Lemos por contribuirem de diversas maneiras a completar este objetivo.

\medskip 
À minha família por todo o encorajamento e suporte. 
Aos meus pais, Ana Maria Marques e Edson Santos, pelo cuidado
e incentivo. Ao meu irmão Pedro Marques por toda a amizade e
suporte. Aos meus sogros, Marcelo Marques e Andreia Vidal,
por todo auxilio e por sempre acreditarem em mim.

\medskip 
Por fim, à minha esposa, Leticia Marques, a quem este trabalho é dedicado,
por toda atenção, incentivo e companheirismo que tem depositado todos os
dias.

\medskip 
Em especial, ao Criador, a quem emana toda a fonte de inspiração e existência.
