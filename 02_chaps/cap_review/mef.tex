The mathematical basis for the Finite Element Method begins in 1909
 with Ritz \cite{ritz1909} in which a continuous problem is replaced by 
a discrete problem with a finite number of degrees of freedom
 where the unknowns were approximated by the product between
 the constants and the base functions chosen
 in order to guarantee the accuracy of the result.
 This procedure is known as \textit{variational formulation}.

\medskip
 Years later, Galerkin (1915) \cite{galerkin1915} uses the Weighted Residual Method
 to determine the constants of the variational formulation
 where the same base functions were used in the weight functions.
 This procedure is known as \textit{Galerkin formulation} and
 is widely used nowadays.

\medskip
During the 1940s, Courant (1943) \cite{courant1943} applied
 variational formulation to a domain discretized by triangular elements.
 In 1965, Zienkicwicz and Cheung \cite{zienkiewicz1965}
 show that the Weighted Residual Method has a good approximation
 of the solution and the Finite Element Method was formalized
 to solve several problems. The proposed mathematical approach
 is often used today.

\medskip
The Finite Element Method has become a very effective tool
 in the solution of several problems and it has been widely
 used in problems of the solids mechanics.
 In fluid mechanics, however, its use became possible only
 later due to the spurious oscillations that it can be seen
 when the convective term is superior to the diffusive term.
 Such oscillations are present not only in the Finite Element Method
 but were also observed in the Finite Difference Method by
 Spalding in 1972 \cite{spalding1972} where it is shown that
 the \textit{upwind} effect helped to reduce these oscillations.

\medskip
In 1976, Christie et al. \cite{christie1976} modify weight
 functions for asymmetric or quadratic functions to reduce
 spurious oscillations in one-dimensional diffusion-convection problems.
 Such modifications produced a \textit{upwind} effect on the solution.
 This procedure became known as \textit{Petrov Galerkin Formulation}.
 In the following year, Heinrich, Huyakorn and Zienkiewicz
 \cite{heinrich1977} generalize the scheme to a two-dimensional problem.
 The global matrices, however, became asymmetrical differently
 from those presented in the Galerkin scheme.

\medskip
In 1982, Brooks and Hughes \cite{brooks1982} proposed a
 new formulation that consists of the weight functions modified
 in order to the diffusion operator acts only in the flow direction.
 This procedure aims to eliminate the excess of
 diffusion perpendicular to the flow that
 the Petrov-Galerkin scheme presented in some cases.
 The formulation does not require the use of high-order
 weight functions and was efficient in eliminating perpendicular
 diffusion. The formulation received the name
 \textit{Streamline Upwind Petrov-Galerkin} (SUPG).


\medskip
In the same year, Pironneau \cite{pironneau1982} presented
 the \textit{Characteristic trajectory}
 applied to the
 Finite Element Method in solving the non-steady convection-diffusion
 and Navier-Stokes equations. Thereby, the author was able
 to derive conservative schemes of the type \textit{upwind}
 with first and second order accurate. As the matrices are symmetric,
 this scheme proved to be advantageous in solving linear
 systems compared to other \textit{upwind} schemes.
 In addition to the method is unconditionally stable.
 The numerical implementation, however, requires an complex searching
 procedure. This method is known as \textit{semi-Lagrangian} Method.

\medskip
In 1984, Donea \cite{donea1984} presents an alternative for
 solving multidimensional and transient convection-diffusion
 problems. This alternative is known as the
 \textit{Taylor-Galerkin} scheme. The scheme consists of using
 the high-order terms of the Taylor expansion to reduce
 spurious oscillations. Unlike upwind schemes, in the
 Taylor-Galerkin scheme there is no need to use modified
 weight functions. The scheme is compared with the formulations
 of Galerkin and Petrov Galerkin and it showed high accuracy and
 low numerical diffusion. 
Although the computational implementation is easy,
the Taylor-Galerkin method
 is conditionally stable.

\medskip
In the same year, Lohner et al. \cite{lohner1984} 
proposed a simpler alternative
to avoid the complex searching procedure in the characteristic
methods. This alternative involves to use the high-order terms of
the Taylor expansion to approximate the departure point.
The procedure is known as \textit{Characteristic Galerkin}.
The main advantages of method are the symmetric global matrices
as well as no searching procedure for each time step. However, the
method is conditionally stable.
Although the Taylor Galerkin and
 Characteristic Galerkin discretization procedures are distinct,
 the system of equations is the same for the convection-diffusion
 equation, where the unknown is a scalar.


\medskip
Several researchers have analyzed the stability and convergencei
 of these methods as the paper of Donea and Quartapelle (1992)
\cite{donea1992}. In this paper, the authors present an analysis
of several methods for solving unsteady problems governed by
advection equations. 
Whereas the Petrov-Galerkin and
 SUPG schemes have the asymmetric matrix system, 
 the Taylor-Galerkin, Characteristic Galerkin and
 semi-Lagrangian methods have the advantage of matrices were
 symmetric facilitating computational implementation.
 In addition to the Taylor-Galerkin and Characteristic Galerkin methods
 have a simpler implementation and they can achieve a high order
accuracy, but they are only conditionally stable.
 For large time step, unlike, the semi-Lagrangian method is
 more efficient because its unconditional stability.
 However, the main disadvantage of this method is the complex 
 searching procedure. It may lead to excessive computational cost
 if it is not well designed.

\medskip
All the methods presented have satisfactory results and
they are well known in the literature.
 These methods, therefore, made it possible to solve
 convective problems using the Finite Element Method
and they are presents in several complex problems today, 
such as the numerical simulation for two-phase flows with
dynamic boundaries presented
 by Anjos, Mangiavacchi and Thome (2020) \cite{anjos2020}.
 For current work, the semi-Lagrangian method was chosen for
 decrease spurious oscillations, due to its unconditional stability
 and the symmetric linear system of the governing equation in an finite
 element context. 

