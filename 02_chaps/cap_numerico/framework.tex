For the numerical code framework, several classes were created in 
order to reuse the code in further simulations. 
Initially, the triangular unstructured mesh generated by GMSH \cite{gmsh}
was imported into the numerical code by the \textit{ImportMSH} class.
This class returns important information for the simulation 
such as:
\textit{nodes number} (\textbf{np}), 
\textit{elements number} (\textbf{ne}), 
\textit{the coordinate vectors} (\textbf{x} and \textbf{y}), 
\textit{the connectivity matrix} (\textbf{IEN}),
\textit{the neighbors nodes} and 
\textit{the Boundary nodes}. 
The \textit{ImportMSH} class is 
enabled to import the linear, quadratic or cubic triangular elements.
The \ref{tempo malha} shows the average processing time for 
mesh import in several unstructured linear triangular meshes.

\vspace{0.5cm}
\begin{table}[H]
\centering
\begin{tabular}{ccc}
\toprule
\textbf{Nodes} & \textbf{Elements} & \textbf{AVG Processing Time} (s) \\
\midrule
10482 & 20142 & 0,6 \\
40819 & 80005 & 2,6 \\
249677 & 495289 & 16,6 \\
993091 & 2010501 & 70,4 \\
\bottomrule
\end{tabular}
\caption{Average processing time for mesh import in several unstructured linear triangular meshes}
\label{tempo malha}
\end{table}

\medskip
After importing the \textit{.msh} file, the elementary and global arrays 
were assembled. 
They are created by the \textit{GaussianQuadrature} class, where it is enabled to assemble the linear, quadratic or cubic triangular element.
For the linear 
triangular element, it is also possible to use elementary 
analytical matrices. For more details consult the work of Lewis, 
Nithiarasu and Seetharamu (2004) \cite{lewis2004}.

\medskip
The global matrix assembly was performed by \textit{Assemble} class, satisfying the local and global matrices index correspondence.
They were initialized as sparse matrices by 
the \textit{Scipy} library \cite{scipy} and 
the \ref{tempo matrizes globais} shows the average processing time 
for global matrices assembly in several unstructured linear 
triangular meshes.

\vspace{0.5cm}
\begin{table}[H]
\centering
\begin{tabular}{ccc}
\toprule
\textbf{Nodes} & \textbf{Elements} & \textbf{AVG Processing Time} (s) \\
\midrule
10482 & 20142 & 72,9 \\
40819 & 80005 & 254,3 \\
249677 & 495289 & 1664,9 \\
993091 & 2010501 & 69059,9 \\

\bottomrule
\end{tabular}
\caption{Average processing time for global matrices assembly in several unstrutured linear triangular meshes}
\label{tempo matrizes globais}
\end{table}
 
