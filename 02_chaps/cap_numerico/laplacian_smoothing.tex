As previously mentioned, the computational mesh velocity assumes 
different values of velocity from the Eulerian and Lagrangean 
description. Thus, it can be represented as a linear 
combination of other speeds, such as:

\begin{equation}
\hat{\texbf{v}} 
= \beta_{1}\textbf{v_{1}}
+ \beta_{2}\textbf{v_{2}}
\end{equation}

where,
$\textbf{v_{1}}$ is the Lagrangian velocity,
$\textbf{v_{2}}$ is the Laplacian smoothing velocity,
$\beta_{1}$ is a parameters controls the Lagragian motion and
.

The choice of these speeds and their parameters must 
be in order to improve the result of the numerical 
simulation and to avoid the degradation of the 
computational elements, especially near the domain boundary, 
where the elements are compressed making the simulation unstable.

Lagrangean velocity is the material flow velocity. 
This portion makes the nodes of the computational 
mesh move in the same direction and feeling as the flow. 
The intensity is proportional to the value of parameter B1, 
depending on the chosen value, the insertion and deletion 
of nodes are required.

The speed of Laplacian smoothing is the speed that nodes 
acquire due to their topological redistribution. 
Considering a knot in a non-uniform mesh, it will be 
moved so that it is centered in consideration of the 
neighbors' 1-ring. With this, it is expected that the 
mesh is smoothed converging to a more uniform 
distribution, as we can see in figure 2.

According to [22] ... copy

 EQ

It is possible to choose several strategies to calculate 
the weight of this equation. In this work, the weight was 
calculated as 1 / e, where e is the distance ... copy. 
Thus the speed due to Laplacian smoothing can be calculated as:

EQ
