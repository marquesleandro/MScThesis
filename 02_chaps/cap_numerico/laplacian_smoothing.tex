As previously mentioned, the computational mesh velocity assumes 
different values of velocity from the Eulerian and Lagrangean 
description. Thus, it can be represented as a linear 
combination of other velocities, such as:

\begin{equation}
\hat{\textbf{v}} 
= \beta_{1} \textbf{v}_{1}
+ \beta_{2} \textbf{v}_{2}
\end{equation}

\medskip
\noindent
where,
$\textbf{v}_{1}$ is the Lagrangian velocity,
$\textbf{v}_{2}$ is the Laplacian smoothing velocity,
$\beta_{1}$ is a parameter controls the Lagragian motion and
$\beta_{2}$ controls the intensity of velocity smoothing.
The choice of these velocity fields and their parameters must 
be in order to improve the result of the numerical 
simulation and to avoid the degradation of the 
computational elements, especially close to the domain boundary, 
where the elements are compressed making the simulation unstable.

\medskip
Lagrangian velocity is the material flow velocity. 
This portion makes the nodes of the computational 
mesh move in the same direction and sense as the flow velocity. 
The intensity is proportional to the value of parameter $\beta_{1}$.
Depending on the chosen value, the insertion and deletion 
of nodes are required.

\medskip
The velocity of Laplacian smoothing is that nodes 
acquire due to their topological redistribution. 
Considering a node in a non-uniform mesh, it will be 
moved to the centroid of the 1-ring neighbors. 
Thereby, it is expected the smoothing procedure
converges to a more uniform point distribution,
as shown in figure 2.

FIG

\medskip
According to [22], the new node position 
$\hat{\textbf{x}}_{i}$ can be approximated using an 
iterative weighted sum
of the 1-ring neighbors of a node:

\begin{equation}
\hat{\textbf{x}}_{i} 
= \sum_{i}^{np} \sum_{j}^{N_1} w_{ij}
\left( \textbf{x}_{j} - \textbf{x}_{i} \right)
\end{equation}

\medskip
\noindent
where, 
$w_{ij}$ is the weight and it can be calculated in several ways,
$N_{1}$ is the 1-ring neighbors of a node,
$\textbf{x}_{j}$ is the coordinate vector of neighbor node and
$\textbf{x}_{i}$ is the coordinate vector of node in previous step that will be moved.
It is possible to choose several strategies to calculate 
the weight of this equation. In this work, the weight was 
calculated as the inverse distance from its neighbor vertices,
that is:
\begin{equation}
w_{ij} = \sum_{i}^{np} \sum_{j}^{N_1}
\frac{1}{\left( \textbf{x}_{j} - \textbf{x}_{i} \right)}
\end{equation}

\medskip
\noindent
Therefore, the Laplacian smoothing velocity can be calculated as:

\begin{equation}
\textbf{v}_{2}
= \sum_{i}^{np} \sum_{j}^{N_1} w_{ij}
\frac{\left( \textbf{x}_{j} - \textbf{x}_{i} \right)}{\Delta t}
\end{equation}


