In this chapter, we will present the main characteristics 
of the computational code developed in Python 2.7 \cite{python} 
using the object-oriented paradigm (OOP) in order to reuse the code 
in other simulations in the future. All developed classes are imported 
into the simulator (\textit{TriSim}), where the result of the 
numerical simulation is exported as presented in the 
simplified \textit{Class Diagram} (UML) of \ref{uml}. 
Initially, the \textit{script} that performs the import 
of the computational mesh for the simulation is presented. 
Then, the assembly of the global matrices is done 
respecting the correspondence between the global and local index. 
Later on, we present the application of boundary conditions for 
both \textit{Dirichlet} and \textit{Neumann}. 
Finally, the solve algorithm for the vorticity-streamfunction 
formulation with the species transport equation is presented.

\vspace{0.5cm}
\begin{figure}[H]
\begin{center}
\begin{tikzpicture}[scale=0.8]
 \draw [line width=1pt] (0,0) rectangle ++(3,3);
 \draw [line width=1pt] (0,2.4) -- (3,2.4);
 \draw [line width=1pt] (0,1.6) -- (3,1.6);
 \node (Simulator) at (1.5,2.7) {\small \textbf{Simulator}};
 \node[align=left] (Simulator) at (1.0,0.8) {\scriptsize + solver()\\[-12pt]
                                              \scriptsize + relatory()\\[-12pt]                                                        

};




 \draw [line width=1pt] (-3.2,5.5) rectangle ++(3.8,3.6);
 \draw [line width=1pt] (-3.2,8.4) -- (0.6,8.4);
 \draw [line width=1pt] (-3.2,7.6) -- (0.6,7.6);
 \node (ImportMSH) at (-1.3,8.7) {\small \textbf{ImportMSH}};
 \node[align=left] (ImportMSH) at (-1.5,6.3) {\scriptsize + coordinates()\\[-12pt]
                                          \scriptsize + IEN()\\[-12pt]                                                        
                                          \scriptsize + neighborsNodes()\\[-12pt]                                                        
                                          \scriptsize + boundaryNodes()\\[-12pt]                                                        

};


 \draw [line width=1pt] (3.2,5.5) rectangle ++(4.1,3.6);
 \draw [line width=1pt] (3.2,8.4) -- (7.3,8.4);
 \draw [line width=1pt] (3.2,7.6) -- (7.3,7.6);
 \node (MeshUpdate) at (5.3,8.7) {\small \textbf{MeshUpdate}};
 \node[align=left] (meshUpdate) at (5.2,6.3) {\scriptsize + lagrangian()\\[-12pt]
                                          \scriptsize + laplacianSmoothing()\\[-12pt]                                                        
                                          \scriptsize + meshVelocity()\\[-12pt]                                                        
                                          \scriptsize + moveMesh()\\[-12pt]                                                        

};





 \draw [line width=1pt] (5.8,1) rectangle ++(3.2,3.2);
 \draw [line width=1pt] (5.8,3.4) -- (9,3.4);
 \draw [line width=1pt] (5.8,2.6) -- (9,2.6);
 \node (ExportVTK) at (7.5,3.7) {\small \textbf{ExportVTK}};
 \node[align=left] (ExportVTK) at (6.7,1.5) {\scriptsize + scalar()\\[-12pt]
                                          \scriptsize + vector()\\[-12pt]                                                        
                                          \scriptsize + save()\\[-12pt]                                                        
};
 


 \draw [line width=1pt] (5.8,-4) rectangle ++(4.3,3.2);
 \draw [line width=1pt] (5.8,-1.6) -- (10.1,-1.6);
 \draw [line width=1pt] (5.8,-2.4) -- (10.1,-2.4);
 \node (SemiLagrangian) at (8.1,-1.2) {\small \textbf{SemiLagrangian}};
 \node[align=left] (SemiLagrangian) at (7.0,-3) {\scriptsize + searching()\\[-12pt]
};
 



 \draw [line width=1pt] (-6.0,-1.9) rectangle ++(3.1,4.9);
 \draw [line width=1pt] (-6,2.4) -- (-3.1,2.4);
 \draw [line width=1pt] (-6,1.6) -- (-3.1,1.6);
 \node (BCApply) at (-4.5,2.7) {\textbf{\small BCApply}};
 \node[align=left] (BCApply) at (-4.5,-0.4) {\scriptsize + couette()\\[-12pt]
                                          \scriptsize + poiseulle()\\[-12pt]                                                        
                                          \scriptsize + halfpoiseulle()\\[-12pt]                                                        
                                          \scriptsize + cavity()\\[-12pt]                                                        
                                          \scriptsize + backwardStep()\\[-12pt]                                                        
                                          \scriptsize + pulsation()\\[-12pt]                                                        
                                          \scriptsize + stent()\\[-12pt]                                                        

};

 \draw [line width=1pt] (-0.1,-7) rectangle ++(3.1,4.1);
 \draw [line width=1pt] (-0.1,-3.6) -- (3,-3.6);
 \draw [line width=1pt] (-0.1,-4.4) -- (3,-4.4);
 \node (Assembly) at (1.5,-3.3) {\small \textbf{Assembly}};
 \node[align=left] (Assembly) at (1.0,-6.0) {\scriptsize + analytic()\\[-12pt]
                                            \scriptsize + linear()\\[-12pt]                                                        
                                            \scriptsize + quad()\\[-12pt]                                                        
                                            \scriptsize + mini()\\[-12pt]                                                        
                                            \scriptsize + cubic()\\[-12pt]                                                        

};





 \draw [line width=1pt] (-6,-6.1) rectangle ++(3.9,3.1);
 \draw [line width=1pt] (-6,-3.6) -- (-2.1,-3.6);
 \draw [line width=1pt] (-6,-4.4) -- (-2.1,-4.4);
 \node (GaussianQuad) at (-4.1,-3.3) {\small \textbf{GaussianQuad}};
 \node[align=left] (GaussianQuad) at (-4.6,-5.5) {\scriptsize + getMass()\\[-12pt]
                                                 \scriptsize + getStiffness()\\[-12pt]                                                        
                                                 \scriptsize + getGradient()\\[-12pt]                                                        

};




 %ImportMSH
 \draw [{Diamond}-,line width=1pt] (1.0,3.0) -- (1.0,4.0);
 \draw [line width=1pt] (1.0,4.0) -- (-1.0,4.0);
 \draw [line width=1pt] (-1.0,4.0) -- (-1.0,5.5) ;

 %MeshUpdate
 \draw [{Diamond}-,line width=1pt] (2.0,3.0) -- (2.0,4.0);
 \draw [line width=1pt] (2.0,4.0) -- (5.0,4.0);
 \draw [line width=1pt] (5.0,4.0) -- (5.0,5.5) ;


 %BCApply
 \draw [{Diamond}-,line width=1pt] (0,1.5) -- (-1.5,1.5);
 \draw [line width=1pt] (-1.5,1.5) -- (-1.5,0.1);
 \draw [line width=1pt] (-1.5,0.1) -- (-2.9,0.1) ;


 %ExportVTK
 \draw [{Diamond}-,line width=1pt] (3,1.5) -- (4.5,1.5);
 \draw [line width=1pt] (4.5,1.5) -- (4.5,2.5);
 \draw [line width=1pt] (4.5,2.5) -- (5.8,2.5) ;

 %SemiLagrangian
 \draw [{Diamond}-,line width=1pt] (2,0) -- (2,-2);
 \draw [line width=1pt] (2,-2) -- (5.8,-2);


 %Assembly
 \draw [{Diamond}-,line width=1pt] (1.0,0) -- (1.0,-2.9);

 %GaussianQuad
 \draw [{Diamond}-,line width=1pt] (-0.1,-5.0) -- (-2.1,-5.0);

\end{tikzpicture}
\end{center}
\caption{Simplified Class Diagram}
\label{uml}
\end{figure}


