Após a importação do arquivo \textit{.msh}
foi realizado a montagem das matrizes globais.
As mesmas foram inicializadas como matrizes
esparsas pela biblioteca \textit{Scipy} \cite{scipy}
e o seguinte \textit{script} foi usado para a montagem:


%\begin{algorithm}[H]
%\caption{Assembly Global Matrix}
%\begin{algorithmic}
% \For {e in range(0,ne)}
%   \State Linear\_Element\\
%   \For {i in range(0,3)}
%    \State ii = IEN[e][i]\\
%    \For {j in range(0,3)}
%     \State jj = IEN[e][j]\\
%     \State Kxx[ii][jj] += kxx\_element[i][j]
%     \State Kxy[ii][jj] += kxy\_element[i][j]
%    \State Kyx[ii][jj] += kyx\_element[i][j]
%    \State Kyy[ii][jj] += kyy\_element[i][j]
%    \State Gx[ii][jj] += gx\_element[i][j]
%    \State Gy[ii][jj] += gy\_element[i][j]
%    \State M[ii][jj] += mass\_element[i][j]\\
%   \EndFor 
%  \EndFor 
% \EndFor 
%\end{algorithmic}
%\end{algorithm}


\begin{verbatim}
__________________________________________________________________________
for e in range(0, ne):                 | loop sobre os elementos 
 linear_element(e)                     | montagem das matrizes elementares
                                         utilizando a quadratura gaussiana
 for i in range(0,3):                  
  ii = IEN[e][i]                       
  
  for j in range(0,3):                  
   jj = IEN[e][j]
                                       
   Kxx[ii,jj] += kxx_element[i][j]     |
   Kxy[ii,jj] += kxy_element[i][j]     |
   Kyx[ii,jj] += kyx_element[i][j]     |
   Kyy[ii,jj] += kyy_element[i][j]     | montagem das matrizes globais
                                       | correspondendo os índices globais
   Gx[ii,jj] += gx_element[i][j]       | e locais
   Gy[ii,jj] += gy_element[i][j]       | 
                                       |
   M[ii,jj] += mass_element[i][j]      |
__________________________________________________________________________
\end{verbatim}


\medskip
A montagem das matrizes elementares é feita pelo módulo 
\textit{linear\_element} cujo paramêtro requerido é o 
número do elemento. Esse módulo faz parte da classe \textit{TElement}
onde utiliza a quadratura gaussiana para o cálculo dos
valores das matrizes elementares. Para o elemento triangular
linear, é possível a utilização das matrizes elementares
analíticas. Para mais detalhe consultar o trabalho de Lewis,
Nithiarasu e Seetharamu (2004) \cite{lewis2004}.

\newpage
Em seguida, a matriz do lado esquerdo conhecida como 
\textit{left hand side (LHS)} é criada para as
equações da função de corrente,
velocidade e concentração respectivamente:

\begin{verbatim}
__________________________________________________________________________
LHS_psi = sps.lil_matrix.copy(K)
LHS_vx = sps.lil_matrix.copy(M)
LHS_vy = sps.lil_matrix.copy(M)
LHS_c = sps.lil_matrix.copy(M)/dt
__________________________________________________________________________
\end{verbatim}

A matriz \textit{LHS} para a equação da vorticidade
é criada durante o loop do algoritmo de solução
a fim de garantir que a mesma será sempre inicializada
utilizando as matrizes globais originais.
É necessário usarmos a função \textit{copy}
porque queremos copiar os valores das matrizes globais
e não referencia-los,
para mais detalhe consultar o \textit{Scipy Community} \cite{numpycopy}.
A \ref{tempo matrizes globais} apresenta o tempo de processamento para 
para a montagem das matrizes globais em diversas
malhas triangulares lineares não estruturadas.

\vspace{0.5cm}
\begin{table}[H]
\centering
\begin{tabular}{ccc}
\toprule
\textbf{N. Nós} & \textbf{N. Elementos} & \textbf{Tempo de Processamento} (s) \\
\midrule
10482 & 20142 & 72,9 \\
40819 & 80005 & 254,3 \\
249677 & 495289 & 1664,9 \\
993091 & 2010501 & 69059,9 \\

\bottomrule
\end{tabular}
\caption{Tempo de montagem das matrizes globais para diversas malhas triangulares não estruturadas}
\label{tempo matrizes globais}
\end{table}
                
