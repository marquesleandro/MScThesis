After importing the \textit{.msh} file, the global arrays 
were assembled. They were initialized as sparse matrices by 
the \textit{Scipy} library \cite{scipy} and the following 
\textit{script} was used for the assembly:

%\begin{algorithm}[H]
%\caption{Assembly Global Matrix}
%\begin{algorithmic}
% \For {e in range(0,ne)}
%   \State Linear\_Element\\
%   \For {i in range(0,3)}
%    \State ii = IEN[e][i]\\
%    \For {j in range(0,3)}
%     \State jj = IEN[e][j]\\
%     \State Kxx[ii][jj] += kxx\_element[i][j]
%     \State Kxy[ii][jj] += kxy\_element[i][j]
%    \State Kyx[ii][jj] += kyx\_element[i][j]
%    \State Kyy[ii][jj] += kyy\_element[i][j]
%    \State Gx[ii][jj] += gx\_element[i][j]
%    \State Gy[ii][jj] += gy\_element[i][j]
%    \State M[ii][jj] += mass\_element[i][j]\\
%   \EndFor 
%  \EndFor 
% \EndFor 
%\end{algorithmic}
%\end{algorithm}


\begin{verbatim}
__________________________________________________________________________
for e in range(0, ne):                      | element loop 
 linear_element(e)                          | elementary matrix
                                              by gaussian quadrature

 for i in range(0,3):                       
  ii = IEN[e][i]                            
  
  for j in range(0,3):                       
   jj = IEN[e][j]
                                            
   Kxx[ii,jj] += kxx_element[i][j]          |
   Kxy[ii,jj] += kxy_element[i][j]          |
   Kyx[ii,jj] += kyx_element[i][j]          |
   Kyy[ii,jj] += kyy_element[i][j]          | globals matrices assembly
                                            | with global and local 
   Gx[ii,jj] += gx_element[i][j]            | index correspondence
   Gy[ii,jj] += gy_element[i][j]            | 
                                            |
   M[ii,jj] += mass_element[i][j]           |
__________________________________________________________________________
\end{verbatim}


\medskip
The assembly of the elementary matrices is done by the module 
\textit{linear\_element} whose required parameter is the 
element number. This module is part of the \textit{TElement} 
class where it uses the Gaussian Quadrature to calculate 
the values of the elementary matrices. For the linear 
triangular element, it is also possible to use elementary 
analytical matrices. For more details consult the work of Lewis, 
Nithiarasu and Seetharamu (2004) \cite{lewis2004}.

\newpage
Then, the left hand matrix known as \textit {left hand side (LHS)} 
is created for the streamfunction, velocity and concentration equations 
respectively:

\begin{verbatim}
__________________________________________________________________________
LHS_psi = sps.lil_matrix.copy(K)
LHS_vx = sps.lil_matrix.copy(M)
LHS_vy = sps.lil_matrix.copy(M)
LHS_c = sps.lil_matrix.copy(M)/dt
__________________________________________________________________________
\end{verbatim}

The \textit{LHS} matrix for the vorticity equation is created 
during the solution algorithm loop to ensure that it will always 
be initialized using the original global matrices. 
It is necessary to use the function \textit{copy} because 
we want to copy the values of the global arrays and not reference them, 
for more details consult the \textit{Scipy Community} \cite{numpycopy}. 
The \ref{tempo matrizes globais} shows the average processing time 
for global matrices assembly in several unstructured linear 
triangular meshes.


\vspace{0.5cm}
\begin{table}[H]
\centering
\begin{tabular}{ccc}
\toprule
\textbf{Nodes} & \textbf{Elements} & \textbf{AVG Processing Time} (s) \\
\midrule
10482 & 20142 & 72,9 \\
40819 & 80005 & 254,3 \\
249677 & 495289 & 1664,9 \\
993091 & 2010501 & 69059,9 \\

\bottomrule
\end{tabular}
\caption{Average processing time for global matrices assembly in several unstrutured linear triangular meshes}
\label{tempo matrizes globais}
\end{table}
                
