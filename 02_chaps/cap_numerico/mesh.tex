O domínio consiste em uma malha não estruturada
gerada pelo software livre \textit{GMSH} proposto por Geuzaine e Remacle (2009) \cite{gmsh} e foi importada
à simulação pela classe \textit{TriMesh}.
Inicialmente transformamos o arquivo \textit{.msh}
em uma lista \textit{Python} pelo script abaixo:

\begin{verbatim}
__________________________________________________________________________
malha = []                             | inicialização da lista Python
with open("arquivo.msh") as mesh:       
 for line in mesh:                     | loop sobre as linhas do arquivo
  row = line.split()                   
  malha.append(row[:])                 | adicionando a linha
                                       | do arquivo na lista Python
__________________________________________________________________________
\end{verbatim}


Em seguida, esta classe retorna informações
importantes para a simulação tais como: 
\textit{o número de nós do domínio} (\textbf{np}),
\textit{o número de elementos do domínio} (\textbf{ne}),
\textit{os vetores de coordenadas} (\textbf{x} e \textbf{y}), 
\textit{a matriz de conectividade} (\textbf{IEN}) e 
\textit{os nós contidos no contorno do domínio}.
A \ref{tempo malha} apresenta o tempo de processamento para a 
importação da malha em diversas
malhas triangulares lineares não estruturadas.

\vspace{0.5cm}
\begin{table}[H]
\centering
\begin{tabular}{ccc}
\toprule
\textbf{N. Nós} & \textbf{N. Elementos} & \textbf{Tempo de Processamento} (s) \\
\midrule
10482 & 20142 & 0,6 \\
40819 & 80005 & 2,6 \\
249677 & 495289 & 16,6 \\
993091 & 2010501 & 70,4 \\
\bottomrule
\end{tabular}
\caption{Tempo de importação para diversas malhas triangulares não estruturadas}
\label{tempo malha}
\end{table}

\newpage
