The domain consists of an unstructured mesh generated by 
the \textit{GMSH} open-source software proposed by 
Geuzaine and Remacle (2009) \cite{gmsh} and 
it was imported into the numerical simulation by 
the \textit{TriMesh} class. 
Initially, we used a well-known computational community script to 
convert the \textit{.msh} file to a 
\textit{Python} list:

\begin{verbatim}
__________________________________________________________________________
mesh = []                             | Python list inicialization
with open("file.msh") as mesh:        | Open file 
 for line in mesh:                    | Loop file each line
  row = line.split()                  | Split file line
  mesh.append(row[:])                 | Add file line from python list 
__________________________________________________________________________
\end{verbatim}

\noindent
Then, this class returns important information for the simulation 
such as:
\textit{Nodes number} (\textbf{np}), 
\textit{Elements number} (\textbf{ne}), 
\textit{The coordinate vectors} (\textbf{x} and \textbf{y}), 
\textit{The connectivity matrix} (\textbf{IEN}) and 
\textit{The Boundary nodes}. 
The \ref{tempo malha} shows the average processing time for 
several unstructured linear triangular meshes import.

\vspace{0.5cm}
\begin{table}[H]
\centering
\begin{tabular}{ccc}
\toprule
\textbf{Nodes} & \textbf{Elements} & \textbf{AVG Processing Time} (s) \\
\midrule
10482 & 20142 & 0,6 \\
40819 & 80005 & 2,6 \\
249677 & 495289 & 16,6 \\
993091 & 2010501 & 70,4 \\
\bottomrule
\end{tabular}
\caption{Average processing time for several unstructured linear triangular mesh import}
\label{tempo malha}
\end{table}

\newpage
