The main difficulty in implementation of 
vorticity-streamfunction formulation 
is the boundary conditions of the vorticity. 
Briefly, the solve algorithm used was:

\vspace{0.5cm}
% Define block styles
\tikzstyle{block} = [rectangle, draw, fill=gray!10!,
    text width=25em, text centered, draw,scale=0.7,text=black!90!]
\tikzstyle{line} = [draw, -latex',scale=0.75]



\begin{figure}[h!]
\begin{center}
\begin{tikzpicture}[node distance = 1.0cm,auto]
    % Place nodes
    \node [block] (step1) {Import mesh};
    \node [block, below of=step1] (step2) {Calculate Gaussian Quadradure and Assemble Matrix};
    \node [block, below of=step2] (step3) {Inicialize Vorticity and Streamfunction};
    \node [block, below of=step3] (step4) {Calculate Laplacian smoothing and ALE velocity};
    \node [block, below of=step4] (step5) {Move Nodes};
    \node [block, below of=step5] (step6) {Calculate Gaussian Quadradure and Assemble Matrix};
    \node [block, below of=step6] (step7) {Calculate vorticity boundary condition};
    \node [block, below of=step7] (step8) {Calculate semi-Lagrangian and Vorticity field};
    \node [block, below of=step8] (step9) {Calculate streamfunction field};
    \node [block, below of=step9] (step10) {Calculate velocity field};
    \node [block, below of=step10] (step11) {Calculate concentration field};
    \node [right of=step4, node distance=4cm] (initialLoop) {};
    \node [right of=step11, node distance=4cm] (finalLoop) {};
    \node [right of=step7, node distance=5.5cm] (textLoop) {};

    \node [draw=none, align=center,scale=0.7,text=black!80!] at (textLoop) {Repeat the procedure \\ for the next time step \\ until the steady state};
    % Draw edges
    \path [line] (step1) -- (step2);
    \path [line] (step2) -- (step3);
    \path [line] (step3) -- (step4);
    \path [line] (step4) -- (step5);
    \path [line] (step5) -- (step6);
    \path [line] (step6) -- (step7);
    \path [line] (step7) -- (step8);
    \path [line] (step8) -- (step9);
    \path [line] (step9) -- (step10);
    \path [line] (step10) -- (step11);
    \path [line,dashed] (step11) -- (finalLoop) -- (initialLoop) |- (step4);
\end{tikzpicture}
\end{center}
\caption{Solve algorithm for Vorticity-Streamfunction Formulation with Species Transport Equation}
\label{solution algorithm} 
\end{figure}  


\medskip
In order to facilitate its implementation in other research, 
we will describe each step of the algorithm in detail. 
The equations are in their matrix form:

\begin{enumerate}
 \item \textbf{Inicialize Vorticity field:}
 \begin{equation}
  Mw = G_{x}v - G_{y}u \notag
 \end{equation}

 \item \textbf{Inicialize Streamfuntion field:}
  \begin{equation}
  \Big[ K_{xx} + K_{yy} \Big] \psi = Mw \notag
  \end{equation}
  \textit{
It is necessary to apply the boundary conditions of $\psi $ in 
the LHS matrix, its lines and columns must be zeros and 
the contribution of the cboundary indices in the vector to the right of the equation, as explained in the section \ ref {tricond}.}


 \item \textbf{Calcular as condições de contorno da vorticidade utilizando a equação:}
 \begin{equation}
  Mw = G_{x}v - G_{y}u \notag
 \end{equation}
 \textit{Após resolvermos esta equação, possuímos os valores de $w$ para todos
 os nós do domínio, mas usaremos apenas os nós do contorno
 para zerar as linhas e as colunas da matriz da equação da vorticidade 
 e sua contribuição no lado direito. Deve-se garantir que a matriz LHS 
 seja inicializada em sua forma original em cada passo de tempo. 
 Para isso, é feito a cada passo de tempo:}

\begin{verbatim}
_____________________________________________________________________
LHS_w = sps.lil_matrix.copy(M)/dt
_____________________________________________________________________
\end{verbatim}

 \textit{O script para zerar as linhas e as colunas é
 semelhante à aplicação da condição de Dirichlet
 que foi explicado na seção \ref{tricond},
 exceto à utilização do vetor auxiliar bc\_1 que
 foi substituído pelo $w$ calculado na
 equação $Mw = G_{x}v - G_{y}u$}

\begin{verbatim}
_____________________________________________________________________
for mm in ibc:                   | loop sobre os nós do contorno de w
 bc_dirichlet = LHS[:,mm]*w[mm]  | o vetor bc_1 é substituído por w
 LHS[:,mm] = 0.0                     
 LHS[mm,:] = 0.0                     
 LHS[mm,mm] = 1.0                    
 bc_dirichlet[mm] = w[mm]        | o vetor bc_1 é substituído por w
_____________________________________________________________________
\end{verbatim}

 \item \textbf{Calcular a vorticidade pela equação:}
  \begin{equation}
  \begin{aligned}
   \frac{M}{\Delta t} w^{n+1}
   =  \frac{M}{\Delta t} w^{n}
   - u \cdot G_x w^{n}
   &- v \cdot G_y w^{n} 
   - \frac{1}{\textit{Re}} \Big[ K_{xx} + K_{yy} \Big] w^{n}
   \\
   & - u
   \frac{\Delta t}{2}
   \big[
   u K_{xx}
   + v K_{yx}
   \big]
   w^{n} 
   - v
   \frac{\Delta t}{2}
   \Big[
   u K_{xy}
   + v K_{yy}
   \Big]
   w^{n} \notag
  \end{aligned}
  \end{equation}
  \textit{onde $w^{n}$ é a vorticidade calculada no passo de tempo anterior 
  e $w^{n+1}$ é a vorticidade que será calculada no passo de tempo em análise. 
  É necessário aplicar as condições de contorno de $w$ na matriz a esquerda da equação
  zerando suas linhas e colunas e a contribuição das colunas dos índices de contorno 
  no vetor a direita da equação,
  como foi explicado no passo 3.}

 \item \textbf{Calcular a função de corrente pela equação:}
  \begin{equation}
  \Big[ K_{xx} + K_{yy} \Big] \psi = Mw \notag
  \end{equation}
  \textit{É necessário aplicar as condições de contorno de $\psi$ na matriz a esquerda da equação
  zerando suas linhas e colunas e a contribuição das colunas dos índices de contorno 
  no vetor a direita da equação,
  conforme explicado na seção \ref{tricond}.}


 \item \textbf{Calcular a velocidade pela equação:}
  \begin{align}
   & Mu = G_{y}\psi  \notag \\
   & Mv = -G_{x}\psi \notag
  \end{align}
  \textit{É necessário aplicar as condições de contorno de u e v em suas 
  respectivas matrizes a esquerda de cada equação
  zerando suas linhas e colunas e a contribuição das colunas dos índices de contorno 
  no vetor a direita de cada equação,
  conforme explicado na seção anterior.}


 \item \textbf{Calcular a concentração pela equação:}
  \begin{equation}
  \begin{aligned}
   \frac{M}{\Delta t} c^{n+1}
   =  \frac{M}{\Delta t} c^{n}
   - u \cdot G_x c^{n}
   & - v \cdot G_y c^{n} 
   - \frac{1}{\textit{ReSc}} \Big[ K_{xx} + K_{yy} \Big] c^{n}
   \\
   & - u
   \frac{\Delta t}{2}
   \big[
   u K_{xx}
   + v K_{yx}
   \big]
   c^{n} 
   - v
   \frac{\Delta t}{2}
   \Big[
   u K_{xy}
   + v K_{yy}
   \Big]
   c^{n} \notag
  \end{aligned}
  \end{equation}
  \textit{Onde $c^{n}$ é a concentração calculada no passo de tempo anterior 
  e $c^{n+1}$ é a concentração que será calculada no passo de tempo em análise. 
  É necessário aplicar as condições de contorno de c na matriz a esquerda da equação
  zerando suas linhas e colunas e a contribuição das colunas dos índices de contorno 
  no vetor a direita da equação,
  conforme explicado na seção anterior.}


 \item \textbf{Retornar ao passo 3 e repetir o procedimento para outro
 passo de tempo.}
\end{enumerate}

\newpage
\noindent
Os passos 1 e 2 ficam fora do \textit{loop} do tempo, enquanto
os passos de 3 a 7 encontram-se dentro do \textit{loop}.
O procedimento para zerar as linhas e as colunas das matrizes globais
pode ser feito antes do \textit{loop}, exceto para o caso
da vorticidade em que a cada passo de tempo devemos zerar as linhas
e as colunas das matrizes globais e atribuir a contribuição dessas colunas
no vetor a direita da equação.

\medskip
\noindent
A seguir será apresentado o \textit{script} usado na solução da equação da 
vorticidade. A mesma ideia foi realizada para cada equação de governo,
alterando apenas o vetor do lado direito e as contribuições das condições
de contorno. Foi utilizado um método iterativo de solução para sistemas
lineares conhecido como \textit{Gradientes Conjugados}.

\begin{verbatim}
_________________________________________________________________________
 RHS = sps.lil_matrix.dot(np.copy(M)/dt,w)\ 
     - np.multiply(vx,sps.lil_matrix.dot(Gx,w))\
     - np.multiply(vy,sps.lil_matrix.dot(Gy,w))\
     - (1.0/Re)*sps.lil_matrix((Kxx+Kyy),w)\
     - (dt/2.0)*np.multiply(u,(np.multiply(u,sps.lil_matrix.dot(Kxx,w))\ 
                             + np.multiply(v,sps.lil_matrix.dot(Kyx,w))))\
     - (dt/2.0)*np.multiply(v,(np.multiply(u,sps.lil_matrix.dot(Kxy,w))\ 
                             + np.multiply(v,sps.lil_matrix.dot(Kyy,w))))
 
 RHS = RHS + (1/Re)*bc_neumann
 RHS = np.multiply(RHS,bc_2)
 RHS = RHS + bc_dirichlet
 
 w = scipy.sparse.linalg.cg(LHS,RHS,w, maxiter=1.0e+05, tol=1.0e-05)
 w = w[0].reshape((len(w[0]),1))
__________________________________________________________________________
\end{verbatim}

\noindent
Onde \textit{RHS} é o vetor do lado direito da equação e significa
\textit{right hand side} e \textit{bc\_2} é um vetor auxiliar
no qual garante que somente os nós sem condição de Dirichlet sejam
resolvidos. Ele consiste em um vetor com dimensões \textit{np}
onde possui o valor de 1 nos índices cujos nós não possuem
condição de Dirichlet e o valor 0 nos índices restantes, isto é,
os nós que possuem condição de Dirichlet. Vale ressaltar que o
vetor \textit{bc\_2} é diferente para cada equação já que os
contornos cuja condição é de Dirichlet variam de equação para equação,
ou seja, o vetor \textit{bc\_2} da equação da vorticidade é
diferente da equação de concentração. O primeiro bloco do \textit{script}
acima consiste no lado direito da equação (\ref{final equation}).
O segundo bloco refere-se a contribuição das condições de Neumann
(para esta simulação é nula) e de Dirichlet, além da aplicação do vetor
auxiliar \textit{bc\_2}. O terceiro bloco consiste na solução da
equação pelo método iterativo dos \textit{Gradientes Conjugados}.

