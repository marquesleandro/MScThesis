Neste capítulo, descreveremos o Método dos
Elementos Finitos (MEF). Enquanto o Método
das Diferenças Finitas (MDF) representa
uma aproximação direta para as equações
diferenciais, a proposta do procedimento
dos elementos finitos é uma aproximação
aplicada para os termos da formulação
variacional como mencionado por Zienkiewicz e Cheung (1965) \cite{zienkiewicz1965}. 
Para mais detalhes sobre o Método dos Elementos
Finitos consultar os trabalhos de Zienkiewicz e Taylor (2000) \cite{zienkiewiczvol3},
Hughes (2000) \cite{hughes2000} e Fish e Belytschko (2007) \cite{fish2007}. \par 

Primeiramente, discretizaremos as equações
de governo no tempo utilizando
a aproximação da série de Taylor
mantendo os termos de segunda ordem da expansão
a fim de reduzir as oscilações espúrias
presentes nas equações do tipo convecção-difusão
como no caso das equações de vorticidade e de concentração.
Em seguida, apresentaremos a formulação
forte das mesmas. 
Feito isso, a formulação fraca das equações
de governo é apresentada e as mesmas são
discretizadas no espaço utilizando a formulação
de Galerkin com um elemento triangular linear.
Portanto, as equações de governo na forma matricial
segundo o esquema \textit{Taylor-Galerkin} é apresentado. Para mais detalhes
sobre o esquema, consultar o trabalho de Donea (1984) \cite{donea1984}.


