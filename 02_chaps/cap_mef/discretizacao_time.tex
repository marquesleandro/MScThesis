Nesta seção discretizaremos as equações de governo no tempo,
através da expansão em série de Taylor para a variável em questão 
a fim de aproximar a derivada temporal. Com o intuito de simplificação,
apresentaremos a discretização da equação da vorticidade.
Um procedimento semelhante poderá ser feito para a equação de transporte de espécie química (Eq. \ref{especie quimica}).
Sendo assim, expandindo os termos da equação da vorticidade (Eq. \ref{vorticidade}), temos:

\begin{equation}
 \frac{\partial w}{\partial t} 
 + u\frac{\partial w}{\partial x}
 + v\frac{\partial w}{\partial y} 
 = \frac{1}{Re} \frac{\partial^{2} w}{\partial x^{2}}
 + \frac{1}{Re} \frac{\partial^{2} w}{\partial y^{2}}
\end{equation}

\medskip
\noindent
isto é:

\begin{equation} \label{vorticity expanded}
 \frac{\partial w}{\partial t} 
 = - u\frac{\partial w}{\partial x}
 - v\frac{\partial w}{\partial y} 
 + \frac{1}{Re} \frac{\partial^{2} w}{\partial x^{2}}
 + \frac{1}{Re} \frac{\partial^{2} w}{\partial y^{2}}
\end{equation}

\noindent
Multiplicando ambos os lado por $\partial /\partial t$, temos:

\begin{equation} \label{vorticity expanded with dt}
 \frac{\partial }{\partial t}
 \Bigg[ 
 \frac{\partial w}{\partial t} 
 \Bigg]
 = 
 \frac{\partial }{\partial t}
 \Bigg[ 
 - u \frac{\partial w}{\partial x}
 - v \frac{\partial w}{\partial y}
 + \frac{1}{Re} \frac{\partial^2 w}{\partial x^2} 
 + \frac{1}{Re} \frac{\partial^2 w}{\partial y^2} 
 \Bigg]
\end{equation}


\noindent
Desta forma, considerando a expansão de Taylor:
\begin{equation}
 w^{n+1} = \sum\limits_{k=0}^{\infty} 
           \frac{\partial^k w^{n}}{\partial t^k} \frac{\Delta t^k}{k!}
\end{equation}

\medskip
\noindent
Desenvolvendo a série, temos:

\begin{equation}
 w^{n+1} = w^{n} 
         + \frac{\partial w^{n}}{\partial t} \frac{\Delta t}{1!} 
         + \frac{\partial^2 w^{n}}{\partial t^2} \frac{\Delta t^2}{2!}
         + \frac{\partial^3 w^{n}}{\partial t^3} \frac{\Delta t^3}{3!}
         + ...
\end{equation}

\medskip
\noindent
Caso os termos de ordem superior a dois forem omitidos, a equação fica 
da forma:

\begin{equation} \label{erro nao omitido}
 w^{n+1} = w^{n} 
         + \frac{\partial w^{n}}{\partial t}\Delta t 
         + \frac{\partial^2 w^{n}}{\partial t^2} \frac{\Delta t^2}{2}
         + O(\Delta t^3)
\end{equation}

\noindent
onde \textit{O($\Delta t^3$)} é o erro devido o truncamento da série.
Graficamente, esta aproximação pode ser representada como apresentado
na \ref{erro vorticidade}, a seguir:

\begin{figure}[H]
\begin{center}
\begin{tikzpicture}[scale=4]
 \draw [->,thick] (0,0)--(0,1.5) node[left] {$w$};
 \draw [->,thick] (0,0)--(2,0) node[below] {$t$};

 \draw (0.4,0.47) to [bend left=0] (0.5,0.5);
 \draw (0.5,0.5) to [bend left=10] (1.5,0.7);
 \draw (1.5,0.7) to [bend left=0] (1.6,0.71);
 
 \draw[dashed] (0.5,0.5) to [bend left=4] (1.5,0.7);
 
 \draw[dotted] (0.5,0.5) to [bend left=0] (0.5,0);
 \draw[dotted] (1.5,0.7) to [bend left=0] (1.5,0);
 
 \draw (0.5,-0.01) to (0.5,-0.2);
 \draw (1.5,-0.01) to (1.5,-0.2);
 \draw [<->,thick] (0.5,-0.1)--(1.5,-0.1);
 \node [draw=none] at (1,-0.2) {$\Delta t$};
 
 \draw[dotted] (0.5,0.5) to [bend left=0] (0,0.5) node[left]{$w^{n}$};
 \draw[dotted] (1.5,0.7) to [bend left=0] (0,0.7) node[left]{$w^{n+1}$};

 \draw [->,thick] (0.8,0.9)--(0.9,0.65);
 \node [draw=none] at (0.8,1) {$exata$};
 \draw [->,thick] (1,0.35)--(0.95,0.57);
 \node [draw=none] at (1.2,0.3) {$aproximada$};
\end{tikzpicture}
\end{center}
\caption{Variação da vorticidade em um passo de tempo}
\label{erro vorticidade}
\end{figure}

\noindent
Omitindo o erro do truncamento, a derivada temporal (Eq. \ref{erro nao omitido}) pode ser aproximada a:

\begin{equation}
 w^{n+1} = w^{n} 
         + \frac{\partial w^{n}}{\partial t}\Delta t 
         + \frac{\partial^2 w^{n}}{\partial t^2} \frac{\Delta t^2}{2}
\end{equation}


\noindent
onde $w^{n+1}$ é a vorticidade que será calculada
e $w^{n}$ é a vorticidade que foi calculada no passo de tempo anterior.
Substituindo a equação da vorticidade (Eq. \ref{vorticity expanded})
e a equação da vorticidade multiplicada pela derivada temporal
(Eq. \ref{vorticity expanded with dt}) na aproximação feita acima, temos:

\begin{equation}
\begin{aligned}
 w^{n+1} = w^{n} 
         + \Delta t 
         \Bigg[
         - & u \frac{\partial w^{n}}{\partial x}
         - v \frac{\partial w^{n}}{\partial y}
         + \frac{1}{Re} \frac{\partial^2 w^{n}}{\partial x^2} 
         + \frac{1}{Re} \frac{\partial^2 w^{n}}{\partial y^2} 
         \Bigg]
         \\[5pt]
         & + \frac{\Delta t^2}{2}
         \Bigg[
         \frac{\partial }{\partial t}
         \Bigg[ 
         - u \frac{\partial w^{n}}{\partial x}
         - v \frac{\partial w^{n}}{\partial y}
         + \frac{1}{Re} \frac{\partial^2 w^{n}}{\partial x^2} 
         + \frac{1}{Re} \frac{\partial^2 w^{n}}{\partial y^2} 
         \Bigg]
         \Bigg]
\end{aligned}
\end{equation}

\noindent
Assumindo que \textit{u} e \textit{v} são constantes, temos:

\begin{equation}
\begin{aligned}
 w^{n+1} = w^{n} 
         + \Delta t &
           \Bigg[
         - u \frac{\partial w^{n}}{\partial x}
         - v \frac{\partial w^{n}}{\partial y}
         + \frac{1}{Re} \frac{\partial^2 w^{n}}{\partial x^2} 
         + \frac{1}{Re} \frac{\partial^2 w^{n}}{\partial y^2} 
         \Bigg]
         \\[5pt]
         & + \frac{\Delta t^2}{2}
         \Bigg[
         - u
         \frac{\partial }{\partial t}
         \frac{\partial w^{n}}{\partial x}
         - v
         \frac{\partial }{\partial t}
         \frac{\partial w^{n}}{\partial y}
         + \frac{1}{Re} \frac{\partial }{\partial t}
         \frac{\partial^2 w^{n}}{\partial x^2} 
         + \frac{1}{Re} \frac{\partial }{\partial t}
         \frac{\partial^2 w^{n}}{\partial y^2} 
         \Bigg]
\end{aligned}
\end{equation}

\noindent
Invertendo as ordens de derivação do último termo, encontramos:

\begin{equation}
\begin{aligned}
 w^{n+1} = w^{n} 
         & + \Delta t 
         \Bigg[
         - u \frac{\partial w^{n}}{\partial x}
         - v \frac{\partial w^{n}}{\partial x}
         + \frac{1}{Re} \frac{\partial^2 w^{n}}{\partial x^2} 
         + \frac{1}{Re} \frac{\partial^2 w^{n}}{\partial y^2} 
         \Bigg]
         \\[5pt]
         & + \frac{\Delta t^2}{2}
         \Bigg[
         - u
         \frac{\partial }{\partial x}
         \frac{\partial w^{n}}{\partial t}
         - v
         \frac{\partial }{\partial y}
         \frac{\partial w^{n}}{\partial t}
         + \frac{1}{Re} \frac{\partial^2 }{\partial x^2}
         \frac{\partial w^{n}}{\partial t} 
         + \frac{1}{Re} \frac{\partial^2 }{\partial y^2}
         \frac{\partial w^{n}}{\partial t} 
         \Bigg]
\end{aligned}
\end{equation}

\noindent
Substituindo os termo $\partial w/ \partial t$ pela equação
de vorticidade (Eq. \ref{vorticity expanded}), temos:

\begin{equation}
\begin{aligned}
 w^{n+1} = w^{n} 
         + \Delta t 
         \Bigg[
         & - u \frac{\partial w^{n}}{\partial x}
         - v \frac{\partial w^{n}}{\partial x}
         + \frac{1}{Re} \frac{\partial^2 w^{n}}{\partial x^2} 
         + \frac{1}{Re} \frac{\partial^2 w^{n}}{\partial y^2} 
         \Bigg]
         \\[5pt]
         + \frac{\Delta t^2}{2}
         \Bigg[
         & - u
         \frac{\partial }{\partial x}
         \Bigg[
          - u \frac{\partial w^{n}}{\partial x}
          - v \frac{\partial w^{n}}{\partial y}
          + \frac{1}{Re} \frac{\partial^2 w^{n}}{\partial x^2} 
          + \frac{1}{Re} \frac{\partial^2 w^{n}}{\partial y^2} 
         \Bigg]
         \\[5pt]
         & - v
         \frac{\partial }{\partial y}
         \Bigg[
          - u \frac{\partial w^{n}}{\partial x}
          - v \frac{\partial w^{n}}{\partial y}
          + \frac{1}{Re} \frac{\partial^2 w^{n}}{\partial x^2} 
          + \frac{1}{Re} \frac{\partial^2 w^{n}}{\partial y^2} 
         \Bigg]
         \\[5pt]
         & + \frac{\partial^2 }{\partial x^2}
         \Bigg[
          - u \frac{\partial w^{n}}{\partial x}
          - v \frac{\partial w^{n}}{\partial y}
          + \frac{1}{Re} \frac{\partial^2 w^{n}}{\partial x^2} 
          + \frac{1}{Re} \frac{\partial^2 w^{n}}{\partial y^2} 
         \Bigg]
         \\[5pt]
         & + \frac{\partial^2 }{\partial y^2}
         \Bigg[
          - u \frac{\partial w^{n}}{\partial x}
          - v \frac{\partial w^{n}}{\partial y}
          + \frac{1}{Re} \frac{\partial^2 w^{n}}{\partial x^2} 
          + \frac{1}{Re} \frac{\partial^2 w^{n}}{\partial y^2} 
         \Bigg]
         \Bigg]
\end{aligned}
\end{equation}

\medskip
\noindent
Truncando os termos de ordem superior a dois, obtemos:

\begin{equation}
\begin{aligned}
 w^{n+1} & = w^{n} 
          + \Delta t 
         \Bigg[
         - u \frac{\partial w^{n}}{\partial x}
         - v \frac{\partial w^{n}}{\partial x}
         + \frac{1}{Re} \frac{\partial^2 w^{n}}{\partial x^2} 
         + \frac{1}{Re} \frac{\partial^2 w^{n}}{\partial y^2} 
         \Bigg]
         \\[5pt]
         & + \frac{\Delta t^2}{2}
         \Bigg[
         - u
         \frac{\partial }{\partial x}
         \Bigg[
          - u \frac{\partial w^{n}}{\partial x}
          - v \frac{\partial w^{n}}{\partial y}
         \Bigg]
          - v
         \frac{\partial }{\partial y}
         \Bigg[
          - u \frac{\partial w^{n}}{\partial x}
          - v \frac{\partial w^{n}}{\partial y}
         \Bigg]
         \Bigg]
         + O(\Delta t^3)
\end{aligned}
\end{equation}

\medskip
\noindent
Omitindo novamente o erro de truncamento, possuímos
a seguinte equação:

\begin{equation}
\begin{aligned}
 w^{n+1} = w^{n} 
         & + \Delta t 
         \Bigg[
         - u \frac{\partial w^{n}}{\partial x}
         - v \frac{\partial w^{n}}{\partial x}
         + \frac{1}{Re} \frac{\partial^2 w^{n}}{\partial x^2} 
         + \frac{1}{Re} \frac{\partial^2 w^{n}}{\partial y^2} 
         \Bigg]
         \\[5pt]
         & + \frac{\Delta t^2}{2}
         \Bigg[
         - u
         \frac{\partial }{\partial x}
         \Bigg[
          - u \frac{\partial w^{n}}{\partial x}
          - v \frac{\partial w^{n}}{\partial y}
         \Bigg]
          - v
         \frac{\partial }{\partial y}
         \Bigg[
          - u \frac{\partial w^{n}}{\partial x}
          - v \frac{\partial w^{n}}{\partial y}
         \Bigg]
         \Bigg]
\end{aligned}
\end{equation}

\noindent
isto é:

\begin{equation}
\begin{aligned}
 \Bigg[
 \frac{w^{n+1} - w^{n}}{\Delta t}
 \Bigg]
 + u \frac{\partial w^{n}}{\partial x}
 & + v \frac{\partial w^{n}}{\partial y}
 = \frac{1}{Re} \frac{\partial^2 w^{n}}{\partial x^2} 
 + \frac{1}{Re} \frac{\partial^2 w^{n}}{\partial y^2} 
 \\[5pt]
 & + \frac{\Delta t}{2} u \frac{\partial }{\partial x}
 \Bigg[
   u \frac{\partial w^{n}}{\partial x}
 + v \frac{\partial w^{n}}{\partial y}
 \Bigg]
 + \frac{\Delta t}{2} v \frac{\partial }{\partial y}
 \Bigg[
   u \frac{\partial w^{n}}{\partial x}
 + v \frac{\partial w^{n}}{\partial y}
 \Bigg]
\end{aligned}
\end{equation}

\medskip
\noindent
Os dois últimos termos da equação acima são conhecidos como
difusão artificial ou difusão numérica e são eles que atuam
para a redução das oscilações espúrias que aparecem para 
Reynolds moderados ou elevados.
Outros esquemas são conhecidos na literatura para eliminar essas
oscilações espúrias tais como \textit{Petrov-Galerkin} para
equações em 1D e \textit{Streamline Upwind Petrov-Galerkin} (SUPG)
para equações em 2D ambos em problemas permanentes. Nesses
esquemas, as funções bases são modificadas para obter um efeito
\textit{upwind}. Para problemas transientes, além do 
\textit{Taylor-Galerkin} temos: \textit{Semi-Lagrangiano} e
\textit{Galerkin Característico}. 
Os esquemas \textit{Taylor-Galerkin} e 
\textit{Galerkin Característico} possuem o mesmo resultado 
quando a variável é escalar como apresentado por Lohner, Morgan e Zienkiewicz (1984) \cite{lohner1984}. 

\medskip
\noindent
Na forma vetorial, as equações de governo discretizadas no tempo
possuem a forma:

\begin{align}
& \overset{.}{w} + \textbf{v}.\nabla w = \frac{1}{Re} \nabla^2 w 
 + \frac{\Delta t}{2} \textbf{v} \cdot \nabla \big[ \textbf{v} \cdot \nabla w \big] \label{vorticity taylor-galerkin} \\[10pt]
& \nabla^2 \psi = - w \\[10pt]
& \textbf{v} = \textbf{D}\psi \\[10pt]
& \overset{.}{c} + \textbf{v}.\nabla c = \frac{1}{ReSc} \nabla^2 c
 + \frac{\Delta t}{2} \textbf{v} \cdot \nabla \big[ \textbf{v} \cdot \nabla c \big] \label{concentration taylor-galerkin}
\end{align}

\noindent
onde $\overset{.}{w}$ e
$\overset{.}{c}$ são 
$\big[ w^{n+1}-w^{n} \big] /\Delta t$ e
$\big[ c^{n+1}-c^{n} \big] /\Delta t$ respectivamente,
\textbf{v} é o vetor velocidade cujas componentes são
$ \textbf{v} = [u,v]$
e \textbf{D} é um operador diferencial com componentes
$ \textbf{D} = [\partial/\partial y, -\partial/\partial x]$.

\newpage
