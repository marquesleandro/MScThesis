\subsection{Método Taylor-Galerkin}

Quando possuímos um número de Reynolds acima de um determinado
valor crítico, é possível observar algumas oscilações espúrias no resultado
da equação convecção-difusão. Estas oscilações não são características
apenas do MEF mas podem ser vistas no MDF também. Para problemas permanentes,
essas oscilações são resolvidas utilizando as formulações 
\textit{Petrov-Galerkin} para equações em 1D 
ou \textit{Streamline Upwind Petrov-Galerkin} (SUPG) para
equações em 2D. Nesses métodos, as funções bases são modificadas
para obter um efeito \textit{upwind}. Para problemas transientes,
os métodos mais utilizados para eliminar essas oscilações são:
\textit{Semi-Lagrangiano}, \textit{Galerkin Característico} e
\textit{Taylor-Galerkin}. 
Nesta seção apresentaremos a formulação do \textbf{Método Taylor-Galerkin}.


\medskip
O método Taylor-Galerkin produzirá um termo difusivo na equação em que
reduzirá as oscilações espúrias. Com o intuito didático, apresentaremos
a formulação para uma equação convecção-difusão em 1D. Para mais informação
consultar \cite{donea1984} e \cite{zienkiewiczvol3}.

\medskip
Dado a equação convecção-difusão em 1D:

\begin{equation} \label{convection diffusion 1D}
 \frac{\partial s}{\partial t} 
 = 
 - u \frac{\partial s}{\partial x}
 + \frac{\partial^2 s}{\partial x^2} 
\end{equation}

\medskip
Onde \textit{s} é uma variável escalar qualquer e \textit{u}
é a velocidade.
Multiplicando ambos os lados por $\partial / \partial t$, temos:

\begin{equation} \label{convection diffusion 1D mod}
 \frac{\partial }{\partial t}
 \bigg[ 
 \frac{\partial s}{\partial t} 
 \bigg]
 = 
 \frac{\partial }{\partial t}
 \bigg[ 
 - u \frac{\partial s}{\partial x}
 + \frac{\partial^2 s}{\partial x^2} 
 \bigg]
\end{equation}

A série de Taylor para o escalar \textit{s} será:
\begin{equation}
 s^{n+1} = s^{n} 
         + \frac{\partial s^{n}}{\partial t} \frac{\Delta t}{1!} 
         + \frac{\partial^2 s^{n}}{\partial t^2} \frac{\Delta t^2}{2!}
         + ...
\end{equation}

\medskip
Caso os termos de ordem superior a dois forem omitidos, a equação fica 
da forma:

\begin{equation}
 s^{n+1} = s^{n} 
         + \frac{\partial s^{n}}{\partial t}\Delta t 
         + \frac{\partial^2 s^{n}}{\partial t^2} \frac{\Delta t^2}{2!}
         + O(\Delta t^3)
\end{equation}

onde \textit{O($\Delta t^3$)} é o erro devido o truncamento da série e
será atribuído ao erro do MEF, assim a aproximação se torna:

\begin{equation} \label{taylor serie 1D}
 s^{n+1} = s^{n} 
         + \frac{\partial s^{n}}{\partial t}\Delta t 
         + \frac{\partial^2 s^{n}}{\partial t^2} \frac{\Delta t^2}{2}
\end{equation}

Substituindo a equação convecção-difusão (\ref{convection diffusion 1D})
e a equação convecção-difusão multiplicada pela derivada temporal
(\ref{convection diffusion 1D mod}) na aproximação feita acima, temos:

\begin{equation}
 s^{n+1} = s^{n} 
         + \Delta t 
         \bigg[
         - u \frac{\partial s^{n}}{\partial x}
         + \frac{\partial^2 s^{n}}{\partial x^2} 
         \bigg]
         + \frac{\Delta t^2}{2}
         \bigg[
         \frac{\partial }{\partial t}
         \bigg[ 
         - u \frac{\partial s^{n}}{\partial x}
         + \frac{\partial^2 s^{n}}{\partial x^2} 
         \bigg]
         \bigg]
\end{equation}

Assumindo que \textit{u} é constante, temos:

\begin{equation}
 s^{n+1} = s^{n} 
         + \Delta t 
         \bigg[
         - u \frac{\partial s^{n}}{\partial x}
         + \frac{\partial^2 s^{n}}{\partial x^2} 
         \bigg]
         + \frac{\Delta t^2}{2}
         \bigg[
         - u
         \frac{\partial }{\partial t}
         \frac{\partial s^{n}}{\partial x}
         + \frac{\partial }{\partial t}
         \frac{\partial^2 s^{n}}{\partial x^2} 
         \bigg]
\end{equation}

Invertendo as ordens de derivação do último termo, ficamos:

\begin{equation}
 s^{n+1} = s^{n} 
         + \Delta t 
         \bigg[
         - u \frac{\partial s^{n}}{\partial x}
         + \frac{\partial^2 s^{n}}{\partial x^2} 
         \bigg]
         + \frac{\Delta t^2}{2}
         \bigg[
         - u
         \frac{\partial }{\partial x}
         \frac{\partial s^{n}}{\partial t}
         + \frac{\partial^2 }{\partial x^2}
         \frac{\partial s^{n}}{\partial t} 
         \bigg]
\end{equation}


Substituindo os termo $\partial s/ \partial t$ pela equação
convecção-difusão (\ref{convection diffusion 1D}), temos:

\begin{equation}
 s^{n+1} = s^{n} 
         + \Delta t 
         \bigg[
         - u \frac{\partial s^{n}}{\partial x}
         + \frac{\partial^2 s^{n}}{\partial x^2} 
         \bigg]
         + \frac{\Delta t^2}{2}
         \bigg[
         - u
         \frac{\partial }{\partial x}
         \bigg[
          - u \frac{\partial s^{n}}{\partial x}
          + \frac{\partial^2 s^{n}}{\partial x^2} 
         \bigg]
         + \frac{\partial^2 }{\partial x^2}
         \bigg[
          - u \frac{\partial s^{n}}{\partial x}
          + \frac{\partial^2 s^{n}}{\partial x^2} 
         \bigg]
         \bigg]
\end{equation}

\medskip
Desenvolvendo a equação, temos:

\begin{equation}
 s^{n+1} = s^{n} 
         + \Delta t 
         \bigg[
         - u \frac{\partial s^{n}}{\partial x}
         + \frac{\partial^2 s^{n}}{\partial x^2} 
         \bigg]
         + \frac{\Delta t^2}{2}
         \bigg[
         - u^2 \frac{\partial^2 s^{n}}{\partial x^2}
         - u \frac{\partial^3 s^{n}}{\partial x^3} 
         - u \frac{\partial^3 s^{n}}{\partial x^3}
         + \frac{\partial^4 s^{n}}{\partial x^4} 
         \bigg]
\end{equation}

\medskip
Truncando os termos de ordem superior a dois, ficamos:

\begin{equation}
 s^{n+1} = s^{n} 
         + \Delta t 
         \bigg[
         - u \frac{\partial s^{n}}{\partial x}
         + \frac{\partial^2 s^{n}}{\partial x^2} 
         \bigg]
         + \frac{\Delta t^2}{2}
         \bigg[
         + u^2 \frac{\partial^2 s^{n}}{\partial x^2}
         + O(\Delta t^3) 
         \bigg]
\end{equation}

\medskip
O erro novamente será atribuído ao MEF, dessa forma, possuímos
a seguinte equação:

\begin{equation}
 s^{n+1} = s^{n} 
         + \Delta t 
         \bigg[
         - u \frac{\partial s^{n}}{\partial x}
         + \frac{\partial^2 s^{n}}{\partial x^2} 
         \bigg]
         + u^2 
         \frac{\Delta t^2}{2}
         \frac{\partial^2 s^{n}}{\partial x^2}
\end{equation}

Isto é:
\begin{equation}
 \bigg[
 \frac{s^{n+1} - s^{n}}{\Delta t}
 \bigg]
 =  
 - u \frac{\partial s^{n}}{\partial x}
 + \frac{\partial^2 s^{n}}{\partial x^2} 
 + u^2 
 \frac{\Delta t}{2}
 \frac{\partial^2 s^{n}}{\partial x^2}
\end{equation}

\medskip
Para uma equação convecção-difusão em 2D, a equação se torna:

\begin{equation}
\begin{aligned}
 \bigg[
 \frac{s^{n+1} - s^{n}}{\Delta t}
 \bigg]
 =  
 - u \frac{\partial s^{n}}{\partial x}
 - v \frac{\partial s^{n}}{\partial y}
 + \frac{\partial^2 s^{n}}{\partial x^2} 
 + \frac{\partial^2 s^{n}}{\partial y^2} 
 + u 
 \frac{\Delta t}{2}
 \frac{\partial}{\partial x}
 \bigg[
 u \frac{\partial s^{n}}{\partial x}
 + v \frac{\partial s^{n}}{\partial y}
 \bigg] \\
 + v 
 \frac{\Delta t}{2}
 \frac{\partial}{\partial y}
 \bigg[
 u \frac{\partial s^{n}}{\partial x}
 + v \frac{\partial s^{n}}{\partial y}
 \bigg]
\end{aligned}
\end{equation}

\medskip
Os métodos \textit{Taylor-Galerkin} e \textit{Galerkin Característico}
possuem o mesmo resultado quando a variável é escalar \cite{lohner1984}.
Os dois últimos termos da equação acima são conhecidos como
difusão artificial ou difusão numérica e são eles que atuam
para a redução das oscilações espúrias que ocorrem quando
o número de Reynolds é acima de um valor crítico. A representação matricial
destes termos segundo o Método dos Elementos Finitos será
apresentado a seguir. Os termos restantes da equação acima já
foram apresentados matricialmente nas seções anteriores quando
a variável escalar era a vorticidade ($\omega$) ou a concentração (\textit{c}).


\begin{align}
 u 
 \frac{\Delta t}{2}
 \frac{\partial}{\partial x}
 \bigg[
 u \frac{\partial s^{n}}{\partial x}
 + v \frac{\partial s^{n}}{\partial y}
 \bigg]
 = 
 u
 \frac{\Delta t}{2}
 \Big[
 u K_{xx}
 + v K_{xy}
 \Big]
 s^{n}\\
 v 
 \frac{\Delta t}{2}
 \frac{\partial}{\partial y}
 \bigg[
 u \frac{\partial s^{n}}{\partial x}
 + v \frac{\partial s^{n}}{\partial y}
 \bigg]
 = 
 v
 \frac{\Delta t}{2}
 \Big[
 u K_{yx}
 + v K_{yy}
 \Big]
 s^{n}
\end{align}

\medskip
Onde \textbf{K\textsubscript{xy}} e \textbf{K\textsubscript{yx}} 
possuem dimensões \textbf{n}x\textbf{n}
e são definidas como:

\begin{align}
  \textbf{K\textsubscript{xy}} & = \textbf{A} k_{xy}^{e} \\
  \textbf{K\textsubscript{yx}} & = \textbf{A} k_{yx}^{e}
\end{align}

\medskip
Onde 
$k^{e}_{xy}$ e
$k^{e}_{yx}$
são as matrizes elementares cuja dimensão para o 
\textit{elemento triangular linear} é \textit{3}x\textit{3} e 
são definidas por:


\begin{equation}
 \begin{aligned}
  k_{xy}^{e} & = \int_{\Omega^{e}} \frac{\partial N_{i}^{e}}{\partial x} \frac{\partial N_{j}^{e}}{\partial y} \\
  k_{yx}^{e} & = \int_{\Omega^{e}} \frac{\partial N_{i}^{e}}{\partial y} \frac{\partial N_{j}^{e}}{\partial x}
 \end{aligned}
\end{equation}


\medskip
Com isso, possuímos:

\begin{equation}
\begin{aligned}
 M \bigg[ \frac{w^{n+1} - w^{n}}{\Delta t} \bigg]
 + u \cdot G_x & w^{n}
 + v \cdot G_y w^{n} 
 + \frac{1}{\textit{Re}} \Big[ K_{xx} + K_{yy} \Big] w^{n+1} \\
 & + u
 \frac{\Delta t}{2}
 \big[
 u K_{xx}
 + v K_{xy}
 \big]
 w^{n} 
 + v
 \frac{\Delta t}{2}
 \Big[
 u K_{yx}
 + v K_{yy}
 \Big]
 w^{n} 
 = 0 \\
\end{aligned}
\end{equation}

\begin{equation}
 - \Big[ K_{xx} + K_{yy} \Big] \psi + Mw  = 0
\end{equation}

\begin{equation}
 Mu - G_y \psi = 0
\end{equation}

\begin{equation}
 Mv + G_x \psi = 0
\end{equation}

\begin{equation}
\begin{aligned}
 M \bigg[ \frac{c^{n+1} - c^{n}}{\Delta t} \bigg]
 + u \cdot G_x c^{n}
 & + v \cdot G_y c^{n} 
 + \frac{1}{\textit{ReSc}} \Big[ K_{xx} + K_{yy} \Big] c^{n+1} \\
 & + u
 \frac{\Delta t}{2}
 \big[
 u K_{xx}
 + v K_{xy}
 \big]
 c^{n}
 + v
 \frac{\Delta t}{2}
 \Big[
 u K_{yx}
 + v K_{yy}
 \Big]
 c^{n}
 = 0  
\end{aligned} 
\end{equation}

\medskip
Apresentaremos a seguir as equações de governo em sua 
forma adimensional discretizadas segundo o Método dos Elementos 
Finitos que foram usadas neste trabalho:

\begin{equation} \label{vorticity taylor}
\begin{aligned}
 \bigg[ \frac{M}{\Delta t} + \frac{1}{\textit{Re}} \Big[ K_{xx} + K_{yy} \Big] \bigg] w^{n+1}
 & = \frac{M}{\Delta t} w^{n}
 - u \cdot G_x  w^{n}
 - v \cdot G_y w^{n} \\
 & - u
 \frac{\Delta t}{2}
 \big[
 u K_{xx}
 + v K_{xy}
 \big]
 w^{n} 
 - v
 \frac{\Delta t}{2}
 \Big[
 u K_{yx}
 + v K_{yy}
 \Big]
 w^{n}\\
\end{aligned}
\end{equation}

\begin{equation}
 \Big[ K_{xx} + K_{yy} \Big] \psi = Mw
\end{equation}

\begin{equation}
 Mu = G_y \psi 
\end{equation}

\begin{equation}
 Mv = - G_x \psi 
\end{equation}

\begin{equation}
\begin{aligned}
 \bigg[ \frac{M}{\Delta t} + \frac{1}{\textit{ReSc}} \Big[ K_{xx} + K_{yy} \Big] \bigg] c^{n+1}
 & = \frac{M}{\Delta t} c^{n}
 - u \cdot G_x  c^{n}
 - v \cdot G_y c^{n} \\
 & - u
 \frac{\Delta t}{2}
 \big[
 u K_{xx}
 + v K_{xy}
 \big]
 c^{n} 
 - v
 \frac{\Delta t}{2}
 \Big[
 u K_{yx}
 + v K_{yy}
 \Big]
 c^{n} 
\end{aligned}
\end{equation}


\newpage
