The Semi-Lagrangian Scheme in centred difference was proposed by \cite{sawyer1963} for atmospheric flow numerical simulation using vorticity-advection equation, allowing to use large time steps without numerical instability. 
However, because of a limited computer capability, the use of such methodology to model several fluid flow problems, with high order differences and fine mesh, came latter in the 1980's throught the work of \cite{robert1981} and \cite{pironneau1982}, where the semi-lagrangian scheme would be able to run models faster than the Eulerian scheme, besides be unconditionally stable and only symmetric linear systems to solve. 
Basically, the semi-lagrangian scheme takes into account the fact that the Eulerian derivative is replaced by the material derivative, then it is discretized and computed along the trajectory characteristic:

The material derivate of the Eqs. \ref{vorticity matrix} and \ref{concentration matrix} may be discretized in the time
domain at the $x_{i}$ node by an explicit first order scheme:



\vspace{-0.4cm}
\begin{equation}
 \frac{D \omega}{D t} = 
 \frac{\omega_{i}^{n+1} - \omega_{d}^{n}}{\Delta t}
\end{equation}

\vspace{-0.8cm}
\begin{equation}
 \frac{D e}{D t} = 
 \frac{e_{i}^{n+1} - e_{d}^{n}}{\Delta t}
\end{equation}


\medskip
\noindent
where, 
the variable $\Delta t$ is time step, 
$\omega_{i}^{n+1}$ and $e_{i}^{n+1}$ are the vorticity and 
concentration fields calculated in current time step at the current 
node position and
$\omega_{d}^{n}$ and $e_{d}^{n}$ are the vorticity and 
concentration fields calculated in previous time step 
at the departure node position.
The departure node is calculated by solving characteristic equation:

\begin{equation} \label{characteristic}
\frac{d\textbf{x}_{d}}{dt} = \textbf{c}
\end{equation}

\medskip
\noindent
where,
$\textbf{c}$ is the relative velocity and
$t$ is time variable and 
it is varies between $t \in \left[t^{n},t^{n+1}\right]$.
Assuming that in the current time the mesh computational nodes
are known, so the initial condition for the Eq. \ref{characteristic}
can be represented by $\textbf{x}_{d}(t^{n+1}) = \textbf{x}_{i}^{n+1}$,
thus solving the equation:

\begin{equation} 
\textbf{x}_{d}^{n+1} - \textbf{x}_{d}^{n} 
= \int_{t^{n}}^{t^{n+1}} \textbf{c} dt, 
\end{equation}

\medskip
\noindent
that is:

\begin{equation} 
\textbf{x}_{d}^{n} = \textbf{x}_{i}^{n+1} 
- \int_{t^{n}}^{t^{n+1}} \textbf{c}  dt, 
\end{equation}


\medskip
\noindent
Considering the relative velocity picewise constant,
the departure point can be calculated by:

\begin{equation} \label{final characteristic}
\textbf{x}_{d}^{n} = \textbf{x}_{i}^{n+1} 
- \textbf{c} \Delta t, 
\end{equation}

\medskip
The \ref{semi-lagrangian figure} shown 
the 1-dimensional characteristic trajectory of
the material point in a moving computational mesh,
where the white square is the material point and
the black point is computational mesh.
The dashed line is the characteristic trajectory
that is represented by the Eq. \ref{final characteristic}.
According to this scheme, the initial point $x_i$ in $t^{n+1}$ is known
and therefore it is used to find 
the departure point $x_d$ position in $t^{n}$.


\vspace{0.5cm}
\begin{figure}[H]
\begin{center}
\begin{tikzpicture}[scale=6]

 % FIGURE A
 % grid
 \draw (0.5,0.7) -- (2.0,0.7);
 \draw (0.5,1.0) -- (2.0,1.0);
 \draw (0.5,1.2) -- (2.0,1.2);

 \draw (0.60,0.7) -- (0.64,1.0) -- (0.60,1.2);
 \draw (1.20,0.7) -- (1.12,1.0) -- (1.10,1.2);
 \draw (1.54,0.7) -- (1.64,1.0) -- (1.50,1.2);
 \draw (1.85,0.7) -- (1.90,1.0) -- (1.90,1.2);



 % material points
 \draw[dashed,-stealth] (1.12,1.0) -- (0.85,0.75);
 \node[square, fill=white, draw, inner sep=0pt, minimum size=9.4pt] at (0.8,0.7) {};
 \node[square, fill=white, draw, inner sep=0pt, minimum size=9.4pt] at (1.12,1.0) {};
 

 % nodes
 \node[circle, fill=black, inner sep=0pt, minimum size=5.2pt] at (0.60,0.7) {};
 \node[circle, fill=black, inner sep=0pt, minimum size=5.2pt] at (1.20,0.7) {};
 \node[circle, fill=black, inner sep=0pt, minimum size=5.2pt] at (1.54,0.7) {};
 \node[circle, fill=black, inner sep=0pt, minimum size=5.2pt] at (1.85,0.7) {};

 \node[circle, fill=black, inner sep=0pt, minimum size=5.2pt] at (0.64,1.0) {};
 \node[circle, fill=black, inner sep=0pt, minimum size=5.2pt] at (1.12,1.0) {};
 \node[circle, fill=black, inner sep=0pt, minimum size=5.2pt] at (1.64,1.0) {};
 \node[circle, fill=black, inner sep=0pt, minimum size=5.2pt] at (1.90,1.0) {};

 \node[circle, fill=black, inner sep=0pt, minimum size=5.2pt] at (0.60,1.2) {};
 \node[circle, fill=black, inner sep=0pt, minimum size=5.2pt] at (1.10,1.2) {};
 \node[circle, fill=black, inner sep=0pt, minimum size=5.2pt] at (1.50,1.2) {};
 \node[circle, fill=black, inner sep=0pt, minimum size=5.2pt] at (1.90,1.2) {};


 % legend
 \node[draw=none, scale=1.0] at (0.80,0.82) {\small $\mathbf{x}_{d}$};
 \node[draw=none, scale=1.0] at (1.22,0.88) {\small $\mathbf{x}_{i}$};

 \node[draw=none, scale=1.0] at (2.10,0.70) {\small $t^{n}$};
 \node[draw=none, scale=1.0] at (2.15,1.00) {\small $t^{n+1}$};
 \node[draw=none, scale=1.0] at (2.15,1.20) {\small $t^{n+2}$};

 \node[draw=none, scale=1.0] at (0.60,0.55) {\small $\mathbf{x}_{i-1}$};
 \node[draw=none, scale=1.0] at (1.22,0.55) {\small $\mathbf{x}_{i}$};
 \node[draw=none, scale=1.0] at (1.64,0.55) {\small $\mathbf{x}_{i+1}$};

\end{tikzpicture}
\end{center}
\caption{An one-dimensional space scheme
 where the departure node $x_{d}$ is found by 
 the characteristic trajectory in a moving mesh.}
\label{semi-lagrangian figure}
\end{figure}





using the initial condition $\mathbf{x}_{i}^{n+1} = \mathbf{x}(t^{n+1})$
 as shown in \ref{semi-lagrangian figure}a. 
A algorithm must be used to find the element that the departure node be, 
then the vorticity field in departure node ($\omega_{d}^{n}$) is 
calculed by barycenter coordinates interpolation between nodes of 
element found.
As shown in \ref{semi-lagrangian figure}b, three situations may 
occur depending on the trajectory: the first and the second 
situations are similar, differentiating only the trajectory length. 
In the first situation, the departure node is inside near 
element from current node, while the second situation 
the departure node is inside far element from current node. 
The third situation, the departure node is outside domain 
then the vorticity field in departure node receives the 
boundary condition value of nearest node to departure node.

\medskip
Therefore, the Eqs. 
\ref{vorticity matrix},
\ref{stream matrix} and
\ref{concentrarion matrix} 
can be shown 
in an
implicit semi-Lagragian 
discretization as:

\begin{equation}
 M \left[ \frac{\omega^{n+1}_{i} - \omega^{n}_{d}}{\Delta t} \right] 
 + \frac{1}{\textit{Re}} \left[ K_{xx} + K_{yy} \right] \omega^{n+1}_{i}
 = 0
\end{equation}

\begin{equation}
 - \left[ K_{xx} + K_{yy} \right] \psi + M \omega = 0
\end{equation}

\begin{equation}
 M \left[ \frac{e^{n+1}_{i} - e^{n}_{d}}{\Delta t} \right]
 + \frac{1}{\textit{ReSc}} \left[ K_{xx} + K_{yy} \right] e^{n+1}_{i}
 = 0
\end{equation}


\medskip
\noindent
that is,

\begin{equation}
 \frac{M}{\Delta t} \omega^{n+1}_{i}
 + \frac{1}{\textit{Re}} \left[ K_{xx} + K_{yy} \right] \omega^{n+1}_{i}
 = \frac{M}{\Delta t} \omega^{n}_{d}
\end{equation}

\begin{equation}
 \left[ K_{xx} + K_{yy} \right] \psi 
 =  M \omega
\end{equation}

\begin{equation}
 \frac{M}{\Delta t} e^{n+1}_{i}
 + \frac{1}{\textit{ReSc}} \left[ K_{xx} + K_{yy} \right] e^{n+1}_{i}
 = \frac{M}{\Delta t} e^{n}_{d}
\end{equation}


\medskip
Therefore, the Vorticity-Streamfunction Formulation with
Species Transport Equation discretized by \textit{Galerkin}
and \textit{semi-Lagrangian Methods} in an ALE-FE context
can be presented in matrix form as:

\begin{equation}
 \left[
 \frac{M}{\Delta t} 
 + \frac{1}{\textit{Re}} \left[ K_{xx} + K_{yy} \right]
 \right] 
 \omega^{n+1}_{i}
 = \frac{M}{\Delta t} \omega^{n}_{d}
\end{equation}

\begin{equation}
 \left[ K_{xx} + K_{yy} \right] \psi 
 =  M \omega
\end{equation}

\begin{equation}
 \left[
 \frac{M}{\Delta t} 
 + \frac{1}{\textit{ReSc}} \left[ K_{xx} + K_{yy} \right]
 \right] 
 e^{n+1}_{i}
 = \frac{M}{\Delta t} e^{n}_{d}
\end{equation}

\medskip
\noindent
whereas material velocity field \textbf{v} is calculated by:
$u =   G_{y} \psi$ and 
$v = - G_{x} \psi$, 
where 
$G_{y}$ and 
$G_{x}$ are the \textit{Gradient global matrix}.
 

