For the numerical simulation of the 
non-steady convection-diffusion equation 
in an Finite Element context, several methods can be chosen, 
such as upwind, Taylor series in time and 
the characteristic trajectory as is the case of the semi-Lagrangian.

\medskip
The semi-Lagrangian method was initially proposed by 
Sawyer (1963) \cite{sawyer1963} for a numerical simulation 
of atmospheric flow in a finite difference context. 
However, it was only in the 1980s that the 
semi-Lagrangian Method was presented according to 
the Finite Element approach by 
Pironneau (1982) \cite{pironneau1982},
where it is shown the unconditional stability of the method and 
a symmetric linear systems to solve.

\medskip
The material derivate of the Eqs. \ref{vorticity matrix} and \ref{concentration matrix} may be discretized in the time
domain at the $x_{i}$ node by an explicit first order scheme:



\vspace{-0.4cm}
\begin{equation} \label{material vorticity discretization}
 \frac{D \omega}{D t} \approx 
 \frac{\omega_{i}^{n+1} - \omega_{d}^{n}}{\Delta t}
\end{equation}

\vspace{-0.8cm}
\begin{equation} \label{material concentration discretization}
 \frac{D e}{D t} \approx
 \frac{e_{i}^{n+1} - e_{d}^{n}}{\Delta t}
\end{equation}


\medskip
\noindent
where, 
the variable $\Delta t$ is time step, 
$\omega_{i}^{n+1}$ and $e_{i}^{n+1}$ are the vorticity and 
concentration fields calculated in current time step at the current 
node position and
$\omega_{d}^{n}$ and $e_{d}^{n}$ are the vorticity and 
concentration fields calculated in previous time step 
at the departure node position.
The departure node is calculated by solving characteristic equation:

\begin{equation} \label{characteristic}
\frac{d\textbf{x}_{d}}{dt} = \textbf{c}
\end{equation}

\medskip
\noindent
where,
$\textbf{c}$ is the relative velocity and
$t$ is time variable and 
it is varies between $t \in \left[t^{n},t^{n+1}\right]$.
Assuming that in the current time the mesh computational nodes
are known, so the initial condition for the Eq. \ref{characteristic}
can be represented by $\textbf{x}_{d}(t^{n+1}) = \textbf{x}_{i}^{n+1}$,
thus integrating both sides of the equation:

\begin{equation} 
\textbf{x}_{d}^{n+1} - \textbf{x}_{d}^{n} 
= \int_{t^{n}}^{t^{n+1}} \textbf{c} dt, 
\end{equation}

\medskip
\noindent
that is:

\begin{equation} 
\textbf{x}_{d}^{n} = \textbf{x}_{i}^{n+1} 
- \int_{t^{n}}^{t^{n+1}} \textbf{c}  dt, 
\end{equation}


\medskip
\noindent
Considering the relative velocity picewise constant in time,
the departure point can be calculated by:

\begin{equation} \label{final characteristic}
\textbf{x}_{d}^{n} = \textbf{x}_{i}^{n+1} 
- \textbf{c} \Delta t, 
\end{equation}

\medskip
The \ref{characteristic trajectory fig} shown 
the 1-dimensional characteristic trajectory of
the material point in a moving computational mesh,
where the white square is the material point and
the black point is computational mesh.
The dashed line is the characteristic trajectory
that is represented by the Eq. \ref{final characteristic}.
According to this scheme, the initial point $x_i$ in $t^{n+1}$ is known
and therefore it is used to find 
the departure point $x_d$ position in $t^{n}$.


\vspace{0.5cm}
\begin{figure}[H]
\caption{An one-dimensional space scheme
 where the departure node $x_{d}$ is found by 
 the characteristic trajectory in a moving mesh.}
\begin{center}
\vspace{-0.5cm}
\begin{tikzpicture}[scale=6]

 % FIGURE A
 % grid
 \draw (0.5,0.7) -- (1.7,0.7);
 \draw (0.5,1.0) -- (1.7,1.0);
 \draw (0.5,1.2) -- (1.7,1.2);

 \draw (0.60,0.7) -- (0.64,1.0) -- (0.60,1.2);
 \draw (1.20,0.7) -- (1.12,1.0) -- (1.10,1.2);
 \draw (1.54,0.7) -- (1.64,1.0) -- (1.50,1.2);



 % material points
 \draw[dashed,-stealth] (1.12,1.0) -- (0.85,0.75);
 \node[square, fill=white, draw, inner sep=0pt, minimum size=9.4pt] at (0.8,0.7) {};
 \node[square, fill=white, draw, inner sep=0pt, minimum size=9.4pt] at (1.12,1.0) {};
 

 % nodes
 \node[circle, fill=black, inner sep=0pt, minimum size=5.2pt] at (0.60,0.7) {};
 \node[circle, fill=black, inner sep=0pt, minimum size=5.2pt] at (1.20,0.7) {};
 \node[circle, fill=black, inner sep=0pt, minimum size=5.2pt] at (1.54,0.7) {};

 \node[circle, fill=black, inner sep=0pt, minimum size=5.2pt] at (0.64,1.0) {};
 \node[circle, fill=black, inner sep=0pt, minimum size=5.2pt] at (1.12,1.0) {};
 \node[circle, fill=black, inner sep=0pt, minimum size=5.2pt] at (1.64,1.0) {};

 \node[circle, fill=black, inner sep=0pt, minimum size=5.2pt] at (0.60,1.2) {};
 \node[circle, fill=black, inner sep=0pt, minimum size=5.2pt] at (1.10,1.2) {};
 \node[circle, fill=black, inner sep=0pt, minimum size=5.2pt] at (1.50,1.2) {};


 % legend
 \node[draw=none, scale=1.0] at (0.80,0.78) {\small $x_{d}$};
 \node[draw=none, scale=1.0] at (1.19,0.93) {\small $x_{i}$};

 \node[draw=none, scale=1.0] at (1.80,0.70) {\small $t^{n}$};
 \node[draw=none, scale=1.0] at (1.85,1.00) {\small $t^{n+1}$};
 \node[draw=none, scale=1.0] at (1.85,1.20) {\small $t^{n+2}$};

 \node[draw=none, scale=1.0] at (0.60,0.65) {\small $x_{i-1}$};
 \node[draw=none, scale=1.0] at (1.22,0.65) {\small $x_{i}$};
 \node[draw=none, scale=1.0] at (1.64,0.65) {\small $x_{i+1}$};

\end{tikzpicture}
\end{center}
\label{characteristic trajectory fig}
\end{figure}


\medskip
The main advantages of the semi-Lagrangian Method are
the unconditional stability and the symmetric linear system to solve.
Hovewer, the method has a disadvantage: the searching procedure.
The searching procedure in a 2-dimensional context may
lead to excessive computational cost if it is not
well designed. In this work, this iterative procedure is implemented
using the 1-ring neighbors of the initial node, where
the barycentric coordinates system is used in each neighbor element.
If it is not found, the procedure continues with the nearest
neighbor node until it finds the departure point. 
This procedure has presented an acceptable computational cost, 
in addition to the advantage of finding the nodes 
even in concave domains as can be seen in the \ref{searching procedure fig}:

\vspace{-0.5cm}
\begin{figure}[H]
\begin{center}
\begin{tikzpicture}[scale=5]

 % -------------------------------------------------------------------
 % FIGURE B
 % boundary 
 \draw (2.5,1.6) -- (4.0,1.6);
 \draw (2.5,1.6) -- (2.5,1.05);
 \draw (3.0,1.05) -- (3.0,0.5);
 \draw (2.5,1.05) -- (3.0,1.05);
 \draw (3.0,0.5) -- (4.0,0.5);

 % nodes
 \coordinate (A) at (2.500,1.0500) {};
 \coordinate (C) at (2.500,1.3330) {};
 \coordinate (D) at (2.500,1.6000) {};
 \coordinate (F) at (3.000,1.0500) {};
 \coordinate (G) at (2.850,1.3560) {};
 \coordinate (H) at (3.100,1.6000) {};
 \coordinate (I) at (3.000,0.5000) {};
 \coordinate (J) at (3.250,0.8300) {};
 \coordinate (K) at (3.250,1.2530) {};
 \coordinate (L) at (3.400,1.6000) {};
 \coordinate (M) at (3.650,0.5000) {};
 \coordinate (N) at (3.450,0.6200) {};
 \coordinate (O) at (3.620,1.0000) {};
 \coordinate (P) at (3.600,1.3300) {};
 \coordinate (Q) at (2.800,1.6000) {};
 \coordinate (R) at (2.750,1.0500) {};
 \coordinate (S) at (3.000,0.8000) {};

 % elements
 \draw (J) -- (K) -- (F) -- cycle;
 \draw (F) -- (K) -- (G) -- cycle;
 \draw (C) -- (G) -- (D) -- cycle;
 \draw (G) -- (H) -- (D) -- cycle;
 \draw (I) -- (N) -- (J) -- cycle;
 \draw (I) -- (M) -- (N) -- cycle;
 \draw (N) -- (O) -- (J) -- cycle;
 \draw (J) -- (O) -- (K) -- cycle;
 \draw (P) -- (L) -- (K) -- cycle;
 \draw (K) -- (L) -- (H) -- cycle;
 \draw (O) -- (P) -- (K) -- cycle;
 \draw (G) -- (K) -- (H) -- cycle;
 \draw (Q) -- (G) -- (H) -- cycle;
 \draw (R) -- (G) -- (R) -- cycle;
 \draw (S) -- (J) -- (F) -- cycle;
 \draw[red,line width=2pt] (A) -- (C) -- (G) -- cycle;
 

% arrow
 \draw[dashed,line width=0.03cm,-stealth]  (2.57,1.20)  -- (J) ;

 % departure nodes
 \node[square, fill=white, draw, inner sep=0pt, minimum size=7.8pt] at (2.57,1.20) {};
 \node[square, fill=white, draw, inner sep=0pt, minimum size=7.8pt] at (J) {};

 % searching procedure path
 \draw[red, line width=2pt]  (J)--(F) ;
 \draw[red, line width=2pt]  (F)--(R) ;
 \draw[red, line width=2pt]  (R)--(A) ;

 % draw nodes
 \node[circle, fill=black, inner sep=0pt, minimum size=5.5pt] at (A) {};
 \node[circle, fill=black, inner sep=0pt, minimum size=5.5pt] at (C) {};
 \node[circle, fill=black, inner sep=0pt, minimum size=5.5pt] at (D) {};
 \node[circle, fill=black, inner sep=0pt, minimum size=5.5pt] at (F) {};
 \node[circle, fill=black, inner sep=0pt, minimum size=5.5pt] at (G) {};
 \node[circle, fill=black, inner sep=0pt, minimum size=5.5pt] at (H) {};
 \node[circle, fill=black, inner sep=0pt, minimum size=5.5pt] at (I) {};
 \node[circle, fill=black, inner sep=0pt, minimum size=5.5pt] at (J) {};
 \node[circle, fill=black, inner sep=0pt, minimum size=5.5pt] at (K) {};
 \node[circle, fill=black, inner sep=0pt, minimum size=5.5pt] at (L) {};
 \node[circle, fill=black, inner sep=0pt, minimum size=5.5pt] at (M) {};
 \node[circle, fill=black, inner sep=0pt, minimum size=5.5pt] at (N) {};
 \node[circle, fill=black, inner sep=0pt, minimum size=5.5pt] at (O) {};
 \node[circle, fill=black, inner sep=0pt, minimum size=5.5pt] at (P) {};
 \node[circle, fill=black, inner sep=0pt, minimum size=5.5pt] at (Q) {};
 \node[circle, fill=black, inner sep=0pt, minimum size=5.5pt] at (R) {};
 \node[circle, fill=black, inner sep=0pt, minimum size=5.5pt] at (S) {};


 

 
 % legend
 \node[draw=none, scale=1.0] at (3.9,1.0) {\small $domain$};
 \node[draw=none, scale=1.0] at (3.8,1.65) {\small $boundary$};

 \node[draw=none, scale=1.0] at (3.32,0.94) {\small $x_{i}$};
 \node[draw=none, scale=1.0] at (2.65,1.27) {\small $x_{d}$};


\end{tikzpicture}
\end{center}
\caption{
The searching procedure in a 2-dimensional concave domain, 
where the dashed line is the characteristic trajectory of 1st-order,
the red line is the searching procedure path,
the $x_{d}$ is the departure point and
the $x_{i}$ is the initial point.
}
\label{searching procedure fig}
\end{figure}


\medskip
During the searching procedure, three situations may 
occur depending on the trajectory, 
as shown in \ref{three situations fig}: the first and the second 
situations are similar, differentiating only the trajectory length. 
In the first situation, the departure node is inside near 
element from current node, while the second situation 
the departure node is inside far element from current node. 
The third situation, the departure node is outside domain 
then the vorticity and concentration fields receive 
the boundary condition value of nearest node to departure node.


\begin{figure}[H]
\begin{center}
\begin{tikzpicture}[scale=5]
 \draw (2.5,1.6) -- (4.0,1.6);
 \draw (2.5,1.6) -- (2.5,0.5);
 \draw (2.5,0.5) -- (4.0,0.5);

 % nodes
 \coordinate (A) at (2.500,0.5000) {};
 \coordinate (B) at (2.500,0.8667) {};
 \coordinate (C) at (2.500,1.2330) {};
 \coordinate (D) at (2.500,1.6000) {};
 \coordinate (E) at (2.733,0.6830) {};
 \coordinate (F) at (2.833,1.0500) {};
 \coordinate (G) at (2.733,1.4160) {};
 \coordinate (H) at (3.000,1.6000) {};
 \coordinate (I) at (3.066,0.5000) {};
 \coordinate (J) at (3.166,0.8660) {};
 \coordinate (K) at (3.150,1.3330) {};
 \coordinate (L) at (3.333,1.6000) {};
 \coordinate (M) at (3.500,0.5000) {};
 \coordinate (N) at (3.333,0.6830) {};
 \coordinate (O) at (3.400,1.0000) {};
 \coordinate (P) at (3.500,1.2300) {};

 % elements
 \draw (A) -- (E) -- (B) -- cycle;
 \draw (A) -- (I) -- (E) -- cycle;
 \draw (E) -- (J) -- (F) -- cycle;
 \draw (E) -- (F) -- (B) -- cycle;
 \draw (B) -- (F) -- (C) -- cycle;
 \draw (J) -- (K) -- (F) -- cycle;
 \draw (F) -- (K) -- (G) -- cycle;
 \draw (C) -- (F) -- (G) -- cycle;
 \draw (C) -- (G) -- (D) -- cycle;
 \draw (G) -- (H) -- (D) -- cycle;
 \draw (I) -- (N) -- (J) -- cycle;
 \draw (I) -- (M) -- (N) -- cycle;
 \draw (N) -- (O) -- (J) -- cycle;
 \draw (J) -- (O) -- (K) -- cycle;
 \draw (P) -- (L) -- (K) -- cycle;
 \draw (K) -- (L) -- (H) -- cycle;
 \draw (O) -- (P) -- (K) -- cycle;
 \draw[red,line width=2pt] (I) -- (J) -- (E) -- cycle;
 \draw[red,line width=2pt] (B) -- (C);
 \draw[red,line width=2pt] (G) -- (K) -- (H) -- cycle;
 

 % draw nodes
 \node[circle, fill=black, inner sep=0pt, minimum size=5.2pt] at (A) {};
 \node[circle, fill=black, inner sep=0pt, minimum size=5.2pt] at (B) {};
 \node[circle, fill=black, inner sep=0pt, minimum size=5.2pt] at (C) {};
 \node[circle, fill=black, inner sep=0pt, minimum size=5.2pt] at (D) {};
 \node[circle, fill=black, inner sep=0pt, minimum size=5.2pt] at (E) {};
 \node[circle, fill=black, inner sep=0pt, minimum size=5.2pt] at (F) {};
 \node[circle, fill=black, inner sep=0pt, minimum size=5.2pt] at (G) {};
 \node[circle, fill=black, inner sep=0pt, minimum size=5.2pt] at (H) {};
 \node[circle, fill=black, inner sep=0pt, minimum size=5.2pt] at (I) {};
 \node[circle, fill=black, inner sep=0pt, minimum size=5.2pt] at (J) {};
 \node[circle, fill=black, inner sep=0pt, minimum size=5.2pt] at (K) {};
 \node[circle, fill=black, inner sep=0pt, minimum size=5.2pt] at (L) {};
 \node[circle, fill=black, inner sep=0pt, minimum size=5.2pt] at (M) {};
 \node[circle, fill=black, inner sep=0pt, minimum size=5.2pt] at (N) {};
 \node[circle, fill=black, inner sep=0pt, minimum size=5.2pt] at (O) {};
 \node[circle, fill=black, inner sep=0pt, minimum size=5.2pt] at (P) {};


 % arrow
 \draw[dashed,line width=0.03cm]  (2.95,0.65) -- (J) ;
 \draw[dashed,line width=0.03cm]  (2.90,1.45) -- (J) ;
 \draw[dashed,line width=0.03cm]  (2.4,1.05)  -- (J) ;


 % departure nodes
 \node[square, fill=white, draw, inner sep=0pt, minimum size=7.8pt] at (2.95,0.65) {};
 \node[square, fill=white, draw, inner sep=0pt, minimum size=7.8pt] at (2.90,1.45) {};
 \node[square, fill=white, draw, inner sep=0pt, minimum size=7.8pt] at (2.4,1.05) {};
 \node[square, fill=white, draw, inner sep=0pt, minimum size=7.8pt] at (J) {};
 \node[circle, fill=black, inner sep=0pt, minimum size=5.2pt] at (J) {};


 
 % legend
 \node[draw=none, scale=0.9] at (3.23,0.95) {\small $x_{i}$};
 
 \node[draw=none, scale=0.9] at (3.05,0.64) {$1$};
 \node[draw=none, scale=0.9] at (3.00,1.45) {$2$};
 \node[draw=none, scale=0.9] at (2.4,1.15) {$3$};

 \node[draw=none, scale=1.0] at (3.7,1.0) {\small $domain$};
 \node[draw=none, scale=1.0] at (3.8,1.65) {\small $boundary$};

\end{tikzpicture}
\end{center}
\caption{
A two-dimensional space scheme
where three situations may occur in the searching procedure.
}
\label{three situations fig}
\end{figure}


\medskip
After the departure point is found, the vorticity ($\omega_{d}$) and 
concentration ($e_{d}$) fields are interpolated by the shape functions
presented in the section \ref{elemento}.
This interpolation procedure, which is typical in the
Eulerian and Arbitrary Lagrangian-Eulerian descriptions,
causes the numerical diffusion.
To decrease the numerical diffusion, high-order elements
can be used as the quadratic or cubic elements.
Another way is to improve the discretization accuracy of the Eqs.
\ref{material vorticity discretization} and
\ref{material concentration discretization}.
Therefore, the Eqs. 
\ref{vorticity matrix},
\ref{stream matrix} and
\ref{concentration matrix} 
can be shown 
in an
implicit semi-Lagragian 
discretization as:

\begin{equation}
 M \left[ \frac{\omega^{n+1}_{i} - \omega^{n}_{d}}{\Delta t} \right] 
 + \frac{1}{\textit{Re}} \left[ K_{xx} + K_{yy} \right] \omega^{n+1}_{i}
 = 0
\end{equation}

\begin{equation}
 - \left[ K_{xx} + K_{yy} \right] \psi + M \omega = 0
\end{equation}

\begin{equation}
 M \left[ \frac{e^{n+1}_{i} - e^{n}_{d}}{\Delta t} \right]
 + \frac{1}{\textit{ReSc}} \left[ K_{xx} + K_{yy} \right] e^{n+1}_{i}
 = 0
\end{equation}


\medskip
\noindent
that is,

\begin{equation}
 \frac{M}{\Delta t} \omega^{n+1}_{i}
 + \frac{1}{\textit{Re}} \left[ K_{xx} + K_{yy} \right] \omega^{n+1}_{i}
 = \frac{M}{\Delta t} \omega^{n}_{d}
\end{equation}

\begin{equation}
 \left[ K_{xx} + K_{yy} \right] \psi 
 =  M \omega
\end{equation}

\begin{equation}
 \frac{M}{\Delta t} e^{n+1}_{i}
 + \frac{1}{\textit{ReSc}} \left[ K_{xx} + K_{yy} \right] e^{n+1}_{i}
 = \frac{M}{\Delta t} e^{n}_{d}
\end{equation}


\medskip
Therefore, the Vorticity-Streamfunction Formulation with
Species Transport Equation discretized by \textit{Galerkin}
and \textit{semi-Lagrangian Methods} in an ALE-FE context
can be presented in matrix form as:

\begin{equation}
 \left[
 \frac{M}{\Delta t} 
 + \frac{1}{\textit{Re}} \left[ K_{xx} + K_{yy} \right]
 \right] 
 \omega^{n+1}_{i}
 = \frac{M}{\Delta t} \omega^{n}_{d}
\end{equation}

\begin{equation}
 \left[ K_{xx} + K_{yy} \right] \psi 
 =  M \omega
\end{equation}

\begin{equation}
 \left[
 \frac{M}{\Delta t} 
 + \frac{1}{\textit{ReSc}} \left[ K_{xx} + K_{yy} \right]
 \right] 
 e^{n+1}_{i}
 = \frac{M}{\Delta t} e^{n}_{d}
\end{equation}

\medskip
\noindent
whereas material velocity field \textbf{v} is calculated by:
$u =   G_{y} \psi$ and 
$v = - G_{x} \psi$, 
where 
$G_{y}$ and 
$G_{x}$ are the \textit{Gradient global matrix}.
 

