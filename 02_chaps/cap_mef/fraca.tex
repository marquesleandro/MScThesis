O resultado da ponderação da equação de
governo integrada sobre o domínio
é conhecida como
\textbf{formulação fraca} como mencionado por Anjos (2007) \cite{anjos2007}.
A seguir apresentaremos a formulação fraca
para o problema de escoamento de um fluido monofásico,
newtoniano e incompressível utilizando
a formulação corrente-vorticidade com  
equação de transporte de espécie química.
Para mais detalhes, consultar o trabalho de Brenner e Scott (1994) \cite{brenner1994}.
Como o objetivo é encontrar uma solução
aproximada, é aceitável supor que seja produzido
um \textbf{Resíduo R} nas equações de governo,
isto é:

\begin{align}
& \overset{.}{w} + \textbf{v}.\nabla w - \frac{1}{Re} \nabla^2 w 
 - \frac{\Delta t}{2} \textbf{v} \cdot \nabla \big[ \textbf{v} \cdot \nabla w \big]
 = R_1 \\[10pt]
& \nabla^2 \psi + w
 = R_2 \\[10pt]
& \textbf{v} - \textbf{D}\psi
 = R_3 \\[10pt]
& \overset{.}{c} + \textbf{v}.\nabla c - \frac{1}{ReSc} \nabla^2 c
 - \frac{\Delta t}{2} \textbf{v} \cdot \nabla \big[ \textbf{v} \cdot \nabla c \big]
 = R_4
\end{align}

Buscaremos forçar o resíduo ser equivalente
a zero em um sentido médio como mencionado por Finlayson (1972) \cite{finlayson1972}, logo:

\begin{align}
 \int_{\Omega} R_1 \cdot \delta d\Omega &= 0 \\
 \int_{\Omega} R_2 \cdot \phi d\Omega &= 0 \\
 \int_{\Omega} R_3 \cdot \xi d\Omega &= 0 \\
 \int_{\Omega} R_4 \cdot \eta d\Omega &= 0
\end{align}



\noindent
onde $\delta$, $\phi$, $\xi$ e $\eta$ são funções peso. A função peso
é um conjunto de funções arbitrárias
dentro de um espaço de funções que será
discutido à frente. Possuímos, então,
as seguintes integrais:

\begin{equation}
 \int_{\Omega} \Bigg\{ 
 \overset{.}{w} + \textbf{v}.\nabla w 
 - \frac{1}{\textit{Re}} \nabla^2 w 
 - \frac{\Delta t}{2} \textbf{v} \cdot \nabla \big[ \textbf{v} \cdot \nabla w \big]
\Bigg\} \cdot \delta d\Omega = 0
\end{equation}

\begin{equation}
 \int_{\Omega} \big\{ \nabla^2 \psi + w \big\} \cdot \phi d\Omega = 0
\end{equation}

\begin{equation}
 \int_{\Omega} \big\{ \textbf{v} - \textbf{D}\psi \big\} \cdot \xi d\Omega = 0
\end{equation}

\begin{equation}
 \int_{\Omega} \Bigg\{ 
 \overset{.}{c} + \textbf{v}.\nabla c 
 - \frac{1}{\textit{ReSc}} \nabla^2 c 
 - \frac{\Delta t}{2} \textbf{v} \cdot \nabla \big[ \textbf{v} \cdot \nabla c \big]
 \Bigg\} \cdot \eta d\Omega = 0
\end{equation}



\noindent
Desenvolvendo a integral, temos:

\begin{equation} \label{diffusion_vorticity} 
   \int_{\Omega} \overset{.}{w} \delta d\Omega 
 + \int_{\Omega} \textbf{v}.\nabla w \delta d\Omega 
 - \frac{1}{\textit{Re}} \int_{\Omega} \nabla^2 w \delta 
 - \frac{\Delta t}{2} \int_{\Omega} \textbf{v} \cdot \nabla \big[ \textbf{v} \cdot \nabla w \big] \delta d\Omega
 = 0
\end{equation}

\begin{equation} \label{diffusion_stream}
   \int_{\Omega} \nabla^2 \psi \phi d\Omega
 + \int_{\Omega} w \phi d\Omega = 0
\end{equation}

\begin{equation}
   \int_{\Omega} \textbf{v} \xi d\Omega
 - \int_{\Omega} \textbf{D}\psi \xi d\Omega = 0
\end{equation}

\begin{equation} \label{diffusion_concentration} 
   \int_{\Omega} \overset{.}{c} \eta d\Omega
 + \int_{\Omega} \textbf{v}.\nabla c \eta d\Omega
 - \frac{1}{\textit{ReSc}} \int_{\Omega} \nabla^2 c \eta d\Omega 
 - \frac{\Delta t}{2} \int_{\Omega} \textbf{v} \cdot \nabla \big[ \textbf{v} \cdot \nabla c \big] \eta d\Omega
 = 0
\end{equation}

\medskip
No termo difusivo das equações (\ref{diffusion_vorticity}, \ref{diffusion_stream} e \ref{diffusion_concentration}),
aplicaremos o teorema de Green com o intuito de diminuir a ordem
da derivada e separar o termo avaliado no contorno. Assim o termo difusivo se transformará em:

\begin{equation} \label{diffusion2_vorticity} 
 - \frac{1}{\textit{Re}} \int_{\Omega} \nabla^2 w \delta d\Omega
 = \frac{1}{\textit{Re}} \int_{\Omega} \nabla w \cdot \nabla \delta d\Omega
 - \frac{1}{\textit{Re}} \int_{\Gamma} \delta \nabla w \cdot \textbf{n} d\Gamma
\end{equation}

\begin{equation} \label{diffusion2_stream} 
 \int_{\Omega} \nabla^2 \psi \phi d\Omega
 = - \int_{\Omega} \nabla \psi \cdot \nabla \phi d\Omega
   + \int_{\Gamma} \phi \nabla \psi \cdot \textbf{n} d\Gamma
\end{equation}

\begin{equation} \label{diffusion2_concentration} 
 - \frac{1}{\textit{ReSc}} \int_{\Omega} \nabla^2 c \eta d\Omega
 = \frac{1}{\textit{ReSc}} \int_{\Omega} \nabla c \cdot \nabla \eta d\Omega
 - \frac{1}{\textit{ReSc}} \int_{\Gamma} \eta \nabla c \cdot \textbf{n} d\Gamma
\end{equation}

\noindent
onde \textbf{n} é o vetor normal orientado para fora do contorno $\Gamma$.
O último termo das equações acima é conhecido como condição natural. Conforme apresentado na seção \ref{formulacao forte},
para o problema proposto neste trabalho, possuímos apenas condições de contorno de Dirichlet 
(conhecida como condição essencial). Desta forma, assumiremos por hipótese que
$\delta = 0$, $\phi = 0$ e $\eta = 0$ nas equações (\ref{diffusion2_vorticity}, \ref{diffusion2_stream} e \ref{diffusion2_concentration})
para todo contorno $\Gamma$. Portanto, a integral em $\Gamma$ é nula.
Assim, o termo difusivo das equações (\ref{diffusion_vorticity}, \ref{diffusion_stream} e \ref{diffusion_concentration}) será:

\begin{equation}
 - \frac{1}{\textit{Re}} \int_{\Omega} \nabla^2 w \delta d\Omega
 = \frac{1}{\textit{Re}} \int_{\Omega} \nabla w \cdot \nabla \delta d\Omega
\end{equation}

\begin{equation} 
 \int_{\Omega} \nabla^2 \psi \phi d\Omega
 = - \int_{\Omega} \nabla \psi \cdot \nabla \phi d\Omega
\end{equation}

\begin{equation} 
 - \frac{1}{\textit{ReSc}} \int_{\Omega} \nabla^2 c \eta d\Omega
 = \frac{1}{\textit{ReSc}} \int_{\Omega} \nabla c \cdot \nabla \eta d\Omega
\end{equation}

\medskip
Para o termo de difusividade numérica das equações
(\ref{diffusion_vorticity} e \ref{diffusion_concentration}),
aplicaremos um procedimento semelhante ao anterior. 
Dessa forma, o termo de difusividade numérica será:

\begin{equation}
 - \frac{\Delta t}{2} \int_{\Omega} \textbf{v} \cdot \nabla \big[ \textbf{v} \cdot \nabla w \big] \delta d\Omega
 = 
 \frac{\Delta t}{2} \int_{\Omega} \big[ \textbf{v} \cdot \nabla w \big] \textbf{v} \cdot \nabla \delta d\Omega
 - \frac{\Delta t}{2} \int_{\Gamma} \big[ \textbf{v} \cdot \nabla w \big] \delta \textbf{v} \cdot \textbf{n} d\Gamma
\end{equation}

\begin{equation}
 - \frac{\Delta t}{2} \int_{\Omega} \textbf{v} \cdot \nabla \big[ \textbf{v} \cdot \nabla c \big] \eta d\Omega
 = 
 \frac{\Delta t}{2} \int_{\Omega} \big[ \textbf{v} \cdot \nabla c \big] \textbf{v} \cdot \nabla \eta d\Omega
 - \frac{\Delta t}{2} \int_{\Gamma} \big[ \textbf{v} \cdot \nabla c \big] \eta \textbf{v} \cdot \textbf{n} d\Gamma
\end{equation}


\newpage
Da mesma forma, podemos anular o termo de condição
natural segundo a hipótese adotada anteriormente.
Portanto, o termo de difusividade numérica das
equações (\ref{diffusion_vorticity} e \ref{diffusion_concentration}) será:

\begin{equation}
 - \frac{\Delta t}{2} \int_{\Omega} \textbf{v} \cdot \nabla \big[ \textbf{v} \cdot \nabla w \big] \delta d\Omega
 = 
 \frac{\Delta t}{2} \int_{\Omega} \big[ \textbf{v} \cdot \nabla w \big] \textbf{v} \cdot \nabla \delta d\Omega
\end{equation}

\begin{equation}
 - \frac{\Delta t}{2} \int_{\Omega} \textbf{v} \cdot \nabla \big[ \textbf{v} \cdot \nabla c \big] \eta d\Omega
 = 
 \frac{\Delta t}{2} \int_{\Omega} \big[ \textbf{v} \cdot \nabla c \big] \textbf{v} \cdot \nabla \eta d\Omega
\end{equation}

\medskip
\noindent
Portanto, substituindo os novos termos difusivos nas equações de governo:

\begin{equation} \label{vorticity weak} 
   \int_{\Omega} \overset{.}{w} \delta d\Omega 
 + \int_{\Omega} \textbf{v}.\nabla w \delta d\Omega 
 + \frac{1}{\textit{Re}} \int_{\Omega} \nabla w \cdot \nabla \delta d\Omega 
 + \frac{\Delta t}{2} \int_{\Omega} \big[ \textbf{v} \cdot \nabla w \big] \textbf{v} \cdot \nabla \delta d\Omega
 = 0
\end{equation}

\begin{equation}
 - \int_{\Omega} \nabla \psi \cdot \nabla \phi d\Omega
 + \int_{\Omega} w \phi d\Omega = 0
\end{equation}

\begin{equation}
   \int_{\Omega} \textbf{v} \xi d\Omega
 - \int_{\Omega} \textbf{D}\psi \xi d\Omega = 0
\end{equation}

\begin{equation} \label{concentration weak}
   \int_{\Omega} \overset{.}{c} \eta d\Omega
 + \int_{\Omega} \textbf{v}.\nabla c \eta d\Omega
 + \frac{1}{\textit{ReSc}} \int_{\Omega} \nabla c \cdot \nabla \eta d\Omega 
 + \frac{\Delta t}{2} \int_{\Omega} \big[ \textbf{v} \cdot \nabla c \big] \textbf{v} \cdot \nabla \eta d\Omega
 = 0
\end{equation}

\medskip
\noindent
Se assumirmos que:

\begin{equation}
 \begin{aligned}
  \textbf{$m_1$}(\overset{.}{w},\delta) & = \int_{\Omega} \overset{.}{w} \delta d\Omega \\ 
  \textbf{$g_1$}(\textbf{v},\delta) & = \int_{\Omega} \textbf{v}.\nabla w \delta d\Omega \\
  \textbf{$k_1$}(w,\delta) & = \int_{\Omega} \nabla w \cdot \nabla \delta d\Omega \\
  \textbf{$k_{n1}$}(w,\delta) & = \int_{\Omega} \big[ \textbf{v} \cdot \nabla w \big] \textbf{v} \cdot \nabla \delta d\Omega \\
  \textbf{$k_2$}(\psi,\phi) & = \int_{\Omega} \nabla \psi \cdot \nabla \phi d\Omega \\
  \textbf{$m_2$}(\psi,\phi) & = \int_{\Omega} w \phi d\Omega \\
 \end{aligned}
 \qquad
 \begin{aligned}  
  \textbf{$m_3$}(\textbf{v},\xi) & = \int_{\Omega} \textbf{v} \xi d\Omega \\
  \textbf{$g_3$}(\psi,\xi) & = \int_{\Omega} \textbf{D}\psi \xi d\Omega \\
  \textbf{$m_4$}(\overset{.}{c},\eta) & = \int_{\Omega} \overset{.}{c} \eta d\Omega \\
  \textbf{$g_4$}(\textbf{v},\eta) & = \int_{\Omega} \textbf{v}.\nabla c \eta d\Omega \\
  \textbf{$k_4$}(c,\eta) & = \int_{\Omega} \nabla c \cdot \nabla \eta d\Omega \\
  \textbf{$k_{n4}$}(c,\delta) & = \int_{\Omega} \big[ \textbf{v} \cdot \nabla c \big] \textbf{v} \cdot \nabla \eta d\Omega 
 \end{aligned}
\end{equation}

\noindent
Então as equações podem ser apresentadas em sua respectiva forma fraca como:

\begin{align}
 \textbf{$m_1$}(\overset{.}{w},\delta) + \textbf{$g_1$}(\textbf{v},\delta) 
 + \frac{1}{\textit{Re}}\textbf{$k_1$}(w,\delta) + \frac{\Delta t}{2} \textbf{$k_{n1}$}(w,\delta) &= 0 \\
 -\textbf{$k_2$}(\psi,\phi) + \textbf{$m_2$}(\psi,\phi) &= 0 \\
  \textbf{$m_3$}(\textbf{v},\xi) - \textbf{$g_3$}(\psi,\xi) &= 0 \\
 \textbf{$m_4$}(\overset{.}{c},\eta) + \textbf{$g_4$}(\textbf{v},\eta) 
 + \frac{1}{\textit{ReSc}}\textbf{$k_4$}(c,\eta) +\frac{\Delta t}{2} \textbf{$k_{n4}$}(c,\eta) &= 0
\end{align}


\noindent
Dados os conjuntos de funções bases:

\begin{equation}
 \begin{aligned}
  \mathbb{W} &= \{w \in \Omega \rightarrow \mathbb{R}^2
  : \int_\Omega w^2 d\Omega < \infty 
  ; w = w_\Gamma\} \\
  \mathbb{P} &= \{\psi \in \Omega \rightarrow \mathbb{R}^2
  : \int_\Omega \psi^2 d\Omega < \infty 
  ; \psi = \psi_\Gamma\} \\
  \mathbb{V} &= \{v \in \Omega \rightarrow \mathbb{R}^2
  : \int_\Omega v^2 d\Omega < \infty 
  ; v = v_\Gamma\} \\
  \mathbb{C} &= \{c \in \Omega \rightarrow \mathbb{R}^2
  : \int_\Omega c^2 d\Omega < \infty 
  ; c = c_\Gamma\}
 \end{aligned}
\end{equation}

\noindent
E Dados os espaços de funções pesos:

\begin{equation}
 \begin{aligned}
  \mathbb{D} &= \{\delta \in \Omega \rightarrow \mathbb{R}^2
  : \int_\Omega \delta^2 d\Omega < \infty
  ; \delta_\Gamma = 0\} \\
  \mathbb{F} &= \{\phi \in \Omega \rightarrow \mathbb{R}^2
  : \int_\Omega \phi^2 d\Omega < \infty 
  ; \phi_\Gamma = 0\} \\
  \mathbb{X} &= \{\delta \in \Omega \rightarrow \mathbb{R}^2
  : \int_\Omega \xi^2 d\Omega < \infty 
  ; \xi_\Gamma = 0\} \\
  \mathbb{N} &= \{\eta \in \Omega \rightarrow \mathbb{R}^2
  : \int_\Omega \eta^2 d\Omega < \infty 
  ; \eta_\Gamma = 0\}
 \end{aligned}
\end{equation}

\noindent
A formulação fraca consiste em
encontrarmos as soluções de $w \in \mathbb{W}$,
$\psi \in \mathbb{P}$, $v \in \mathbb{V}$ e $c \in \mathbb{C}$
tal que:

\begin{align}
 \textbf{$m_1$}(\overset{.}{w},\delta) + \textbf{$g_1$}(\textbf{v},\delta) 
 + \frac{1}{\textit{Re}}\textbf{$k_1$}(w,\delta) + \frac{\Delta t}{2} \textbf{$k_{n1}$}(w,\delta) &= 0 \\
 -\textbf{$k_2$}(\psi,\phi) + \textbf{$m_2$}(\psi,\phi) &= 0 \\
  \textbf{$m_3$}(\textbf{v},\xi) - \textbf{$g_3$}(\psi,\xi) &= 0 \\
 \textbf{$m_4$}(\overset{.}{c},\eta) + \textbf{$g_4$}(\textbf{v},\eta) 
 + \frac{1}{\textit{ReSc}}\textbf{$k_4$}(c,\eta) +\frac{\Delta t}{2} \textbf{$k_{n4}$}(c,\eta) &= 0
\end{align}


\noindent
para todo $\delta \in \mathbb{D}$, $\phi \in \mathbb{F}$, 
$\xi \in \mathbb{X}$ e $\eta \in \mathbb{N}$.
