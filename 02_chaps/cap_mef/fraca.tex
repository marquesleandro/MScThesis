The result of the weighted governing equations over the domain 
is known as \textbf{Weak formulation}.
Then, the weak formulation 
for a single-phase, Newtonian and incompressible fluid using 
the vorticity-streamfunction formulation with species transport equation
will be shown, for more details see the work of Brenner and Scott (1994) 
\cite{brenner1994}. 
Therefore, as the objective is to find an approximate solution, 
it is acceptable to assume that a \textbf{Residue R} 
is produced in the governing equations, that is:


\begin{align}
& \overset{.}{w} + \textbf{v}.\nabla w - \frac{1}{Re} \nabla^2 w 
 - \frac{\Delta t}{2} \textbf{v} \cdot \nabla \big[ \textbf{v} \cdot \nabla w \big]
 = R_1 \\[10pt]
& \nabla^2 \psi + w
 = R_2 \\[10pt]
& \textbf{v} - \textbf{D}\psi
 = R_3 \\[10pt]
& \overset{.}{c} + \textbf{v}.\nabla c - \frac{1}{ReSc} \nabla^2 c
 - \frac{\Delta t}{2} \textbf{v} \cdot \nabla \big[ \textbf{v} \cdot \nabla c \big]
 = R_4
\end{align}

We will try to force the residue to be equivalent to zero 
in a weighted sense as mentioned by Finlayson (1972) \cite{finlayson1972}, then:

\begin{align}
 \int_{\Omega} R_1 \cdot \delta d\Omega &= 0 \\
 \int_{\Omega} R_2 \cdot \phi d\Omega &= 0 \\
 \int_{\Omega} R_3 \cdot \xi d\Omega &= 0 \\
 \int_{\Omega} R_4 \cdot \eta d\Omega &= 0
\end{align}



\noindent
where $\delta$, $\phi$, $\xi$ e $\eta$ are weight function.
The weight functions are a set of arbitrary functions 
that belong to a function space that will be discussed later. 
We then have the following integrals:

\begin{equation}
 \int_{\Omega} \Bigg\{ 
 \overset{.}{w} + \textbf{v}.\nabla w 
 - \frac{1}{\textit{Re}} \nabla^2 w 
 - \frac{\Delta t}{2} \textbf{v} \cdot \nabla \big[ \textbf{v} \cdot \nabla w \big]
\Bigg\} \cdot \delta d\Omega = 0
\end{equation}

\begin{equation}
 \int_{\Omega} \big\{ \nabla^2 \psi + w \big\} \cdot \phi d\Omega = 0
\end{equation}

\begin{equation}
 \int_{\Omega} \big\{ \textbf{v} - \textbf{D}\psi \big\} \cdot \xi d\Omega = 0
\end{equation}

\begin{equation}
 \int_{\Omega} \Bigg\{ 
 \overset{.}{c} + \textbf{v}.\nabla c 
 - \frac{1}{\textit{ReSc}} \nabla^2 c 
 - \frac{\Delta t}{2} \textbf{v} \cdot \nabla \big[ \textbf{v} \cdot \nabla c \big]
 \Bigg\} \cdot \eta d\Omega = 0
\end{equation}



\noindent
Developing the integrals, we have:

\begin{equation} \label{diffusion_vorticity} 
   \int_{\Omega} \overset{.}{w} \delta d\Omega 
 + \int_{\Omega} \textbf{v}.\nabla w \delta d\Omega 
 - \frac{1}{\textit{Re}} \int_{\Omega} \nabla^2 w \delta 
 - \frac{\Delta t}{2} \int_{\Omega} \textbf{v} \cdot \nabla \big[ \textbf{v} \cdot \nabla w \big] \delta d\Omega
 = 0
\end{equation}

\begin{equation} \label{diffusion_stream}
   \int_{\Omega} \nabla^2 \psi \phi d\Omega
 + \int_{\Omega} w \phi d\Omega = 0
\end{equation}

\begin{equation}
   \int_{\Omega} \textbf{v} \xi d\Omega
 - \int_{\Omega} \textbf{D}\psi \xi d\Omega = 0
\end{equation}

\begin{equation} \label{diffusion_concentration} 
   \int_{\Omega} \overset{.}{c} \eta d\Omega
 + \int_{\Omega} \textbf{v}.\nabla c \eta d\Omega
 - \frac{1}{\textit{ReSc}} \int_{\Omega} \nabla^2 c \eta d\Omega 
 - \frac{\Delta t}{2} \int_{\Omega} \textbf{v} \cdot \nabla \big[ \textbf{v} \cdot \nabla c \big] \eta d\Omega
 = 0
\end{equation}

\medskip
In the diffusive term of the equations (\ref{diffusion_vorticity}, \ref{diffusion_stream} and \ref{diffusion_concentration}),
we will apply Green's theorem in order to decrease 
the derivative order and separate the term evaluated in the boundary. 
Thus the diffusive term will become:

\begin{equation} \label{diffusion2_vorticity} 
 - \frac{1}{\textit{Re}} \int_{\Omega} \nabla^2 w \delta d\Omega
 = \frac{1}{\textit{Re}} \int_{\Omega} \nabla w \cdot \nabla \delta d\Omega
 - \frac{1}{\textit{Re}} \int_{\Gamma} \delta \nabla w \cdot \textbf{n} d\Gamma
\end{equation}

\begin{equation} \label{diffusion2_stream} 
 \int_{\Omega} \nabla^2 \psi \phi d\Omega
 = - \int_{\Omega} \nabla \psi \cdot \nabla \phi d\Omega
   + \int_{\Gamma} \phi \nabla \psi \cdot \textbf{n} d\Gamma
\end{equation}

\begin{equation} \label{diffusion2_concentration} 
 - \frac{1}{\textit{ReSc}} \int_{\Omega} \nabla^2 c \eta d\Omega
 = \frac{1}{\textit{ReSc}} \int_{\Omega} \nabla c \cdot \nabla \eta d\Omega
 - \frac{1}{\textit{ReSc}} \int_{\Gamma} \eta \nabla c \cdot \textbf{n} d\Gamma
\end{equation}

\noindent
where \textbf{n} is the normal vector oriented outside the $\Gamma$. 
The last term in the above equations is known as a natural condition. 
As presented in the section \ref{formulacao forte}, 
for the problem proposed in this work, we have only Dirichlet conditions 
(known as an essential condition).

Therefore,
we will consider 
$\delta = 0$, $\phi = 0$ and $\eta = 0$ assumptions in the equations (\ref{diffusion2_vorticity}, \ref{diffusion2_stream} and \ref{diffusion2_concentration})
for all $\Gamma$. 
Thereby, 
the integral in $\Gamma$ will be null
and the diffusive term of the equations 
(\ref{diffusion_vorticity}, \ref{diffusion_stream} and \ref{diffusion_concentration}) become:

\begin{equation}
 - \frac{1}{\textit{Re}} \int_{\Omega} \nabla^2 w \delta d\Omega
 = \frac{1}{\textit{Re}} \int_{\Omega} \nabla w \cdot \nabla \delta d\Omega
\end{equation}

\begin{equation} 
 \int_{\Omega} \nabla^2 \psi \phi d\Omega
 = - \int_{\Omega} \nabla \psi \cdot \nabla \phi d\Omega
\end{equation}

\begin{equation} 
 - \frac{1}{\textit{ReSc}} \int_{\Omega} \nabla^2 c \eta d\Omega
 = \frac{1}{\textit{ReSc}} \int_{\Omega} \nabla c \cdot \nabla \eta d\Omega
\end{equation}

\medskip
For the numerical diffusivity term of the equations 
(\ref{diffusion_vorticity} and \ref{diffusion_concentration}), 
we will apply a procedure similar to the previous one. 
Thus, the term of numerical diffusivity will be:


\begin{equation}
 - \frac{\Delta t}{2} \int_{\Omega} \textbf{v} \cdot \nabla \big[ \textbf{v} \cdot \nabla w \big] \delta d\Omega
 = 
 \frac{\Delta t}{2} \int_{\Omega} \big[ \textbf{v} \cdot \nabla w \big] \textbf{v} \cdot \nabla \delta d\Omega
 - \frac{\Delta t}{2} \int_{\Gamma} \big[ \textbf{v} \cdot \nabla w \big] \delta \textbf{v} \cdot \textbf{n} d\Gamma
\end{equation}

\begin{equation}
 - \frac{\Delta t}{2} \int_{\Omega} \textbf{v} \cdot \nabla \big[ \textbf{v} \cdot \nabla c \big] \eta d\Omega
 = 
 \frac{\Delta t}{2} \int_{\Omega} \big[ \textbf{v} \cdot \nabla c \big] \textbf{v} \cdot \nabla \eta d\Omega
 - \frac{\Delta t}{2} \int_{\Gamma} \big[ \textbf{v} \cdot \nabla c \big] \eta \textbf{v} \cdot \textbf{n} d\Gamma
\end{equation}


\newpage
In the same way, we can cancel the natural condition term 
according to the assumption previously adopted. 
Therefore, the numeric diffusivity term for the equations 
(\ref{diffusion_vorticity} and \ref{diffusion_concentration}) will be:

\begin{equation}
 - \frac{\Delta t}{2} \int_{\Omega} \textbf{v} \cdot \nabla \big[ \textbf{v} \cdot \nabla w \big] \delta d\Omega
 = 
 \frac{\Delta t}{2} \int_{\Omega} \big[ \textbf{v} \cdot \nabla w \big] \textbf{v} \cdot \nabla \delta d\Omega
\end{equation}

\begin{equation}
 - \frac{\Delta t}{2} \int_{\Omega} \textbf{v} \cdot \nabla \big[ \textbf{v} \cdot \nabla c \big] \eta d\Omega
 = 
 \frac{\Delta t}{2} \int_{\Omega} \big[ \textbf{v} \cdot \nabla c \big] \textbf{v} \cdot \nabla \eta d\Omega
\end{equation}

\medskip
\noindent
Therefore, replacing the new diffusive terms in the governing equations:

\begin{equation} \label{vorticity weak} 
   \int_{\Omega} \overset{.}{w} \delta d\Omega 
 + \int_{\Omega} \textbf{v}.\nabla w \delta d\Omega 
 + \frac{1}{\textit{Re}} \int_{\Omega} \nabla w \cdot \nabla \delta d\Omega 
 + \frac{\Delta t}{2} \int_{\Omega} \big[ \textbf{v} \cdot \nabla w \big] \textbf{v} \cdot \nabla \delta d\Omega
 = 0
\end{equation}

\begin{equation}
 - \int_{\Omega} \nabla \psi \cdot \nabla \phi d\Omega
 + \int_{\Omega} w \phi d\Omega = 0
\end{equation}

\begin{equation}
   \int_{\Omega} \textbf{v} \xi d\Omega
 - \int_{\Omega} \textbf{D}\psi \xi d\Omega = 0
\end{equation}

\begin{equation} \label{concentration weak}
   \int_{\Omega} \overset{.}{c} \eta d\Omega
 + \int_{\Omega} \textbf{v}.\nabla c \eta d\Omega
 + \frac{1}{\textit{ReSc}} \int_{\Omega} \nabla c \cdot \nabla \eta d\Omega 
 + \frac{\Delta t}{2} \int_{\Omega} \big[ \textbf{v} \cdot \nabla c \big] \textbf{v} \cdot \nabla \eta d\Omega
 = 0
\end{equation}

\medskip
\noindent
If we consider that:

\begin{equation}
 \begin{aligned}
  \textbf{$m_1$}(\overset{.}{w},\delta) & = \int_{\Omega} \overset{.}{w} \delta d\Omega \\ 
  \textbf{$g_1$}(\textbf{v},\delta) & = \int_{\Omega} \textbf{v}.\nabla w \delta d\Omega \\
  \textbf{$k_1$}(w,\delta) & = \int_{\Omega} \nabla w \cdot \nabla \delta d\Omega \\
  \textbf{$k_{n1}$}(w,\delta) & = \int_{\Omega} \big[ \textbf{v} \cdot \nabla w \big] \textbf{v} \cdot \nabla \delta d\Omega \\
  \textbf{$k_2$}(\psi,\phi) & = \int_{\Omega} \nabla \psi \cdot \nabla \phi d\Omega \\
  \textbf{$m_2$}(\psi,\phi) & = \int_{\Omega} w \phi d\Omega \\
 \end{aligned}
 \qquad
 \begin{aligned}  
  \textbf{$m_3$}(\textbf{v},\xi) & = \int_{\Omega} \textbf{v} \xi d\Omega \\
  \textbf{$g_3$}(\psi,\xi) & = \int_{\Omega} \textbf{D}\psi \xi d\Omega \\
  \textbf{$m_4$}(\overset{.}{c},\eta) & = \int_{\Omega} \overset{.}{c} \eta d\Omega \\
  \textbf{$g_4$}(\textbf{v},\eta) & = \int_{\Omega} \textbf{v}.\nabla c \eta d\Omega \\
  \textbf{$k_4$}(c,\eta) & = \int_{\Omega} \nabla c \cdot \nabla \eta d\Omega \\
  \textbf{$k_{n4}$}(c,\delta) & = \int_{\Omega} \big[ \textbf{v} \cdot \nabla c \big] \textbf{v} \cdot \nabla \eta d\Omega 
 \end{aligned}
\end{equation}

\noindent
the equations, thus, can be shown in the weak form respectively:

\begin{align}
 \textbf{$m_1$}(\overset{.}{w},\delta) + \textbf{$g_1$}(\textbf{v},\delta) 
 + \frac{1}{\textit{Re}}\textbf{$k_1$}(w,\delta) + \frac{\Delta t}{2} \textbf{$k_{n1}$}(w,\delta) &= 0 \\
 -\textbf{$k_2$}(\psi,\phi) + \textbf{$m_2$}(\psi,\phi) &= 0 \\
  \textbf{$m_3$}(\textbf{v},\xi) - \textbf{$g_3$}(\psi,\xi) &= 0 \\
 \textbf{$m_4$}(\overset{.}{c},\eta) + \textbf{$g_4$}(\textbf{v},\eta) 
 + \frac{1}{\textit{ReSc}}\textbf{$k_4$}(c,\eta) +\frac{\Delta t}{2} \textbf{$k_{n4}$}(c,\eta) &= 0
\end{align}


\noindent
Considering the sets of basis functions:

\begin{equation}
 \begin{aligned}
  \mathbb{W} &= \{w \in \Omega \rightarrow \mathbb{R}^2
  : \int_\Omega w^2 d\Omega < \infty 
  ; w = w_\Gamma\} \\
  \mathbb{P} &= \{\psi \in \Omega \rightarrow \mathbb{R}^2
  : \int_\Omega \psi^2 d\Omega < \infty 
  ; \psi = \psi_\Gamma\} \\
  \mathbb{V} &= \{v \in \Omega \rightarrow \mathbb{R}^2
  : \int_\Omega v^2 d\Omega < \infty 
  ; v = v_\Gamma\} \\
  \mathbb{C} &= \{c \in \Omega \rightarrow \mathbb{R}^2
  : \int_\Omega c^2 d\Omega < \infty 
  ; c = c_\Gamma\}
 \end{aligned}
\end{equation}

\noindent
and the set of weight functions space:

\begin{equation}
 \begin{aligned}
  \mathbb{D} &= \{\delta \in \Omega \rightarrow \mathbb{R}^2
  : \int_\Omega \delta^2 d\Omega < \infty
  ; \delta_\Gamma = 0\} \\
  \mathbb{F} &= \{\phi \in \Omega \rightarrow \mathbb{R}^2
  : \int_\Omega \phi^2 d\Omega < \infty 
  ; \phi_\Gamma = 0\} \\
  \mathbb{X} &= \{\delta \in \Omega \rightarrow \mathbb{R}^2
  : \int_\Omega \xi^2 d\Omega < \infty 
  ; \xi_\Gamma = 0\} \\
  \mathbb{N} &= \{\eta \in \Omega \rightarrow \mathbb{R}^2
  : \int_\Omega \eta^2 d\Omega < \infty 
  ; \eta_\Gamma = 0\}
 \end{aligned}
\end{equation}

\noindent
The weak formulation consists 
to find the solutions of 
$w \in \mathbb{W}$,
$\psi \in \mathbb{P}$, $v \in \mathbb{V}$ and $c \in \mathbb{C}$
such that:

\begin{align}
 \textbf{$m_1$}(\overset{.}{w},\delta) + \textbf{$g_1$}(\textbf{v},\delta) 
 + \frac{1}{\textit{Re}}\textbf{$k_1$}(w,\delta) + \frac{\Delta t}{2} \textbf{$k_{n1}$}(w,\delta) &= 0 \\
 -\textbf{$k_2$}(\psi,\phi) + \textbf{$m_2$}(\psi,\phi) &= 0 \\
  \textbf{$m_3$}(\textbf{v},\xi) - \textbf{$g_3$}(\psi,\xi) &= 0 \\
 \textbf{$m_4$}(\overset{.}{c},\eta) + \textbf{$g_4$}(\textbf{v},\eta) 
 + \frac{1}{\textit{ReSc}}\textbf{$k_4$}(c,\eta) +\frac{\Delta t}{2} \textbf{$k_{n4}$}(c,\eta) &= 0
\end{align}


\noindent
for all $\delta \in \mathbb{D}$, $\phi \in \mathbb{F}$, 
$\xi \in \mathbb{X}$ and $\eta \in \mathbb{N}$.
