Governing equations in differential form 
with boundary conditions are known as \textbf{Strong Formulation}. 
Thus, the strong formulation for the proposed problem is:

\begin{align}
& \frac{D w}{D t}
 =
 \frac{1}{Re} \nabla^{2} w
 \\[10pt] 
& \nabla^{2} \psi
 = 
 - 
 w \\[10pt]
& \frac{D e}{Dt}
 =
 \frac{1}{ReSc} \nabla^{2} e
\end{align}


\medskip
\noindent
These equations are valid in 
$\Omega \subset \mathbb{R}^2$ domain
with the following boundary conditions:

\begin{equation} \label{bc}
 \begin{aligned}
  w = w_\Gamma \quad & \mbox{in $\Gamma_1$}\\
  \psi = \psi_\Gamma \quad & \mbox{in $\Gamma_2$}\\
  e = e_\Gamma \quad & \mbox{in $\Gamma_4$}
\end{aligned}
\end{equation}

\vspace{1cm}
The result of the weighted governing equations over the domain 
is known as \textbf{Weak formulation}.
Then, the weak formulation 
for a single-phase, Newtonian and incompressible fluid using 
the vorticity-streamfunction formulation with species transport equation
will be shown, for more details see the work of Brenner and Scott (1994) 
\cite{brenner1994}. 
Therefore, as the objective is to find an approximate solution, 
it is acceptable to assume that a Residue \textbf{R} 
is produced in the governing equations, that is:


\begin{align}
& \frac{D w}{D t}
 -
 \frac{1}{Re} \nabla^{2} w
 = R_1 \\[10pt] 
& \nabla^{2} \psi
 + 
 w 
 = R_2 \\[10pt]
& \frac{D e}{Dt}
 -
 \frac{1}{ReSc} \nabla^{2} e
 = R_3
\end{align}

\medskip
We will force the residue to be equivalent to zero 
in a weighted sense, as mentioned by Finlayson (1972) \cite{finlayson1972}, then:

\begin{align}
 \int_{\Omega} R_1 \cdot \delta d\Omega &= 0 \\
 \int_{\Omega} R_2 \cdot \phi d\Omega &= 0 \\
 \int_{\Omega} R_3 \cdot \eta d\Omega &= 0
\end{align}



\noindent
where $\delta$, $\phi$ and $\eta$ are weight function.
The weight functions are a set of arbitrary functions 
that belong to a function space that will be discussed later. 
We then have the following integrals:

\begin{equation}
 \int_{\Omega} \Bigg\{ 
 \frac{D w}{D t}
 -
 \frac{1}{Re} \nabla^{2} w
\Bigg\} \cdot \delta d\Omega = 0
\end{equation}

\begin{equation}
 \int_{\Omega} \big\{ \nabla^2 \psi + w \big\} \cdot \phi d\Omega = 0
\end{equation}

\begin{equation}
 \int_{\Omega} \Bigg\{ 
 \frac{D e}{Dt}
 -
 \frac{1}{ReSc} \nabla^{2} e
 \Bigg\} \cdot \eta d\Omega = 0
\end{equation}



\noindent
Developing the integrals, we have:

\begin{equation} \label{diffusion_vorticity} 
   \int_{\Omega} \frac{D \omega}{Dt} \delta d\Omega 
 - \frac{1}{\textit{Re}} \int_{\Omega} \nabla^2 w \delta d\Omega
 = 0
\end{equation}

\begin{equation} \label{diffusion_stream}
   \int_{\Omega} \nabla^2 \psi \phi d\Omega
 + \int_{\Omega} w \phi d\Omega = 0
\end{equation}

\begin{equation} \label{diffusion_concentration} 
   \int_{\Omega} \frac{D e}{Dt} \eta d\Omega
 - \frac{1}{\textit{ReSc}} \int_{\Omega} \nabla^2 e \eta d\Omega 
 = 0
\end{equation}

\medskip
In the diffusive term of the equations (\ref{diffusion_vorticity}, \ref{diffusion_stream} and \ref{diffusion_concentration}),
we will apply Green's theorem in order to decrease 
the derivative order and separate the term evaluated in the boundary. 
Thus the diffusive term will become:

\begin{equation} \label{diffusion2_vorticity} 
 - \frac{1}{\textit{Re}} \int_{\Omega} \nabla^2 w \delta d\Omega
 = \frac{1}{\textit{Re}} \int_{\Omega} \nabla w \cdot \nabla \delta d\Omega
 - \frac{1}{\textit{Re}} \int_{\Gamma} \delta \nabla w \cdot \textbf{n} d\Gamma
\end{equation}

\begin{equation} \label{diffusion2_stream} 
 \int_{\Omega} \nabla^2 \psi \phi d\Omega
 = - \int_{\Omega} \nabla \psi \cdot \nabla \phi d\Omega
   + \int_{\Gamma} \phi \nabla \psi \cdot \textbf{n} d\Gamma
\end{equation}

\begin{equation} \label{diffusion2_concentration} 
 - \frac{1}{\textit{ReSc}} \int_{\Omega} \nabla^2 e \eta d\Omega
 = \frac{1}{\textit{ReSc}} \int_{\Omega} \nabla e \cdot \nabla \eta d\Omega
 - \frac{1}{\textit{ReSc}} \int_{\Gamma} \eta \nabla e \cdot \textbf{n} d\Gamma
\end{equation}

\noindent
where \textbf{n} is the normal vector oriented outside the $\Gamma$. 
The last term in the above equations is known as a natural condition. 
As previously mentioned, 
the problem proposed in this work has only Dirichlet conditions 
(known as an essential condition).
Therefore,
we will consider the assumptions
$\delta = 0$, $\phi = 0$ and $\eta = 0$ in the equations (\ref{diffusion2_vorticity}, \ref{diffusion2_stream} and \ref{diffusion2_concentration})
for all $\Gamma$. 
Thus, 
the integral in $\Gamma$ will be null
and the diffusive term of the equations 
(\ref{diffusion_vorticity}, \ref{diffusion_stream} and \ref{diffusion_concentration}) become:

\begin{equation}
 - \frac{1}{\textit{Re}} \int_{\Omega} \nabla^2 w \delta d\Omega
 = \frac{1}{\textit{Re}} \int_{\Omega} \nabla w \cdot \nabla \delta d\Omega
\end{equation}

\begin{equation} 
 \int_{\Omega} \nabla^2 \psi \phi d\Omega
 = - \int_{\Omega} \nabla \psi \cdot \nabla \phi d\Omega
\end{equation}

\begin{equation} 
 - \frac{1}{\textit{ReSc}} \int_{\Omega} \nabla^2 e \eta d\Omega
 = \frac{1}{\textit{ReSc}} \int_{\Omega} \nabla e \cdot \nabla \eta d\Omega
\end{equation}

\medskip
\noindent
Therefore, replacing the new diffusive terms in the governing equations:

\begin{equation} \label{vorticity weak} 
   \int_{\Omega} \frac{D \omega}{Dt} \delta d\Omega 
 + \frac{1}{\textit{Re}} \int_{\Omega} \nabla w \cdot \nabla \delta d\Omega 
 = 0
\end{equation}

\begin{equation} \label{stream weak}
 - \int_{\Omega} \nabla \psi \cdot \nabla \phi d\Omega
 + \int_{\Omega} w \phi d\Omega = 0
\end{equation}

\begin{equation} \label{concentration weak} 
   \int_{\Omega} \frac{D e}{Dt} \eta d\Omega
 + \frac{1}{\textit{ReSc}} \int_{\Omega} \nabla e \cdot \nabla \eta d\Omega 
 = 0
\end{equation}


\medskip
\noindent
If we consider that:

\begin{equation}
 \begin{aligned}
  \textbf{$m_1$}(\frac{D \omega}{Dt},\delta) & = \int_{\Omega} \frac{D \omega}{Dt} \delta d\Omega \\ 
  \textbf{$m_2$}(\psi,\phi) & = \int_{\Omega} w \phi d\Omega \\
  \textbf{$m_3$}(\frac{De}{Dt},\eta) & = \int_{\Omega} \frac{De}{Dt} \eta d\Omega \\
 \end{aligned}
 \qquad
 \begin{aligned}  
  \textbf{$k_1$}(w,\delta) & = \int_{\Omega} \nabla w \cdot \nabla \delta d\Omega \\
  \textbf{$k_2$}(\psi,\phi) & = \int_{\Omega} \nabla \psi \cdot \nabla \phi d\Omega \\
  \textbf{$k_3$}(e,\eta) & = \int_{\Omega} \nabla e \cdot \nabla \eta d\Omega \\
 \end{aligned}
\end{equation}

\noindent
The equations, thus, can be shown in the weak form respectively:

\begin{align}
 \textbf{$m_1$}(\frac{D \omega}{Dt},\delta) 
 + \frac{1}{\textit{Re}}\textbf{$k_1$}(w,\delta) 
 & = 0 \\
 - \textbf{$k_2$}(\psi,\phi) 
 + \textbf{$m_2$}(\psi,\phi) 
 & = 0 \\
 \textbf{$m_3$}(\frac{De}{De},\eta) 
 + \frac{1}{\textit{ReSc}}\textbf{$k_3$}(e,\eta) 
 & = 0
\end{align}


\noindent
Assuming that the sets of basis functions are:

\begin{equation}
 \begin{aligned}
  \mathbb{W} &= \{w \in \Omega \rightarrow \mathbb{R}^2
  : \int_\Omega w^2 d\Omega < \infty 
  ; w = w_\Gamma\} \\
  \mathbb{P} &= \{\psi \in \Omega \rightarrow \mathbb{R}^2
  : \int_\Omega \psi^2 d\Omega < \infty 
  ; \psi = \psi_\Gamma\} \\
  \mathbb{E} &= \{e \in \Omega \rightarrow \mathbb{R}^2
  : \int_\Omega e^2 d\Omega < \infty 
  ; e = e_\Gamma\}
 \end{aligned}
\end{equation}

\noindent
and the set of weight functions space:

\begin{equation}
 \begin{aligned}
  \mathbb{D} &= \{\delta \in \Omega \rightarrow \mathbb{R}^2
  : \int_\Omega \delta^2 d\Omega < \infty
  ; \delta_\Gamma = 0\} \\
  \mathbb{F} &= \{\phi \in \Omega \rightarrow \mathbb{R}^2
  : \int_\Omega \phi^2 d\Omega < \infty 
  ; \phi_\Gamma = 0\} \\
  \mathbb{N} &= \{\eta \in \Omega \rightarrow \mathbb{R}^2
  : \int_\Omega \eta^2 d\Omega < \infty 
  ; \eta_\Gamma = 0\}
 \end{aligned}
\end{equation}

\noindent
The weak formulation consists 
to find the solutions of 
$w \in \mathbb{W}$,
$\psi \in \mathbb{P}$, and $e \in \mathbb{E}$
such that:

\begin{align}
 \textbf{$m_1$}(\frac{D \omega}{Dt},\delta) 
 + \frac{1}{\textit{Re}}\textbf{$k_1$}(w,\delta) 
 & = 0 \\
 - \textbf{$k_2$}(\psi,\phi) 
 + \textbf{$m_2$}(\psi,\phi) 
 & = 0 \\
 \textbf{$m_3$}(\frac{De}{De},\eta) 
 + \frac{1}{\textit{ReSc}}\textbf{$k_3$}(e,\eta) 
 & = 0
\end{align}

\noindent
for all $\delta \in \mathbb{D}$, 
$\phi \in \mathbb{F}$ and 
$\eta \in \mathbb{N}$.

