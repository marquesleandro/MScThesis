In numerical simulations, the choice of initial and boundary conditions
is important to ensure result accuracy for any modeled problem 
by differential equations. 
The boundary conditions used
are briefly explained below, 
followed by their detailed specifications 
in each particular case in the validations and results sections:

\begin{itemize}
 \item \textit{inflow condition}:
 this is specified when an mass inflow is desired.
 For such a condition, $u = u_{0}$
 and $v = v_{0}$.

 \item \textit{wall condition}:
 this is specified at wall boundaries (moving wall
 and noslip conditions).
 All the velocity components are specified with 
 the same wall velocity values.

 \item \textit{outflow condition}: 
 this condition represents a state where is close to a
 fully developed profile.
 Usually no value is specified for the unknowns.
\end{itemize}

As mentioned by Batchelor (1964) \textbf{reference},
the $\psi$ is constant along a streamline, then
the streamfunction boundary condition can be calculated by
$\psi_{2} - \psi_{1} = \int \left(udy - vdx\right)$,
where can be used $u$ and $v$ velocity inflow component.
The $\psi_{1}$ and $\psi_{2}$ are usually called 
bottom and top streamlines, because the difference
between two $\psi$ values is equal to volume flow
rate across inflow boundary. In this work, was set
null value for bottom streamline. For top streamline,
was calculated using $u_{0}$ and $v_{0}$ inflow velocity
components, that is, $\psi_{2} = \int \left(u_{0}dy - v_{0}dx\right)$.

\medskip
One of the main difficulties for
Streamfunction-Vorticity Formulation is due to absence of vorticity
boundary condition, as shown in \textbf{references}.
In this work, the vorticity boundary contition
was calculated by $\omega =$ \textbf{curl\ v}
at each time step.
