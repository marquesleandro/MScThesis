O princípio de conservação de espécie química estabelece que:

\medskip
\begin{center}
\textarray{A taxa de acumulação\\
           da quantidade de\\ 
           espécie química\\ 
           dentro do volume\\
           de controle}
           $= -$ 
\textarray{O fluxo líquido\\
           da quantidade\\ 
           de espécie\\
           química que\\
           cruza a fronteira}
           $+$ 
\textarray{A resultante da \\
           taxa de geração \\ 
           de espécie quí- \\
           mica no volume}
\end{center}

\medskip
Matematicamente, a taxa de acumulação da quantidade de espécie química
dentro do volume de controle pode ser representada como:

\begin{equation} \label{esp 1} 
 \int_{V} \frac{\partial c}{\partial t} dV
\end{equation}

\medskip
O fluxo líquido da quantidade de espécie
química que cruza a fronteira do volume pode ser
decomposto na soma de dois fluxos:

\begin{equation}  
 \oint_{S} c \textbf{v} \cdot \textbf{n} dA
 -
 \oint_{S} D \nabla c \cdot \textbf{n} dA
\end{equation}

\medskip
\noindent
onde $D$ é o coeficiente de difusão de espécie química.
A resultante da taxa de geração de espécie química
no volume pode ser representada por:

\begin{equation} 
 \int_{V} \overset{.}{R} dV
\end{equation}

\medskip
\noindent
Dessa forma, segundo o princípio de conservação de espécie
química:

\begin{equation}
 \int_{V} \frac{\partial c}{\partial t} dV
 = 
 - 
 \oint_{S} c \textbf{v} \cdot \textbf{n} dA
 +
 \oint_{S} D \nabla c \cdot \textbf{n} dA
 +
 \int_{V} \overset{.}{R} dV
\end{equation}

\medskip
\noindent
Aplicando o \textit{Teorema de Gauss} nas integrais
de área:

\begin{equation}
 \int_{V} \frac{\partial c}{\partial t} dV
 = 
 - 
 \int_{V} \nabla \cdot \big( c \textbf{v} \big) dV
 +
 \int_{V} \nabla \cdot \big( D \nabla c \big) dV
 +
 \int_{V} \overset{.}{R} dV
\end{equation}

\medskip
\noindent
isto é:

\begin{equation} \label{esp 2}
 \int_{V} \Bigg[ \frac{\partial c}{\partial t}
 + 
 \nabla \cdot \big( c \textbf{v} \big)
 -
 \nabla \cdot \big( D \nabla c \big)
 -
 \overset{.}{R} \Bigg] dV = 0
\end{equation}



\medskip
\noindent
Considerando o fato de $dV \neq 0$,
a Eq. \ref{esp 2} pode ser expressa como:

\begin{equation}
 \frac{\partial c}{\partial t}
 + 
 \nabla \cdot \big( c \textbf{v} \big)
 -
 \nabla \cdot \big( D \nabla c \big)
 -
 \overset{.}{R} = 0
\end{equation}



\medskip
\noindent
isto é:

\begin{equation}
 \frac{\partial c}{\partial t}
 +
 \nabla \cdot \big( c \textbf{v} \big)
 =
 \nabla \cdot \big( D \nabla c \big)
 +
 \overset{.}{R}
\end{equation}

\medskip
\noindent
Desenvolvendo o lado esquerdo da equação, temos:

\begin{equation}
 \frac{\partial c}{\partial t}
 +
 \textbf{v} \cdot \nabla c
 + 
 c \nabla \cdot \textbf{v}
 =
 \nabla \cdot \big( D \nabla c \big)
 +
 \overset{.}{R}
\end{equation}

\medskip
Vimos que o último termo do lado esquerdo
é nulo devido a hipótese de incompressibilidade
do fluido (Eq. \ref{incompressible continuity equation}),
logo:

\begin{equation}
 \frac{\partial c}{\partial t}
 +
 \textbf{v} \cdot \nabla c
 =
 \nabla \cdot \big( D \nabla c \big)
 +
 \overset{.}{R}
\end{equation}

\newpage
Considerando que o coeficiente de 
difusão é constante e sem geração de
espécie química, 
a equação de conservação de espécie química
pode ser reescrita como:

\begin{equation} \label{chemical species}
 \frac{\partial c}{\partial t}
 +
 \textbf{v} \cdot \nabla c
 =
 D \nabla^{2} c
\end{equation}

\medskip
\noindent
A Eq. \ref{chemical species} é conhecida como 
\textit{Equação de Transporte de Espécie Química}
para um fluido incompressível, com coeficiente
de difusão constante e sem geração de espécie
química.
