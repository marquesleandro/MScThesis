The species transport conservation establish that:

\medskip
\begin{center}
\textarray{Chemical species\\
           accumulation rate\\ 
           on control volume}
           $= -$ 
\textarray{Chemical species\\
           flux crossing\\ 
           the boundary}
           $+$ 
\textarray{The resultant of \\
           chemical species\\ 
           source rate }
\end{center}

\medskip
Mathematically, chemical species accumulation on control volume
can be represented by:

\begin{equation} \label{esp 1} 
 \int_{V} \frac{\partial e}{\partial t} dV
\end{equation}

\medskip
The chemical species flux crossing the boundary
can be splitted into two flux:

\begin{equation}  
 \oint_{S} e \textbf{c} \cdot \textbf{n} dA
 -
 \oint_{S} D \nabla e \cdot \textbf{n} dA
\end{equation}

\medskip
\noindent
where $D$ is coefficient of chemical species diffusion. 
The resultant of chemical species source rate
on volume can be represented by:

\begin{equation} 
 \int_{V} \overset{.}{R} dV
\end{equation}

\medskip
\noindent
Thereby, acccording to species transport conservation:

\begin{equation}
 \int_{V} \frac{\partial e}{\partial t} dV
 = 
 - 
 \oint_{S} e \textbf{c} \cdot \textbf{n} dA
 +
 \oint_{S} D \nabla e \cdot \textbf{n} dA
 +
 \int_{V} \overset{.}{R} dV
\end{equation}

\medskip
\noindent
Applying the \textit{Gauss Theorem} on surface integrals:

\begin{equation}
 \int_{V} \frac{\partial e}{\partial t} dV
 = 
 - 
 \int_{V} \nabla \cdot \big( e \textbf{c} \big) dV
 +
 \int_{V} \nabla \cdot \big( D \nabla e \big) dV
 +
 \int_{V} \overset{.}{R} dV
\end{equation}

\medskip
\noindent
that is:

\begin{equation} \label{esp 2}
 \int_{V} \Bigg[ \frac{\partial e}{\partial t}
 + 
 \nabla \cdot \big( e \textbf{c} \big)
 -
 \nabla \cdot \big( D \nabla e \big)
 -
 \overset{.}{R} \Bigg] dV = 0
\end{equation}



\medskip
\noindent
Whereas the $dV \neq 0$,
the Eq. \ref{esp 2} can be represented by:

\begin{equation}
 \frac{\partial e}{\partial t}
 + 
 \nabla \cdot \big( e \textbf{c} \big)
 -
 \nabla \cdot \big( D \nabla e \big)
 -
 \overset{.}{R} = 0
\end{equation}



\medskip
\noindent
that is:

\begin{equation}
 \frac{\partial e}{\partial t}
 +
 \nabla \cdot \big( e \textbf{c} \big)
 =
 \nabla \cdot \big( D \nabla e \big)
 +
 \overset{.}{R}
\end{equation}

\medskip
\noindent
Developing the left hand side of equation, we have:

\begin{equation}
 \frac{\partial e}{\partial t}
 +
 \textbf{v} \cdot \nabla e
 + 
 e \nabla \cdot \textbf{c}
 =
 \nabla \cdot \big( D \nabla e \big)
 +
 \overset{.}{R}
\end{equation}

\medskip
The last term of above equation is null because the
fluid incompressibility assumption
(Eq. \ref{incompressible continuity equation}),
thus:

\begin{equation}
 \frac{\partial e}{\partial t}
 +
 \textbf{c} \cdot \nabla e
 =
 \nabla \cdot \big( D \nabla e \big)
 +
 \overset{.}{R}
\end{equation}

\newpage
Considering the diffusion coefficient
is constant and without chemical species
generation,
the species transport equation can rewritten as:

\begin{equation} \label{chemical species}
 \frac{\partial e}{\partial t}
 +
 \textbf{c} \cdot \nabla e
 =
 D \nabla^{2} e
\end{equation}

\medskip
\noindent
The Eq. \ref{chemical species} is known as
\textit{Species Transport Equation}
for an incompressible fluid, with constant diffusion coefficient
and without chemical species generation for the
Arbitrary Lagrangia-Eulerian description.
