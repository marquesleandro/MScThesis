The choice of the kinematical description of the continuum is 
extremely important for the development of a computational code, 
since it directly affects the accuracy of the numerical result. 
In the literature, two classic descriptions are commonly used, 
namely Lagrangian and Eulerian.

\medskip
The Lagrangian description is one where each computational mesh node 
moves at the same velocity as the material point, as can be seen in
 Fig. (3.1a). Thus, for each time step, we have a new computational mesh. 
The main advantage of this description is that the value of the 
computational mesh node will have the same value as the material point
 and thus, numerical diffusion will not be observed. In addition, 
the Lagrangian description makes it possible to perform 
fluid-structure interaction simulations. However, for large 
deformations, it is necessary to implement an insetion and deletion
node algorithm for computational mesh.

\medskip
The Eulerian description is the one where the computational mesh node 
remains fixed for each time step, as can be seen in Fig. (3.1b).
Thus, the computational mesh node value will be an interpolation 
of the material point, causing the presence of numerical diffusion 
in the solution. However, the computational cost is relatively 
attractive since it is not necessary the remeshing in each time step.

\medskip
A description, however, that combines the advantages of these two 
classic descriptions as well as minimizing their disadvantages would be
most appropriate. It is in this context that the Arbitrary 
Lagrangian-Eulerian description (ALE) was developed. 
This description considers that the velocity field of computational 
mesh is unlike than the material point and null value, as can be seen 
in Fig. (3.1c). In this way, it can be calculated as a 
linear combination of other velocity fields, so that we have an 
optimal relationship between numerical diffusion and mesh deformation. 
In addition, it is possible to assign several velocity values to 
specific regions of the problem in order to improve the 
accuracy of the solution.



\begin{figure}[H]
\begin{center}
\begin{tikzpicture}[scale=1.5]

 % (a) Lagrangian
 % -------------------------------------------------------------------
 % bottom line
 \draw (0,5) -- (6,5);

 \node[square, fill=white, draw, inner sep=0pt, minimum size=8pt] at (0,5) {};
 \node[square, fill=white, draw, inner sep=0pt, minimum size=8pt] at (1,5) {};
 \node[square, fill=white, draw, inner sep=0pt, minimum size=8pt] at (2.7,5) {};
 \node[square, fill=white, draw, inner sep=0pt, minimum size=8pt] at (4,5) {};
 \node[square, fill=white, draw, inner sep=0pt, minimum size=8pt] at (5.1,5) {};
 \node[square, fill=white, draw, inner sep=0pt, minimum size=8pt] at (6,5) {};

 \node[circle, fill=black, inner sep=0pt, minimum size=4pt] at (0,5) {};
 \node[circle, fill=black, inner sep=0pt, minimum size=4pt] at (1,5) {};
 \node[circle, fill=black, inner sep=0pt, minimum size=4pt] at (2.7,5) {};
 \node[circle, fill=black, inner sep=0pt, minimum size=4pt] at (4,5) {};
 \node[circle, fill=black, inner sep=0pt, minimum size=4pt] at (5.1,5) {};
 \node[circle, fill=black, inner sep=0pt, minimum size=4pt] at (6.0,5) {};

 % top line
 \draw (0.5,6) -- (5.4,6);

 \node[square, fill=white, draw, inner sep=0pt, minimum size=8pt] at (0.5,6) {};
 \node[square, fill=white, draw, inner sep=0pt, minimum size=8pt] at (1.4,6) {};
 \node[square, fill=white, draw, inner sep=0pt, minimum size=8pt] at (2.3,6) {};
 \node[square, fill=white, draw, inner sep=0pt, minimum size=8pt] at (3.5,6) {};
 \node[square, fill=white, draw, inner sep=0pt, minimum size=8pt] at (4.6,6) {};
 \node[square, fill=white, draw, inner sep=0pt, minimum size=8pt] at (5.4,6) {};

 \node[circle, fill=black, inner sep=0pt, minimum size=4pt] at (0.5,6) {};
 \node[circle, fill=black, inner sep=0pt, minimum size=4pt] at (1.4,6) {};
 \node[circle, fill=black, inner sep=0pt, minimum size=4pt] at (2.3,6){};
 \node[circle, fill=black, inner sep=0pt, minimum size=4pt] at (3.5,6) {};
 \node[circle, fill=black, inner sep=0pt, minimum size=4pt] at (4.6,6) {};
 \node[circle, fill=black, inner sep=0pt, minimum size=4pt] at (5.4,6) {};

 % mesh motion
 \draw [dashed] (0,  5) -- (0.5,6)  ;
 \draw [dashed] (1,  5) -- (1.4,6)  ;
 \draw [dashed] (2.7,5) -- (2.3,6);
 \draw [dashed] (4,  5) -- (3.5,6)  ;
 \draw [dashed] (5.1,5) -- (4.6,6);
 \draw [dashed] (6,  5) -- (5.4,6)  ;

 % particle motion
 \draw [dotted] (0,  5.15) -- (0.5,5.85);
 \draw [dotted] (1,  5.15) -- (1.4,5.85);
 \draw [dotted] (2.7,5.15) -- (2.3,5.85);
 \draw [dotted] (4,  5.15) -- (3.5,5.85);
 \draw [dotted] (5.1,5.15) -- (4.6,5.85);
 \draw [dotted] (6,  5.15) -- (5.4,5.85):



 % Eulerian legend 
 \draw [latexnew-latex] (-0.3,5) -- (-0.3,6);
 \node at (-0.5,5.5) {t};
 \node at (3,6.5) {\scriptsize Lagrangian description};
 % -------------------------------------------------------------------
 


 % (b) Eulerian
 % -------------------------------------------------------------------
 % bottom line
 \draw (0,2.5) -- (6,2.5);

 \node[square, fill=white, draw, inner sep=0pt, minimum size=8pt] at (0,2.5) {};
 \node[square, fill=white, draw, inner sep=0pt, minimum size=8pt] at (1,2.5) {};
 \node[square, fill=white, draw, inner sep=0pt, minimum size=8pt] at (2.7,2.5) {};
 \node[square, fill=white, draw, inner sep=0pt, minimum size=8pt] at (4,2.5) {};
 \node[square, fill=white, draw, inner sep=0pt, minimum size=8pt] at (5.1,2.5) {};
 \node[square, fill=white, draw, inner sep=0pt, minimum size=8pt] at (6,2.5) {};

 \node[circle, fill=black, inner sep=0pt, minimum size=4pt] at (0,  2.5) {};
 \node[circle, fill=black, inner sep=0pt, minimum size=4pt] at (1,  2.5) {};
 \node[circle, fill=black, inner sep=0pt, minimum size=4pt] at (2.7,2.5) {};
 \node[circle, fill=black, inner sep=0pt, minimum size=4pt] at (4,  2.5) {};
 \node[circle, fill=black, inner sep=0pt, minimum size=4pt] at (5.1,2.5) {};
 \node[circle, fill=black, inner sep=0pt, minimum size=4pt] at (6.0,2.5) {};

 % top line
 \draw (0,3.5) -- (6,3.5);

 \node[square, fill=white, draw, inner sep=0pt, minimum size=8pt] at (0.5,3.5) {};
 \node[square, fill=white, draw, inner sep=0pt, minimum size=8pt] at (1.4,3.5) {};
 \node[square, fill=white, draw, inner sep=0pt, minimum size=8pt] at (2.3,3.5) {};
 \node[square, fill=white, draw, inner sep=0pt, minimum size=8pt] at (3.5,3.5) {};
 \node[square, fill=white, draw, inner sep=0pt, minimum size=8pt] at (4.6,3.5) {};
 \node[square, fill=white, draw, inner sep=0pt, minimum size=8pt] at (5.4,3.5) {};

 \node[circle, fill=black, inner sep=0pt, minimum size=4pt] at (0,  3.5) {};
 \node[circle, fill=black, inner sep=0pt, minimum size=4pt] at (1,  3.5) {};
 \node[circle, fill=black, inner sep=0pt, minimum size=4pt] at (2.7,3.5){};
 \node[circle, fill=black, inner sep=0pt, minimum size=4pt] at (4,  3.5) {};
 \node[circle, fill=black, inner sep=0pt, minimum size=4pt] at (5.1,3.5) {};
 \node[circle, fill=black, inner sep=0pt, minimum size=4pt] at (6.0,3.5) {};

 % mesh motion
 \draw [dashed] (0,  2.5) -- (0,  3.5)  ;
 \draw [dashed] (1,  2.5) -- (1,  3.5)  ;
 \draw [dashed] (2.7,2.5) -- (2.7,3.5)  ;
 \draw [dashed] (4,  2.5) -- (4,  3.5)  ;
 \draw [dashed] (5.1,2.5) -- (5.1,3.5)  ;
 \draw [dashed] (6,  2.5) -- (6,  3.5)  ;

 % particle motion
 \draw [dotted] (0,  2.65) -- (0.5,3.35);
 \draw [dotted] (1,  2.65) -- (1.4,3.35);
 \draw [dotted] (2.7,2.65) -- (2.3,3.35);
 \draw [dotted] (4,  2.65) -- (3.5,3.35);
 \draw [dotted] (5.1,2.65) -- (4.6,3.35);
 \draw [dotted] (6,  2.65) -- (5.4,3.35):



 % Eulerian legend 
 \draw [latexnew-latex] (-0.3,2.5) -- (-0.3,3.5);
 \node at (-0.5,3) {t};
 \node at (3,4.0) {\scriptsize Eulerian description};
 % -------------------------------------------------------------------
 


 % (c) ALE
 % -------------------------------------------------------------------
 % bottom line
 \draw (0,0) -- (6,0);

 \node[square, fill=white, draw, inner sep=0pt, minimum size=8pt] at (0,0) {};
 \node[square, fill=white, draw, inner sep=0pt, minimum size=8pt] at (1,0) {};
 \node[square, fill=white, draw, inner sep=0pt, minimum size=8pt] at (2.7,0) {};
 \node[square, fill=white, draw, inner sep=0pt, minimum size=8pt] at (4,0) {};
 \node[square, fill=white, draw, inner sep=0pt, minimum size=8pt] at (5.1,0) {};
 \node[square, fill=white, draw, inner sep=0pt, minimum size=8pt] at (6,0) {};

 \node[circle, fill=black, inner sep=0pt, minimum size=4pt] at (0,0) {};
 \node[circle, fill=black, inner sep=0pt, minimum size=4pt] at (1,0) {};
 \node[circle, fill=black, inner sep=0pt, minimum size=4pt] at (2.7,0) {};
 \node[circle, fill=black, inner sep=0pt, minimum size=4pt] at (4,0) {};
 \node[circle, fill=black, inner sep=0pt, minimum size=4pt] at (5.1,0) {};
 \node[circle, fill=black, inner sep=0pt, minimum size=4pt] at (6.0,0) {};

 % top line
 \draw (0.2,1) -- (5.8,1);

 \node[square, fill=white, draw, inner sep=0pt, minimum size=8pt] at (0.5,1) {};
 \node[square, fill=white, draw, inner sep=0pt, minimum size=8pt] at (1.4,1) {};
 \node[square, fill=white, draw, inner sep=0pt, minimum size=8pt] at (2.3,1) {};
 \node[square, fill=white, draw, inner sep=0pt, minimum size=8pt] at (3.5,1) {};
 \node[square, fill=white, draw, inner sep=0pt, minimum size=8pt] at (4.6,1) {};
 \node[square, fill=white, draw, inner sep=0pt, minimum size=8pt] at (5.4,1) {};

 \node[circle, fill=black, inner sep=0pt, minimum size=4pt] at (0.2,1) {};
 \node[circle, fill=black, inner sep=0pt, minimum size=4pt] at (1.1,1) {};
 \node[circle, fill=black, inner sep=0pt, minimum size=4pt] at (2.8,1) {};
 \node[circle, fill=black, inner sep=0pt, minimum size=4pt] at (3.8,1) {};
 \node[circle, fill=black, inner sep=0pt, minimum size=4pt] at (4.9,1) {};
 \node[circle, fill=black, inner sep=0pt, minimum size=4pt] at (5.8,1) {};

 % mesh motion
 \draw [dashed] (0,0.0) -- (0.2,1);
 \draw [dashed] (1,0.0) -- (1.1,1);
 \draw [dashed] (2.7,0.0) -- (2.8,1);
 \draw [dashed] (4,0.0) -- (3.8,1);
 \draw [dashed] (5.1,0.0) -- (4.9,1);
 \draw [dashed] (6,0.0) -- (5.8,1);

 % particle motion
 \draw [dotted] (0,0.15) -- (0.5,0.85);
 \draw [dotted] (1,0.15) -- (1.4,0.85);
 \draw [dotted] (2.7,0.15) -- (2.3,0.85);
 \draw [dotted] (4,0.15) -- (3.5,0.85);
 \draw [dotted] (5.1,0.15) -- (4.6,0.85);
 \draw [dotted] (6,0.15) -- (5.4,0.85):



 % ALE legend 
 \draw [latexnew-latex] (-0.3,0) -- (-0.3,1);
 \node at (-0.5,0.5) {t};
 \node at (3,1.5) {\scriptsize ALE description};
 

 % picture legend
 \node[square, draw, inner sep=0pt, minimum size=8pt] at (0.5,-0.7);
 \node[circle, fill=black, inner sep=0pt, minimum size=4pt] at (0.5,-1.2);
 \draw [dotted] (3.5,-0.7) -- (4.08,-0.7);
 \draw [dashed] (3.5,-1.2) -- (4.1,-1.2);

 \node at (1.7,-0.7) {\tiny material point};
 \node at (1.2,-1.2) {\tiny node};
 \node at (5.2,-0.7) {\tiny particle motion};
 \node at (5.1,-1.2) {\tiny mesh motion};
 % -------------------------------------------------------------------


\end{tikzpicture}
\end{center}
\caption{One-dimensional examples of the (a) Lagrangian description, (b) Eulerian description and (c) ALE description.}
\end{figure}



\medskip
The ALE description was first implemented in the finite difference
 method, as observed in the Hirt et al. (1974) \cite{hirt1974} 
and was subsequently adopted in the finite elements context, 
as presented by Donea (1982) \cite{donea1982}. In this description, 
the referential domain that describes the computational mesh moving 
is different from the material domain and the spatial domain, 
as shown in figure 1. However, it is possible to correlate these 
frameworks. For instance, if the operator Z is equal to the 
identity matrix (\textbf{I}), then the referential and material domain 
is the same and, subsequently, the node velocity of the 
computational mesh 
is equivalent to the material points velocity (Lagrangian description). 
But, if the operator Y is equal to \textbf{I}, the computational mesh
velocity is equivalent to null value and then the Eulerian description is
obtained. For more details, it is possible to consult the works 
of Donea (2004) \cite{donea2004} and Hughes (1981) \cite{hughes1981}.

%\medskip
%Therefore, the material point velocity that travels from 
%the material domain to the spatial domain may be obtained 
%according to the X operator:
%
%equation 4 (donea2004)
%
%\medskip
%\noindent
%Whereas, the computational mesh velocity that travels from the 
%referential domain to the spatial domain may be obtained according 
%to the Y operator:
%
%equation 7 (donea2004)
%
%\medskip
%\noindent
%Finally, we can describe the transport of any property f in 
%the spatial domain according to the ALE description as:
%
%equation 3.3 (phd2012)
