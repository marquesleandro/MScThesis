In this section, the non-dimensional form
of continuity, Navier-Stokes and
species transport equations are shown.
The non-dimensionalization helps to understand 
which terms of the equation influence most during 
a given simulation in addition to allowing experiments 
with small scale models 
The following parameters was used in non-dimensionalization:

\begin{equation}
 \begin{aligned}
  p & = \rho_{0} U^{2} p^{*} \\[10pt]
 \textbf{v} & = U \textbf{v}^{*} \\
 \end{aligned}
 \qquad
 \begin{aligned}
 c & = ( c_{s} - c_{0} ) c^{*} + c_{0}\\[10pt]
 \textbf{g} & = g_{0} \textbf{g}^{*} \\
 \end{aligned}
 \qquad
 \begin{aligned}
 \nu & = \nu_{0} \nu^{*} \\[10pt]
 \rho & = \rho_{0} \rho^{*} \\
 \end{aligned}
 \qquad
 \begin{aligned}
 D & = D_{0} D^{*} \\[5pt]
 \nabla & = \frac{1}{L} \nabla^{*} \\
 \end{aligned}
 \qquad
 \begin{aligned}
 x & = L x^{*} \\[5pt]
 t & = \frac{L}{U} t^{*} \\
 \end{aligned}
 \nonumber
\end{equation}


\medskip
\noindent
where the asterisk identify the non-dimensional unknowns.
Replacing the above parameters in Eq. \ref{incompressible
continuity equation}, we have:

\begin{equation}
 \frac{U}{L} \nabla^{*} \cdot \textbf{v}^{*} = 0
\end{equation}

\medskip
\noindent
Multiplying both sides by $U/L$:

\begin{equation} \label{continuidade adimensional 1}
 \nabla^{*} \cdot \textbf{v}^{*} = 0
\end{equation}

\medskip
\noindent
A similar procedure is performed in
Eq. \ref{navier-stokes}, that is:

\begin{equation}
 \frac{U^{2}}{L} \frac{\partial \textbf{v$^{*}$}}{\partial t^{*}} 
 + 
 \frac{U^{2}}{L} \textbf{v$^{*}$} \cdot \nabla^{*} \textbf{v$^{*}$}
 =
 -
 \frac{U^{2}}{L} \frac{1}{\rho^{*}} \nabla^{*} p^{*}
 +
 \frac{\nu_{0} U}{L^{2}} \nu^{*} \nabla^{*2} \textbf{v$^{*}$}
 +
 g_{0} \textbf{g$^{*}$}
\end{equation}

\medskip
\noindent
Multiplying both sides by $L/U^{2}$:

\begin{equation} \label{navier-stokes adimensional 2}
 \frac{\partial \textbf{v$^{*}$}}{\partial t^{*}} 
 + 
 \textbf{v$^{*}$} \cdot \nabla^{*} \textbf{v$^{*}$}
 =
 -
 \frac{1}{\rho^{*}} \nabla^{*} p^{*}
 +
 \frac{\nu_{0}}{UL} \nu^{*} \nabla^{*2} \textbf{v$^{*}$}
 +
 \frac{g_{0}L}{U^{2}} \textbf{g$^{*}$}
\end{equation}

\medskip
\noindent
that is:

\begin{equation} \label{navier-stokes adimensional 1}
 \frac{\partial \textbf{v$^{*}$}}{\partial t^{*}} 
 + 
 \textbf{v$^{*}$} \cdot \nabla^{*} \textbf{v$^{*}$}
 =
 -
 \nabla^{*} p^{*}
 +
 \frac{\nu_{0}}{UL} \nabla^{*2} \textbf{v$^{*}$}
 +
 \frac{g_{0}L}{U^{2}} \textbf{g$^{*}$}
\end{equation}

\medskip
\noindent
In the Eq. \ref{chemical species}, a similar procedure is performed:

\begin{equation}
 (c_{s}-c_{0}) \frac{U}{L} \frac{\partial c^{*}}{\partial t^{*}}
 +
 (c_{s}-c_{0}) \frac{U}{L} \textbf{v$^{*}$} \cdot \nabla^{*} c^{*}
 =
 (c_{s}-c_{0}) \frac{D_{0}}{L^{2}} D^{*} \nabla^{*2} c^{*}
\end{equation}

\medskip
\noindent
Multiplying both sides by $L/U(c_{s}-c_{0})$, we have:

\begin{equation} \label{especie quimica adimensional 2}
 \frac{\partial c^{*}}{\partial t^{*}}
 +
 \textbf{v$^{*}$} \cdot \nabla^{*} c^{*}
 =
 \frac{D_{0}}{UL} D^{*} \nabla^{*2} c^{*}
\end{equation}

\medskip
\noindent
that is

\begin{equation} \label{especie quimica adimensional 1}
 \frac{\partial c^{*}}{\partial t^{*}}
 +
 \textbf{v$^{*}$} \cdot \nabla^{*} c^{*}
 =
 \frac{D_{0}}{UL} \nabla^{*2} c^{*}
\end{equation}

\newpage
Important non-dimensional groups are found in
Eqs. \ref{continuidade adimensional 1}, \ref{navier-stokes adimensional 1}
and \ref{especie quimica adimensional 1}. 
The description of theses groups are shown below:

\begin{itemize}
 \item \textbf{Reynolds Number} ($Re$): 
 Is ratio of inertial forces to viscous forces that 
 it acts a moving fluid particle.
 It is represented by:
 \begin{equation}
  Re = \frac{UL}{\nu_{0}}
 \end{equation}
 
 where, $\nu_{0}$, $U$ and $L$ are kinematic viscosity of fluid, 
 the velocity field
 and characteristic length, respectively.
 The Reynolds number is often used to define the
 flow, where $Re < Re_{critical}$ is
 a laminar flow and $Re > Re_{critical}$
 is a turbulent flow \cite{fox2012}.

 \item \textbf{Froude number} ($Fr$): 
 Is ratio of inertial force to gravitational forces
 Is represented by:
 \begin{equation}
  Fr = \frac{U}{\sqrt{g_{0}L}}
 \end{equation}
 
 where, $g_{0}$ is the gravitational acceleration.

 \item \textbf{Mass Péclet number} ($Pe_{m}$): 
 Is ratio advective transport to diffusion transport
 Is represented by:
 \begin{equation}
  Pe_{m} = \frac{D_{0}}{UL}
 \end{equation}
 
 where, $D_{0}$ is diffusion coefficient of chemical species.
 The $Pe_{m}$ number is often
 shown as the product of two other non-dimensional groups: 
 the \textit{Reynolds number} and the \textit{Schmidt number}.

 \newpage
 \item \textbf{Schmidt number} ($Sc$): 
 Is ratio hydrodynamic buondary layer thickness to
 chemical species diffusion.
 Is represented by:
 \begin{equation}
  Sc = \frac{\nu_{0}}{D_{0}}
 \end{equation}
\end{itemize}

\medskip
Replacing these non-dimensional groups in Eqs.
\ref{continuidade adimensional 1}, \ref{navier-stokes adimensional 1}
and \ref{especie quimica adimensional 1} and
removing the asterisk, we have:

\begin{equation} \label{continuidade adimensional}
 \nabla \cdot \textbf{v} = 0
\end{equation}

\begin{equation} \label{navier-stokes adimensional}
 \frac{\partial \textbf{v}}{\partial t} 
 + 
 \textbf{v} \cdot \nabla \textbf{v}
 =
 -
 \nabla p
 +
 \frac{1}{Re} \nabla^{2} \textbf{v}
 +
 \frac{1}{Fr^{2}} \textbf{g}
\end{equation}

\begin{equation} \label{especie quimica adimensional}
 \frac{\partial c}{\partial t}
 +
 \textbf{v} \cdot \nabla c
 =
 \frac{1}{ReSc} \nabla^{2} c
\end{equation}

\medskip
The Eqs.
\ref{continuidade adimensional}, 
\ref{navier-stokes adimensional} e 
\ref{especie quimica adimensional}
are, respectively, non-dimensional form of Continuity,
Navier-Stokes and Species Transport equations
for a newtonian and incompressble flow.

