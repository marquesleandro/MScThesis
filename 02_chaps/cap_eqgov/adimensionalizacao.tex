Nesta seção a forma adimensional das equações
da continuidade, da Navier-Stokes e do
transporte de espécie química é apresentada.
A adimensionalização ajuda a entender quais os
termos da equação influencia mais durante uma
determinada simulação além de possibilitar
experimentos com modelos em escala reduzida. 
Os seguintes parâmetros
foram usados na adimensionalização:

\begin{equation}
 \begin{aligned}
  p & = \rho_{0} U^{2} p^{*} \\[10pt]
 \textbf{v} & = U \textbf{v}^{*} \\
 \end{aligned}
 \qquad
 \begin{aligned}
 c & = ( c_{s} - c_{0} ) c^{*} + c_{0}\\[10pt]
 \textbf{g} & = g_{0} \textbf{g}^{*} \\
 \end{aligned}
 \qquad
 \begin{aligned}
 \nu & = \nu_{0} \nu^{*} \\[10pt]
 \rho & = \rho_{0} \rho^{*} \\
 \end{aligned}
 \qquad
 \begin{aligned}
 D & = D_{0} D^{*} \\[5pt]
 \nabla & = \frac{1}{L} \nabla^{*} \\
 \end{aligned}
 \qquad
 \begin{aligned}
 x & = L x^{*} \\[5pt]
 t & = \frac{L}{U} t^{*} \\
 \end{aligned}
 \nonumber
\end{equation}


\medskip
\noindent
onde o asterisco identifica as variáveis adimensionais.
Substituindo os parâmetros acima na Eq. \ref{incompressible
continuity equation}, temos:

\begin{equation}
 \frac{U}{L} \nabla^{*} \cdot \textbf{v}^{*} = 0
\end{equation}

\medskip
\noindent
Multiplicando ambos os lados por $U/L$:

\begin{equation} \label{continuidade adimensional 1}
 \nabla^{*} \cdot \textbf{v}^{*} = 0
\end{equation}

\medskip
\noindent
Um procedimento semelhante
é realizado na Eq. \ref{navier-stokes}, isto é:

\begin{equation}
 \frac{U^{2}}{L} \frac{\partial \textbf{v$^{*}$}}{\partial t^{*}} 
 + 
 \frac{U^{2}}{L} \textbf{v$^{*}$} \cdot \nabla^{*} \textbf{v$^{*}$}
 =
 -
 \frac{U^{2}}{L} \frac{1}{\rho^{*}} \nabla^{*} p^{*}
 +
 \frac{\nu_{0} U}{L^{2}} \nu^{*} \nabla^{*2} \textbf{v$^{*}$}
 +
 g_{0} \textbf{g$^{*}$}
\end{equation}

\medskip
\noindent
Multiplicando ambos os lados por $L/U^{2}$:

\begin{equation} \label{navier-stokes adimensional 2}
 \frac{\partial \textbf{v$^{*}$}}{\partial t^{*}} 
 + 
 \textbf{v$^{*}$} \cdot \nabla^{*} \textbf{v$^{*}$}
 =
 -
 \frac{1}{\rho^{*}} \nabla^{*} p^{*}
 +
 \frac{\nu_{0}}{UL} \nu^{*} \nabla^{*2} \textbf{v$^{*}$}
 +
 \frac{g_{0}L}{U^{2}} \textbf{g$^{*}$}
\end{equation}

\medskip
\noindent
isto é:

\begin{equation} \label{navier-stokes adimensional 1}
 \frac{\partial \textbf{v$^{*}$}}{\partial t^{*}} 
 + 
 \textbf{v$^{*}$} \cdot \nabla^{*} \textbf{v$^{*}$}
 =
 -
 \nabla^{*} p^{*}
 +
 \frac{\nu_{0}}{UL} \nabla^{*2} \textbf{v$^{*}$}
 +
 \frac{g_{0}L}{U^{2}} \textbf{g$^{*}$}
\end{equation}

\medskip
\noindent
Para a Eq. \ref{chemical species} um procedimento semelhante é
realizado:

\begin{equation}
 (c_{s}-c_{0}) \frac{U}{L} \frac{\partial c^{*}}{\partial t^{*}}
 +
 (c_{s}-c_{0}) \frac{U}{L} \textbf{v$^{*}$} \cdot \nabla^{*} c^{*}
 =
 (c_{s}-c_{0}) \frac{D_{0}}{L^{2}} D^{*} \nabla^{*2} c^{*}
\end{equation}

\medskip
\noindent
Multiplicando ambos os lados por $L/U(c_{s}-c_{0})$, temos:

\begin{equation} \label{especie quimica adimensional 2}
 \frac{\partial c^{*}}{\partial t^{*}}
 +
 \textbf{v$^{*}$} \cdot \nabla^{*} c^{*}
 =
 \frac{D_{0}}{UL} D^{*} \nabla^{*2} c^{*}
\end{equation}

\medskip
\noindent
isto é

\begin{equation} \label{especie quimica adimensional 1}
 \frac{\partial c^{*}}{\partial t^{*}}
 +
 \textbf{v$^{*}$} \cdot \nabla^{*} c^{*}
 =
 \frac{D_{0}}{UL} \nabla^{*2} c^{*}
\end{equation}

\newpage
Importantes grupos adimensionais são encontrados nas
Eqs. \ref{continuidade adimensional 1}, \ref{navier-stokes adimensional 1}
e \ref{especie quimica adimensional 1}. 
A descrição destes grupos são apresentados abaixo:

\begin{itemize}
 \item \textbf{Número de Reynolds} ($Re$): 
 Relação entre as forças de inércia e as forças viscosas
 que agem sobre uma partícula de fluido em movimento.
 É representado por:
 \begin{equation}
  Re = \frac{UL}{\nu_{0}}
 \end{equation}
 
 onde, $\nu_{0}$, $U$ e $L$ são os valores da 
 viscosidade cinemática do fluido, 
 velocidade de referência
 e comprimento característico do problema
 respectivamente. O número de Reynolds é
 frequentemente utilizado para definir o
 escoamento, onde $Re < Re_{critico}$ é
 definido como laminar e $Re > Re_{critico}$
 como turbulento. Segundo Fox, McDonald e Pritchard (2012) \cite{fox2012}, o
 $Re_{critico}$ para escoamentos em tubos é
 $Re_{critico} \simeq 2300$.

 \item \textbf{Número de Froude} ($Fr$): 
 Relação entre as forças de inércia e as forças gravitacionais.
 É representado por:
 \begin{equation}
  Fr = \frac{U}{\sqrt{g_{0}L}}
 \end{equation}
 
 onde, $g_{0}$ é a aceleração da gravidade
 referencial.

 \item \textbf{Número de Péclet de massa} ($Pe_{m}$): 
 Relação entre a dimensão da camada característica do
 problema e a espessua da camada limite de
 concentração de espécie química. Também pode ser
 interpretado como a relação entre a concentração
 transferida por convecção e por difusão.
 É representado por:
 \begin{equation}
  Pe_{m} = \frac{D_{0}}{UL}
 \end{equation}
 
 onde, $D_{0}$ é o coeficiente de difusão da
 espécie química. O número $Pe_{m}$ é frequentemente
 apresentado pelo produto de outros dois grupos tais
 como o \textit{número de Reynolds} e o \textit{número de Schmidt}.

 \newpage
 \item \textbf{Número de Schmidt} ($Sc$): 
 Relação entre a espessura da camada limite hidrodinâmica
 e a difusão de espécie química.
 É representado por:
 \begin{equation}
  Sc = \frac{\nu_{0}}{D_{0}}
 \end{equation}
\end{itemize}

\medskip
Substituindo estes grupos adimensionais nas Eqs.
\ref{continuidade adimensional 1}, \ref{navier-stokes adimensional 1}
e \ref{especie quimica adimensional 1} e retirando o asterisco, temos:

\begin{equation} \label{continuidade adimensional}
 \nabla \cdot \textbf{v} = 0
\end{equation}

\begin{equation} \label{navier-stokes adimensional}
 \frac{\partial \textbf{v}}{\partial t} 
 + 
 \textbf{v} \cdot \nabla \textbf{v}
 =
 -
 \nabla p
 +
 \frac{1}{Re} \nabla^{2} \textbf{v}
 +
 \frac{1}{Fr^{2}} \textbf{g}
\end{equation}

\begin{equation} \label{especie quimica adimensional}
 \frac{\partial c}{\partial t}
 +
 \textbf{v} \cdot \nabla c
 =
 \frac{1}{ReSc} \nabla^{2} c
\end{equation}

\medskip
As Eqs.
\ref{continuidade adimensional}, 
\ref{navier-stokes adimensional} e 
\ref{especie quimica adimensional}
são a forma adimensional das equações
da continuidade, Navier-Stokes e
transporte de espécie química para um
fluido newtoniano e incompressível respectivamente.

