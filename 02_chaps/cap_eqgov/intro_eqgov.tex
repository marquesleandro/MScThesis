Neste trabalho, o fluido é considerado como
um meio contínuo. Isso significa que dado um elemento
de fluido infinitesimal, o mesmo é suficientemente grande
para que não haja a presença de espaços vazios em seu meio.
Dessa forma, um escoamento
pode ser modelado segundo os princípios de conservação
universal tais como:

\begin{itemize}
 \item Conservação da Massa
 \item Conservação da Quantidade de Movimento Linear
 \item Conservação de Espécie Química
\end{itemize}

Estes são os princípios que governam o escoamento
proposto neste trabalho. Na seção \ref{conservacao massa},
apresentaremos o princípio da 
conservação da massa e a \textit{equação da continuidade}
para um fluido incompressível. 
Na seção \ref{conservacao movimento}, a \textit{equação de Navier-Stokes} 
para um fluido incompressível é apresentada
segundo o princípio de conservação da
quantidade de movimento linear para um elemento de fluido.
Já na seção \ref{conservacao especie}, apresentaremos a \textit{equação de Transporte
de Espécie Química}. 
Em seguida, as equações de governo são adimensionalizadas na seção \ref{adimensionalizacao}
e a equação de Navier-Stokes é apresentada
segundo a \textit{formulação corrente-vorticidade} na seção \ref{corrente vorticidade}.
