In this work, the fluid is considered as a continuum medium. 
This means that given an element of infinitesimal fluid,
 it is large enough that there are no empty spaces
 in its domain. Thereby, the flow can be modeled
 according to universal conservation principles such as:

\begin{itemize}
 \item Mass Conservation
 \item Linear Momentum Conservation
 \item Species Transport Conservation
\end{itemize}

These are the principles that govern the flow proposed in this work.
 In the section \ref{conservacao massa}, we will present the principle
 of mass conservation and the \textit{continuity equation} for
 an incompressible fluid. In the section
 \ref{conservacao movimento}, the
 \textit{Navier Stokes equation} for an incompressible fluid
 is presented according to the principle of momentum linear conservation
 for a fluid element. In the section \ref{conservacao especie},
 we will present the \textit{Species Transport Conservation}.
 Then, the governing equations are non-dimensionalization
 in the \ref{adimensionalizacao} section and
 the Navier-Stokes equation is presented according to
 \textit{vorticity-streamfunction formulation} in the
 \ref{corrente vorticidade} section.
