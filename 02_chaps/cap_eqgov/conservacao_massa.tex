As presented by Batchelor (1967) \cite{batchelor1967},
 the principle of mass conservation without source term
 establishes that:


\medskip
\begin{center}
\textarray{Mass accumulation rate\\
           on control volume}
           $= -$ 
\textarray{Mass flux crossing\\ 
           the boundary}
\end{center}

\noindent Mathematically, the mass accumulation rate within on volume
 can be represented by:

\begin{equation} \label{massa 1} 
 \int_{V} \frac{\partial}{\partial t} dm
\end{equation}


\noindent where the infinitesimal mass $dm$ is defined
 as $dm = \rho dV$. Replacing it in Eq. \ref{massa 1}
 and considering that the control volume does not vary over time,
 we have:

\begin{equation}
 \int_{V} \frac{\partial}{\partial t} dm
 =
 \int_{V} \frac{\partial}{\partial t} \big( \rho dV \big)
 = 
 \int_{V} \frac{\partial \rho}{\partial t} dV
 +
 \int_{V} \rho \frac{\partial dV}{\partial t}
 = 
 \int_{V} \frac{\partial \rho}{\partial t} dV
\end{equation}

\medskip
\noindent The mass flux crossing boundary can be mathematically represented by:


\begin{equation}  
 \oint_{S} \rho \left(\textbf{v} - \hat{\textbf{v}}\right) \cdot \textbf{n} dA
\end{equation}

\medskip
\noindent Thereby, according to mass conservation theorem:

\begin{equation}
 \int_{V} \frac{\partial \rho}{\partial t} dV
 = - 
 \oint_{S} \rho \left(\textbf{v} - \hat{\textbf{v}}\right) \cdot \textbf{n} dA
\end{equation}

\medskip
\noindent Applying the \textit{Gauss theorem} on surface integral:

\begin{equation}
 \int_{V} \frac{\partial \rho}{\partial t} dV
 = - 
 \int_{V} \nabla \cdot \left[ \rho \left(\textbf{v} - \hat{\textbf{v}}\right) \right] dV
\end{equation}

\medskip
\noindent
that is:

\begin{equation} \label{massa 2}
 \int_{V} \left[ \frac{\partial \rho}{\partial t}
 + 
 \nabla \cdot 
\left[ \rho \left(\textbf{v} - \hat{\textbf{v}}\right) \right] 
\right] dV 
 = 0 
\end{equation}

\medskip
\noindent 
Whereas the $dV \neq 0$,
the Eq. \ref{massa 2} can be presented as:

\begin{equation} \label{continuity equation}
 \frac{\partial \rho}{\partial t}
 + 
 \nabla \cdot 
\left[ \rho \left(\textbf{v} - \hat{\textbf{v}}\right) \right] 
 = 0 
\end{equation}

\medskip
\noindent 
where $\rho$ is density, $\textbf{v}$ is material velocity
 whose components are $\textbf{v} = \left[u,v\right]$,
$\hat{\textbf{v}}$ is mesh velocity,
$\nabla$ is Del operator whose components are 
$\nabla = \left[ \partial/\partial x, \partial / \partial y \right]$,
$x$ and $y$ are coordinates components and
$t$ is time.
The Eq. \ref{continuity equation} is known
as \textit{Continuity Equation} 
for Arbitrary Lagrangian-Eulerian description \cite{donea1982}.
Developing the equation, we have:

\begin{equation}
 \frac{\partial \rho}{\partial t}
 +
 \left(\textbf{v} - \hat{\textbf{v}}\right) \cdot \nabla \rho
 +
 \rho \nabla \cdot \textbf{v}
 = 0
\end{equation}

\medskip
According to fluid incompressible assumption,
the density does not depende on time and on coordinates.
Therefore, the 
$\partial \rho / \partial t$ e $\nabla \rho$ derivatives are 
null values.
Thus, the mass conservation is reduced to:

\begin{equation} \label{massa 3}
 \rho \nabla \cdot \textbf{v}
 = 0 
\end{equation}

\medskip
\noindent that is:

\begin{equation} \label{incompressible continuity equation}
 \nabla \cdot \textbf{v}
 = 0 
\end{equation}

\medskip
\noindent This equation is the \textit{continuity equation} for an incompressible flow.


\newpage



