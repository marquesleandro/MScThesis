Conforme apresentado por Pontes e Mangiavacchi (2016) \cite{pontes2016}, o princípio
de conservação da massa sem geração estabelece que:

\medskip
\begin{center}
\textarray{A taxa de acumu-\\
           lação de massa\\ 
           dentro do volume}
           $= -$ 
\textarray{O fluxo líquido\\ 
           de massa que cruza\\
           a fronteira}
\end{center}

\newpage
Matematicamente, a taxa de acumulação de 
massa dentro do volume pode ser representada como:

\begin{equation} \label{massa 1} 
 \int_{V} \frac{\partial}{\partial t} dm
\end{equation}

\noindent onde a massa infinitesimal $dm$ é definida como
$dm = \rho dV$. Substituindo-o na Eq. \ref{massa 1} e considerando
que o volume de controle não varia com o tempo, temos:

\begin{equation}
 \int_{V} \frac{\partial}{\partial t} dm
 =
 \int_{V} \frac{\partial}{\partial t} \big( \rho dV \big)
 = 
 \int_{V} \frac{\partial \rho}{\partial t} dV
 +
 \int_{V} \rho \frac{\partial dV}{\partial t}
 = 
 \int_{V} \frac{\partial \rho}{\partial t} dV
\end{equation}

\medskip
O fluxo líquido de massa que cruza a fronteira do volume pode ser
representado matematicamente como:

\begin{equation}  
 \oint_{S} \rho \textbf{v} \cdot \textbf{n} dA
\end{equation}

\medskip
\noindent Dessa forma, segundo o princípio de conservação da massa:

\begin{equation}
 \int_{V} \frac{\partial \rho}{\partial t} dV
 = - 
 \oint_{S} \rho \textbf{v} \cdot \textbf{n} dA
\end{equation}

\medskip
\noindent Aplicando o \textit{Teorema de Gauss} na integral
de área:

\begin{equation}
 \int_{V} \frac{\partial \rho}{\partial t} dV
 = - 
 \int_{V} \nabla \cdot \big( \rho \textbf{v} \big) dV
\end{equation}

\medskip
\noindent
isto é:

\begin{equation} \label{massa 2}
 \int_{V} \Bigg[ \frac{\partial \rho}{\partial t}
 + 
 \nabla \cdot \big( \rho \textbf{v} \big) \Bigg] dV
 = 0 
\end{equation}

\medskip
\noindent 
Considerando o fato de $dV \neq 0$,
a Eq. \ref{massa 2} pode ser expressa como:

\begin{equation} \label{continuity equation}
 \frac{\partial \rho}{\partial t}
 + 
 \nabla \cdot \big( \rho \textbf{v} \big)
 = 0 
\end{equation}

\medskip
\noindent 
onde $\rho$ é a massa específica, $\textbf{v}$ é o vetor
velocidade cujas componentes são $\textbf{v} = \big[u,v\big]$,
$\nabla$ é um operador diferencial cujas componentes são
$\nabla = \big[ \partial/\partial x, \partial / \partial y \big]$,
$x$ e $y$ são as componentes espaciais e
$t$ é a componente temporal.
A Eq. \ref{continuity equation} é conhecida
como \textit{Equação da Continuidade}.
Desenvolvendo a equação, temos:

\begin{equation}
 \frac{\partial \rho}{\partial t}
 +
 \textbf{v} \cdot \nabla \rho
 +
 \rho \nabla \cdot \textbf{v}
 = 0
\end{equation}

\medskip
Considerando a hipótese de incompressibilidade do fluido,
a massa específica não depende do tempo e da posição. Dessa forma, as derivadas
$\partial \rho / \partial t$ e $\nabla \rho$ são iguais a zero.
Assim, a conservação de massa é reduzida a:

\begin{equation} \label{massa 3}
 \rho \nabla \cdot \textbf{v}
 = 0 
\end{equation}

\medskip
\noindent isto é:

\begin{equation} \label{incompressible continuity equation}
 \nabla \cdot \textbf{v}
 = 0 
\end{equation}

\medskip
\noindent Esta é a equação da continuidade para um fluido incompressível.


\newpage



