The same concept of mass conservation is applied to the
linear momentum conservation. 
Therefore, the linear momentum conservation 
for a region with arbitrary motion is:


\begin{equation}
 \int_{V} \frac{\partial}{\partial t} \big( \rho \textbf{v} \big) dV
 = - 
 \oint_{S} \rho \textbf{v} \textbf{c} \cdot \textbf{n} dA
 +
 \oint_{S} \sigma \cdot \textbf{n} dA
 +
 \int_{V} \rho \textbf{g} dV
\end{equation}

\medskip
\noindent
where $\sigma$ is the stress tensor and 
$\textbf{g}$ is gravity vector.
Applying the \textit{Gauss Theorem} on surface integrals:

\begin{equation}
 \int_{V} \frac{\partial}{\partial t} \big( \rho \textbf{v} \big) dV
 = - 
 \int_{V} \nabla \cdot \big( \rho \textbf{v} \textbf{c} \big) dV
 +
 \int_{V} \nabla \cdot \sigma dV
 +
 \int_{V} \rho \textbf{g} dV
\end{equation}

\medskip
\noindent
that is:

\begin{equation} \label{qm 2}
 \int_{V} \Bigg[ \frac{\partial}{\partial t} \big( \rho \textbf{v} \big)
 + 
 \nabla \cdot \big( \rho \textbf{v} \textbf{c} \big)
 -
 \nabla \cdot \sigma
 -
 \rho \textbf{g} \Bigg] dV = 0
\end{equation}



\medskip
\noindent
In view of the $dV \neq 0$,
the Eq. \ref{qm 2} can be presented as:



\begin{equation}
 \frac{\partial}{\partial t} \big( \rho \textbf{v} \big)
 + 
 \nabla \cdot \big( \rho \textbf{v} \textbf{c} \big)
 -
 \nabla \cdot \sigma
 -
 \rho \textbf{g} = 0
\end{equation}


\medskip
\noindent
that is:

\begin{equation}
 \frac{\partial}{\partial t} \big( \rho \textbf{v} \big) 
 +
 \nabla \cdot \big( \rho \textbf{v} \textbf{c} \big)
 =
 \nabla \cdot \sigma
 +
 \rho \textbf{g}
\end{equation}

\medskip
\noindent
Developing the left hand side of equation, we have:

\begin{equation}
 \rho \frac{\partial \textbf{v}}{\partial t}
 +
 \textbf{v} \frac{\partial \rho}{\partial t}
 +
 \rho \textbf{c} \cdot \nabla \textbf{v}
 + 
 \textbf{v} \nabla \cdot \big( \rho \textbf{c} \big)
 =
 \rho \Bigg[ \frac{\partial \textbf{v}}{\partial t} + \textbf{v} \cdot \nabla \textbf{v} \Bigg]
 +
 \textbf{v} \Bigg[ \frac{\partial \rho}{\partial t} + \nabla \cdot \big( \rho \textbf{c} \big) \Bigg]
\end{equation}

\medskip
The last term of above equation is null because
 the \textit{continuity equation} (Eq. \ref{continuity equation}).
Thus, the linear momentum equation can be rewritten as:

\begin{equation} \label{qm 3}
 \rho \Bigg[ \frac{\partial \textbf{v}}{\partial t} + \textbf{c} \cdot \nabla \textbf{v} \Bigg]
 =
 \nabla \cdot \sigma
 +
 \rho \textbf{g}
\end{equation}

\medskip
\noindent
The stress tensor $\sigma$ can be split into
two tensors:

\begin{equation}
 \sigma = -p \textbf{I} + \tau
\end{equation}

\medskip
\noindent
where, $p$ is pressure field, \textbf{I} is the identity matrix and
$\tau$ id deviatoric stress. 
Replacing them in Eq. \ref{qm 3}, we have:

\begin{equation}
 \rho \Bigg[ \frac{\partial \textbf{v}}{\partial t} + \textbf{c} \cdot \nabla \textbf{v} \Bigg]
 =
 \nabla \cdot \big[ -p \textbf{I} + \tau \big]
 +
 \rho \textbf{g}
\end{equation}

\medskip
\noindent
that is:

\begin{equation} \label{qm 4}
 \rho \Bigg[ \frac{\partial \textbf{v}}{\partial t} + \textbf{c} \cdot \nabla \textbf{v} \Bigg]
 =
 -
 \nabla p
 +
 \nabla \cdot \tau
 +
 \rho \textbf{g}
\end{equation}

\medskip
The deviatoric stress $\tau$ depends on strain tensor rate and
we can define it relating to medium physical properties.
Whereas a homogeneous, isotropic fluid and the deviatoric stress
as a continuous and linear function of velocity gradient,
we have:

\begin{equation}
 \tau = \mu \big[ \nabla \textbf{v} + \big( \nabla \textbf{v} \big)^{T} \big]
      + \lambda \textbf{I} \nabla \cdot \textbf{v}
\end{equation}

\medskip
\noindent
where $\mu$ is dynamic viscosity of fluid,
$\lambda$ is known as the second viscosity coefficient and
\textbf{I} is identidy matrix.
Replacing them in Eq. \ref{qm 4}, we have:



\begin{equation} 
 \rho \Bigg[ \frac{\partial \textbf{v}}{\partial t} + \textbf{c} \cdot \nabla \textbf{v} \Bigg]
 =
 -
 \nabla p
 +
 \nabla \cdot \big[ 
 \mu \big[ \nabla \textbf{v} + \big( \nabla \textbf{v} \big)^{T} \big]
 + \lambda \textbf{I} \nabla \cdot \textbf{v}
 \big]
 +
 \rho \textbf{g}
\end{equation}


\medskip
\noindent
that is:

\begin{equation} 
 \rho \Bigg[ \frac{\partial \textbf{v}}{\partial t} + \textbf{c} \cdot \nabla \textbf{v} \Bigg]
 =
 -
 \nabla p
 +
 \nabla \cdot \big[ \mu \big[ \nabla \textbf{v} + \big( \nabla \textbf{v} \big)^{T} \big] \big]
 + 
 \nabla \cdot \big[ \lambda \textbf{I} \nabla \cdot \textbf{v} \big]
 +
 \rho \textbf{g}
\end{equation}


\medskip
\noindent
Taking into consideration that the dynamic viscosity 
$\mu$ does not depends on coordinates,
we have:

\begin{equation} 
 \rho \Bigg[ \frac{\partial \textbf{v}}{\partial t} + \textbf{c} \cdot \nabla \textbf{v} \Bigg]
 =
 -
 \nabla p
 +
 \mu \big[ \nabla \cdot \nabla \textbf{v} + \nabla \cdot \big( \nabla \textbf{v} \big)^{T} \big]
 +
 \nabla \cdot \big[ \lambda \textbf{I} \nabla \cdot \textbf{v} \big]
 +
 \rho \textbf{g}
\end{equation}

\medskip
\noindent
that is:

\begin{equation} 
 \rho \Bigg[ \frac{\partial \textbf{v}}{\partial t} + \textbf{c} \cdot \nabla \textbf{v} \Bigg]
 =
 -
 \nabla p
 +
 \mu \big[ \nabla^{2} \textbf{v} + \nabla \big( \nabla \cdot \textbf{v} \big) \big]
 +
 \nabla \cdot \big[ \lambda \textbf{I} \nabla \cdot \textbf{v} \big]
 +
 \rho \textbf{g}
\end{equation}

\medskip
\noindent
According to Eq. \ref{incompressible continuity equation}, we have:

\begin{equation} \label{navier-stokes}
 \frac{\partial \textbf{v}}{\partial t} + \textbf{c} \cdot \nabla \textbf{v}
 =
 -
 \frac{1}{\rho} \nabla p
 +
 \nu \nabla^{2} \textbf{v}
 +
 \textbf{g}
\end{equation}

\medskip
\noindent
where $\nu$ is the kinematic viscosity of fluid.
 The Eq. \ref{navier-stokes} is known as
\textit{Navier-Stokes Equation} and is valid for a
homogeneous, isotropic, incompressible fluid
and with viscosity that it does not depends on coordinates,
according to the Arbitrary Lagrangian-Eulerian description
\cite{donea1982}.

