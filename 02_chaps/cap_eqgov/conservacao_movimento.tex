O mesmo conceito da conservação de massa é aplicado para
a conservação da quantidade de movimento linear. Dessa forma,
o princípio de conservação da quantidade de movimento linear estabelece que:

\medskip
\begin{center}
\textarray{A taxa de acumulação\\
           da quantidade de\\ 
           movimento linear\\ 
           dentro do volume\\
           de controle}
           $= -$ 
\textarray{O fluxo líquido\\
           da quantidade\\ 
           de movimento\\
           linear que\\
           cruza a fronteira}
           $+$ 
\textarray{A resultante das \\
           forças aplicadas \\ 
           à superfície de \\
           controle e as\\
           forças de volume}
\end{center}

\medskip
Matematicamente, a taxa de acumulação da quantidade de movimento linear
dentro do volume de controle pode ser representada como:

\begin{equation} \label{qml 1} 
 \int_{V} \frac{\partial}{\partial t} \big( \rho \textbf{v} \big) dV
\end{equation}

\medskip
O fluxo líquido da quantidade de movimento
linear que cruza a fronteira do volume pode ser
representado matematicamente como:

\begin{equation}  
 \oint_{S} \rho \textbf{v} \textbf{v} \cdot \textbf{n} dA
\end{equation}

\medskip
A resultante das forças é dividida nas forças aplicadas
à superfície de controle e nas forças de volume.
A resultante das forças aplicadas à superfície de controle
pode ser representada por:

\begin{equation}  
 \oint_{S} \sigma \cdot \textbf{n} dA
\end{equation}

\medskip
\noindent
onde $\sigma$ é o tensor de tensões. A resultante
das forças de volume é representada por:

\begin{equation} 
 \int_{V} \rho \textbf{g} dV
\end{equation}

\medskip
\noindent
onde $\textbf{g}$ é o vetor da aceleração da gravidade.
Dessa forma, segundo o princípio de conservação da quantidade
de movimento linear:

\begin{equation}
 \int_{V} \frac{\partial}{\partial t} \big( \rho \textbf{v} \big) dV
 = - 
 \oint_{S} \rho \textbf{v} \textbf{v} \cdot \textbf{n} dA
 +
 \oint_{S} \sigma \cdot \textbf{n} dA
 +
 \int_{V} \rho \textbf{g} dV
\end{equation}

\medskip
\noindent
Aplicando o \textit{Teorema de Gauss} nas integrais
de área:

\begin{equation}
 \int_{V} \frac{\partial}{\partial t} \big( \rho \textbf{v} \big) dV
 = - 
 \int_{V} \nabla \cdot \big( \rho \textbf{v} \textbf{v} \big) dV
 +
 \int_{V} \nabla \cdot \sigma dV
 +
 \int_{V} \rho \textbf{g} dV
\end{equation}

\medskip
\noindent
isto é:

\begin{equation} \label{qm 2}
 \int_{V} \Bigg[ \frac{\partial}{\partial t} \big( \rho \textbf{v} \big)
 + 
 \nabla \cdot \big( \rho \textbf{v} \textbf{v} \big)
 -
 \nabla \cdot \sigma
 -
 \rho \textbf{g} \Bigg] dV = 0
\end{equation}



\medskip
\noindent
Considerando o fato de $dV \neq 0$,
a Eq. \ref{qm 2} pode ser expressa como:

\begin{equation}
 \frac{\partial}{\partial t} \big( \rho \textbf{v} \big)
 + 
 \nabla \cdot \big( \rho \textbf{v} \textbf{v} \big)
 -
 \nabla \cdot \sigma
 -
 \rho \textbf{g} = 0
\end{equation}


\medskip
\noindent
isto é:

\begin{equation}
 \frac{\partial}{\partial t} \big( \rho \textbf{v} \big) 
 +
 \nabla \cdot \big( \rho \textbf{v} \textbf{v} \big)
 =
 \nabla \cdot \sigma
 +
 \rho \textbf{g}
\end{equation}

\medskip
\noindent
Desenvolvendo o lado esquerdo da equação, temos:

\begin{equation}
 \rho \frac{\partial \textbf{v}}{\partial t}
 +
 \textbf{v} \frac{\partial \rho}{\partial t}
 +
 \rho \textbf{v} \cdot \nabla \textbf{v}
 + 
 \textbf{v} \cdot \nabla \big( \rho \textbf{v} \big)
 =
 \rho \Bigg[ \frac{\partial \textbf{v}}{\partial t} + \textbf{v} \cdot \nabla \textbf{v} \Bigg]
 +
 \textbf{v} \Bigg[ \frac{\partial \rho}{\partial t} + \nabla \big( \rho \textbf{v} \big) \Bigg]
\end{equation}

\newpage
Vimos que o último termo da equação acima
é nulo pois trata-se da equação da continuidade (Eq. \ref{continuity equation}).
Logo, a equação de conservação de momento pode ser reescrita como:

\begin{equation} \label{qm 3}
 \rho \Bigg[ \frac{\partial \textbf{v}}{\partial t} + \textbf{v} \cdot \nabla \textbf{v} \Bigg]
 =
 \nabla \cdot \sigma
 +
 \rho \textbf{g}
\end{equation}

\medskip
\noindent
O tensor de tensões $\sigma$ pode ser decomposto
na soma de outros dois tensores:

\begin{equation}
 \sigma = -p \textbf{I} + \tau
\end{equation}

\medskip
\noindent
onde, $p$ é o campo de pressão, \textbf{I} é a matriz identidade e
$\tau$ é o tensor desviatório. Substituindo na Eq. \ref{qm 3} temos:

\begin{equation}
 \rho \Bigg[ \frac{\partial \textbf{v}}{\partial t} + \textbf{v} \cdot \nabla \textbf{v} \Bigg]
 =
 \nabla \cdot \big[ -p \textbf{I} + \tau \big]
 +
 \rho \textbf{g}
\end{equation}

\medskip
\noindent
isto é:

\begin{equation} \label{qm 4}
 \rho \Bigg[ \frac{\partial \textbf{v}}{\partial t} + \textbf{v} \cdot \nabla \textbf{v} \Bigg]
 =
 -
 \nabla p
 +
 \nabla \cdot \tau
 +
 \rho \textbf{g}
\end{equation}

\medskip
O tensor desviatório $\tau$ depende da taxa do tensor deformação
e podemos defini-lo relacionando as propriedades físicas do meio.
Considerando um fluido homogêneo, isotrópico e
o tensor desviatório como sendo uma função contínua e linear do
gradiente de velocidade, temos:

\begin{equation}
 \tau = \mu \big[ \nabla \textbf{v} + \big( \nabla \textbf{v} \big)^{T} \big]
      + \lambda \textbf{I} \nabla \cdot \textbf{v}
\end{equation}

\medskip
\noindent
onde $\mu$ é a viscosidade dinâmica do fluido,
$\lambda$ é conhecido como o segundo coeficiente
de viscosidade e \textbf{I} é a matriz identidade.
Substituindo na Eq. \ref{qm 4}, temos:

\begin{equation} 
 \rho \Bigg[ \frac{\partial \textbf{v}}{\partial t} + \textbf{v} \cdot \nabla \textbf{v} \Bigg]
 =
 -
 \nabla p
 +
 \nabla \cdot \big[ 
 \mu \big[ \nabla \textbf{v} + \big( \nabla \textbf{v} \big)^{T} \big]
 + \lambda \textbf{I} \nabla \cdot \textbf{v}
 \big]
 +
 \rho \textbf{g}
\end{equation}


\medskip
\noindent
isto é:

\begin{equation} 
 \rho \Bigg[ \frac{\partial \textbf{v}}{\partial t} + \textbf{v} \cdot \nabla \textbf{v} \Bigg]
 =
 -
 \nabla p
 +
 \nabla \cdot \big[ \mu \big[ \nabla \textbf{v} + \big( \nabla \textbf{v} \big)^{T} \big] \big]
 + 
 \nabla \cdot \big[ \lambda \textbf{I} \nabla \cdot \textbf{v} \big]
 +
 \rho \textbf{g}
\end{equation}


\medskip
\noindent
Considerando que viscosidade dinâmica $\mu$ não depende da posição, temos:

\begin{equation} 
 \rho \Bigg[ \frac{\partial \textbf{v}}{\partial t} + \textbf{v} \cdot \nabla \textbf{v} \Bigg]
 =
 -
 \nabla p
 +
 \mu \big[ \nabla \cdot \nabla \textbf{v} + \nabla \cdot \big( \nabla \textbf{v} \big)^{T} \big]
 +
 \nabla \cdot \big[ \lambda \textbf{I} \nabla \cdot \textbf{v} \big]
 +
 \rho \textbf{g}
\end{equation}

\medskip
\noindent
isto é:

\begin{equation} 
 \rho \Bigg[ \frac{\partial \textbf{v}}{\partial t} + \textbf{v} \cdot \nabla \textbf{v} \Bigg]
 =
 -
 \nabla p
 +
 \mu \big[ \nabla^{2} \textbf{v} + \nabla \big( \nabla \cdot \textbf{v} \big) \big]
 +
 \nabla \cdot \big[ \lambda \textbf{I} \nabla \cdot \textbf{v} \big]
 +
 \rho \textbf{g}
\end{equation}

\medskip
\noindent
Segundo a Eq. \ref{incompressible continuity equation}, temos:

\begin{equation} \label{navier-stokes}
 \frac{\partial \textbf{v}}{\partial t} + \textbf{v} \cdot \nabla \textbf{v}
 =
 -
 \frac{1}{\rho} \nabla p
 +
 \nu \nabla^{2} \textbf{v}
 +
 \textbf{g}
\end{equation}

\medskip
\noindent
onde $\nu$ é o coeficiente de viscosidade
cinemática do fluido. A Eq. \ref{navier-stokes} é conhecida como 
\textit{Equação de Navier-Stokes} e é válida
para um fluido homogêneo, isotrópico, incompressível
e com viscosidade que não varie em função do espaço.

\newpage
