A equação de Navier-Stokes possui um forte
acoplamento entre o campo de pressão e o 
campo das velocidades. Este acoplamento
dificulta a implementação desta equação computacionalmente.
O desacoplamento da pressão e 
da velocidade é possível ao utilizar a \textit{formulação corrente-vorticidade}.
Para isso, substituiremos na equação \ref{navier-stokes adimensional} a
seguinte identidade vetorial:

\begin{equation}
 \textbf{v} \cdot \nabla \textbf{v}
 = 
 \nabla \frac{v^{2}}{2}
 - 
 \textbf{v} \times \nabla \times \textbf{v}
\end{equation}

\medskip
\noindent
Logo:

\begin{equation}
 \frac{\partial \textbf{v}}{\partial t} 
 + 
 \nabla \frac{v^{2}}{2}
 - 
 \textbf{v} \times \nabla \times \textbf{v}
 =
 -
 \nabla p
 +
 \frac{1}{Re} \nabla^{2} \textbf{v}
 +
 \frac{1}{Fr^{2}} \textbf{g}
\end{equation}

\medskip
\noindent
Operando o rotacional em ambos os lados na equação acima:

\begin{equation}
 \nabla \times \frac{\partial \textbf{v}}{\partial t} 
 + 
 \nabla \times \nabla \frac{v^{2}}{2}
 - 
 \nabla \times \textbf{v} \times \nabla \times \textbf{v}
 =
 -
 \nabla \times \nabla p
 +
 \frac{1}{Re} \nabla \times \nabla^{2} \textbf{v}
 +
 \frac{1}{Fr^{2}} \nabla \times \textbf{g}
\end{equation}

\medskip
\noindent
isto é:

\begin{equation}
 \frac{\partial}{\partial t} \big[ \nabla \times \textbf{v} \big]
 + 
 \nabla \times \nabla \frac{v^{2}}{2}
 - 
 \nabla \times \big[ \textbf{v} \times \nabla \times \textbf{v} \big]
 =
 -
 \nabla \times \nabla p
 +
 \frac{1}{Re} \nabla^{2} \big[ \nabla \times \textbf{v} \big]
 +
 \frac{1}{Fr^{2}} \nabla \times \textbf{g}
\end{equation}

\medskip
Os termos que contêm o operador gradiente se anulam, 
já que o rotacional do gradiente de um escalar
é zero. O último termo também se anula pois as
derivadas de uma constante, como no caso de \textbf{g},
são iguais a zero. Dessa forma, temos:

\begin{equation}
 \frac{\partial}{\partial t} \big[ \nabla \times \textbf{v} \big]
 - 
 \nabla \times \big[ \textbf{v} \times \nabla \times \textbf{v} \big]
 =
 \frac{1}{Re} \nabla^{2} \big[ \nabla \times \textbf{v} \big]
\end{equation}

\medskip
O vetor $\nabla \times \textbf{v}$ é conhecido
como \textit{vorticidade} ($w$).
Assim a equação pode ser representada como:

\begin{equation} \label{vort 1}
 \frac{\partial w}{\partial t}
 - 
 \nabla \times \big[ \textbf{v} \times w \big]
 =
 \frac{1}{Re} \nabla^{2} w
\end{equation}

\medskip
O segundo termo do lado esquerdo da Eq. \ref{vort 1}
pode ser substituido pela seguinte identidade vetorial:


\begin{equation}
 \nabla \times \big[ \textbf{v} \times w \big]
 =
 -
 \textbf{v} \cdot \nabla w
 +
 w \cdot \nabla v
\end{equation}

\medskip
\noindent
Sendo assim a Eq. \ref{vort 1} se transforma:

\begin{equation} \label{vort 2}
 \frac{\partial w}{\partial t}
 +
 \textbf{v} \cdot \nabla w
 - 
 w \cdot \nabla \textbf{v}
 =
 \frac{1}{Re} \nabla^{2} w
\end{equation}

\medskip
Para escoamentos bidimensionais, como no caso deste trabalho, a vorticidade é perpendicular
ao vetor velocidade. Sendo assim, o produto $w \cdot \nabla \textbf{v}$
se anulará como apresentado por Pontes e Mangiavacchi (2016) \cite{pontes2016}. Portanto:

\begin{equation} \label{equacao vorticidade}
 \frac{\partial w}{\partial t}
 +
 \textbf{v} \cdot \nabla w
 =
 \frac{1}{Re} \nabla^{2} w
\end{equation}

\medskip
A Eq. \ref{equacao vorticidade} é conhecida como
\textit{equação da vorticidade} para escoamentos
bidimensionais de um fluido newtoniano
e incompressível. Para um escoamento permanente e bidimensional
de um fluido incompressível,
a velocidade pode ser calculada a partir da vazão volumétrica.
Desta forma, a velocidade é substituida por um escalar.
Tal escalar é conhecido como \textit{função de corrente}
($\psi$). 
A relação entre as componentes da velocidade e
a função de corrente é apresentado
expandindo a equação da continuidade
(Eq. \ref{incompressible continuity equation}):

\begin{equation} \label{cor 1}
 \frac{\partial u}{\partial x}
 +
 \frac{\partial v}{\partial y}
 =
 0
\end{equation}
 
\medskip
A seguinte relação entre a função de corrente
e as componentes da velocidade pode ser definida
de modo que a Eq. \ref{cor 1} seja satisfeita:

\begin{equation} \label{cor 2}
\begin{aligned}
 u = \frac{\partial \psi}{\partial y}
 \qquad
 v = - \frac{\partial \psi}{\partial x}
\end{aligned}
\end{equation}

\medskip
A relação entre a função de corrente
e a vorticidade é apresentada expandindo
a operação $\nabla \times \textbf{v}$
para o caso bidimensional:

\begin{equation} \label{cor 3}
 \nabla \times \textbf{v}
 = 
 \frac{\partial v}{\partial x}
 - 
 \frac{\partial u}{\partial y}
\end{equation}

\medskip
\noindent
Substituindo a Eq. \ref{cor 2} na Eq. \ref{cor 3},
temos:

\begin{equation}
 w
 =
 - 
 \frac{\partial}{\partial x} \frac{\partial v}{\partial x}
 -
 \frac{\partial}{\partial y} \frac{\partial u}{\partial y}
\end{equation}

\medskip
\noindent
isto é:

\begin{equation}
 w
 = 
 -
 \nabla^{2} \psi
\end{equation}

\medskip
Dessa forma, as equações que governam o problema proposto
em sua forma adimensional e com a formulação corrente-vorticidade são apresentadas
abaixo:

\begin{align}
& \frac{\partial w}{\partial t}
 +
 \textbf{v} \cdot \nabla w
 =
 \frac{1}{Re} \nabla^{2} w \label{vorticidade}
 \\[10pt] 
& \nabla^{2} \psi
 = 
 - 
 w \label{corrente} \\[10pt]
& \textbf{v} = \textbf{D} \psi \label{continuidade} \\[10pt]
& \frac{\partial c}{\partial t}
 +
 \textbf{v} \cdot \nabla c
 =
 \frac{1}{ReSc} \nabla^{2} c \label{especie quimica}
\end{align}

\medskip
\noindent
onde \textbf{D} é um operador diferencial
cujas componentes são $\big[
\partial / \partial y,
- \partial / \partial x \big]$. 


