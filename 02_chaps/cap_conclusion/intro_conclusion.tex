Neste trabalho foi apresentado a equação Navier-Stokes
utilizando a formulação corrente-vorticidade acoplada
com a equação de transporte de espécie química numa abordagem
do Método dos Elementos Finitos onde o esquema Taylor-Galerkin
foi aplicado às equações de governo. Como a formulação corrente-vorticidade
não apresenta o acoplamento entre a velocidade e pressão,
podemos utilizar o elemento triangular linear sem restrição
possibilitando assim uma facilidade na implementação do código
númerico além das variáveis envolvidas serem escalares e não vetoriais
como no caso das variáveis primitivas.

\medskip
O código numérico apresentou
resultados satisfatórios comparados às soluções analíticas dos
\textit{Escoamento de Couette}, \textit{Escoamento de Poiseuille}
e \textit{Escoamento de Poiseuille em Meio Domínio} onde a condição
de superfície livre escorregamento no eixo de simetria foi aplicada.
Foi simulado, também, o escoamento numa cavidade com tampa móvel (\textit{lid-driven cavity flow})
onde os resultados foram comparados com aqueles apresentados por \cite{ghia1982} e \cite{marchi2009}
para vários números de Reynolds. 
Por fim, foi apresentado a comparação
entre os esquemas \textit{Galerkin} e \textit{Taylor-Galerkin} para um
escoamento puramente convectivo de uma função parabólica onde foi
possível observar a eficácia do esquema \textit{Taylor-Galerkin} 
em comparação ao esquema \textit{Galerkin} para a redução das
oscilações espúrias. Proporcionando, assim, a validação do código numérico
para problemas bidimensionais em coordenadas cartesianas e submetido à condição de contorno de
\textit{Dirichlet}.

\medskip
O objetivo desse trabalho é conhecer a dinâmica do escoamento
sanguíneo em uma artéria coronária com aterosclerose e com 
stent farmacológico implantado. Dessa forma, foi apresentado a simulação
para quatro geometrias modeladas como bidimensionais e
em coordenadas cartesianas. O perfil do campo de velocidade
se mostrou diferente daquele apresentado por \cite{wang2017},
onde as geometrias foram modeladas em coordenadas cilindricas,
principalmente para as geometrias em que o stent farmacológico
está presente. Tal desvio já era esperado porque, além de não usarmos
as variáveis primitivas neste trabalho, as artérias foram modeladas
como duas placas bidimensionais paralelas entre si e imóveis.
Em um problema tridimensional, o perfil do campo de velocidade
adquiria o formato matemático de uma calha e não um parabolóide
como é encontrado nas coordenadas cilindricas. Portanto, embora
o resultado qualitativo seja semelhante, o perfil do campo
de velocidade e seus respectivos valores são diferentes.

\medskip
\noindent
Para trabalhos futuros, destacamos três desenvolvimentos:

\begin{itemize}
 \item Utilização das variáveis primitivas na equação Navier-Stokes
       e seu desenvolvimento em coordenadas cilindricas

 \item Utilização do esquema \textit{Semi-Lagrangeano} para as derivadas materiais
       em substituição do esquema \textit{Taylor-Galerkin} com os termos de segunda ordem
       para a redução das oscilações espúrias  

 \item Simular a transferência de espécie química na parede da artéria modelando
       a mesma como um meio poroso e aplicando, assim, a \textit{Lei de Darcy}
\end{itemize}

