Apresentaremos neste capítulo os resultados obtidos
de quatro casos com a simulação numérica da equação de Navier-Stokes
utilizando a formulação corrente-vorticidade 
com a equação de transporte de espécie química onde
possuimos escoamentos bidimensionais incompressíveis e monofásicos
para todos os casos.
A primeira seção trata-se do \textit{Escoamento de Couette} e a solução
numérica é comparada com a solução analítica.
Já a segunda a seção trata-se do \textit{Escoamento de Poiseuille} e
a solução numérica também é comparada com a solução analítica.
A terceira seção refere-se ao escoamento de \textit{Poiseuille} 
em meio domínio, onde a condição de livre escorregamento é aplicada
sobre o eixo de simetria.
A quarta seção refere-se ao escoamento em uma cavidade com tampa deslizante
(\textit{lid-driven cavity flow}) e
a solução é comparada com os resultados apresentados por Ghia et al. (1982) \cite{ghia1982} e Marchi et al. (2009) \cite{marchi2009}. 
Já na quinta seção é apresentado a comparação dos 
esquemas Galerkin e Taylor-Galerkin do transporte de
um escalar submetido a um escoamento puramente convectivo.

\medskip
Todas as simulações numéricas foram realizadas nos computadores do \textit{Laboratório
de Ensaios Numéricos (LEN)} do \textit{Grupo de Estudos e Simulações
Ambientais em Reservatórios (GESAR)} com a seguinte configuração:

\begin{itemize}
 \item AMD FX-8350 4GHz com 8 núcleos, 32Gb de Memória RAM, 1000Gb de HD.
       Sistema operacional LINUX Ubuntu 16.04 LTS, utilizando a linguagem Python 2.7 
\end{itemize}
