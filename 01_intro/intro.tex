\noindent\textbf{INTRODUCTION}
\\

According to the World Health Organization (WHO), more people die each year from cardiovascular disease (CVD) than any other cause in the world in the last 15 years \cite{oms2018}.
It is estimated that 15.2 million people died from CVD in 2016,
representing 26.7\% of all deaths in the world. 
In Brazil, about 38\% of deaths due to CVD is in the procutive age range
(18 to 65 years) and the estimated
costs of CVD were R\$ 37.1 billion
in 2015, that is, 0.7\% GDP \cite{siqueira2017}.
About 60\% of CVD deaths
occurred due to coronary artery disease (CAD).
The main cause of CAD is the atherosclerosis which consists of
the accumulation of fatty plaques inside the artery wall causing
a decrease in lumen diameter.
The Atherosclerosis can be prevented with a change in harmful habits
such as: cigarette smoking, physical inactivity/low fitness and poor dietary habits \cite{spring2013}.
For a corrective approach, however, two treatments can be performed:
the \textit{Coronary Artery Bypass Sugery} (CABG) and
\textit{Percutaneous Transluminal Coronary Angioplasty} (PTCA).
The PTCA a procedure minimally invasive where a wire tube,
called \textit{stents} is placed \cite{sigwart1987}.

\medskip
This work is a continuation of that developed by Marques (2018)
\cite{bsc2018}, however
the goals of the current work plan are the development of a
Finite Element code using the Lagragian-Eulerian Arbitrary (ALE) 
description for the Vorticity-Streamfunction Formulation 
with Species Transport Equation, to implement the semi-Lagrangian Method
in place of the Taylor-Galerkin Method and
to know the dynamics of blood flow in a 
coronary artery with atherosclerosis and drug-eluting stent placed
for the several Schmidt numbers.

\medskip
The equations that govern the dynamics of blood flow in a coronary artery were developed according to the continuum hypothesis.
In this way, the principles of mass conservation, linear momentum conservation and chemical species were used.
The blood was considered as an incompressible, newtonian and one-phase 
fluid, as well as the mass diffusivity was approximated as constant.
The Navier-Stokes equation is shown according to the 
vorticity-streamfunction formulation with species transport 
equation without internal source term in an 
Arbitrary Lagrangian-Eulerian (ALE).

\medskip
The domain were discretized using an unstructured triangular 
mesh generated by GMSH open source \cite{gmsh} and 
the governing equations were discretized in spatial domain 
by the Galerkin Method. 
Due to decoupling between the velocity and pressure fields 
provided by vorticity-streamfunction formulation, the linear 
triangular element can be used without breaking the 
Babuska-Brezzi restriction \cite{babuska1971}\cite{brezzi1974}.
The Navier-Stokes and Species Transport equations were 
discretized in time using the semi-Lagragian Method 
\cite{pironneau1982} where the searching procedure is
discussed.

\medskip
The computational development was done in Python \cite{python} 
using Object-Oriented Paradigm (OOP) and the Chapter 4 
presents the computational cost for each simulator process 
in addition to the solution algorithm used
and the Laplacian Smoothing Method is discussed.
The numerical code validation was performed by 
comparison between numerical and analytical solutions 
in three benchmark problems:
\textit{Couette Flow}, \textit{Poiseuille Flow} and 
\textit{Half Poiseuille Flow}. 
Moreover, the horizontal and vertical velocities 
in \textit{Lid-Driven Cavity} was compared with 
those presented by \textit{Ghia et al.} \cite{ghia1982} and \textit{Marchi et al.} \cite{marchi2009}.
Then, the comparison between the semi-Lagrangian Method for 
an unstructured linear triangular and a quadratic mesh 
is presented in a pure convective flow.

\medskip
The blood flow and the concentration transport 
in coronary artery were simulated in 2 geometries 
types as suggested by \textit{Wang et al.} \cite{wang2017}, 
however with some modications to the cartesian coordinates.
These geometries types consist of one:
(1) coronary artery with atherosclerosis with 40\% lumen obstruction 
and with drug-eluting stent placed;
(2) real coronary artery with atherosclerosis and with drug-eluting stent placed.
The numerical simulation visualization was performed using \textit{Paraview} open source as proposed by \textit{Henderson (2007)} \cite{paraview}.


\medskip
\noindent
This work was organized as follows:

\begin{itemize}
 \item Introduction\\[-1cm] 
 \item Chapter 1: Literature Review\\[-1cm]
 \item Chapter 2: Governing Equations\\[-1cm]
 \item Chapter 3: Finite Element Method\\[-1cm]
 \item Chapter 4: Computational Code\\[-1cm]
 \item Chapter 5: Validation\\[-1cm]
 \item Chapter 6: Results\\[-1cm]
 \item Conclusion
\end{itemize}
