\noindent\textbf{INTRODUÇÃO}
\\

De acordo com a Organização Mundial da Saúde (OMS),
mais pessoas morrem anualmente devido às doenças cardiovasculares (DCV)
do que qualquer outra causa no mundo a cada ano \cite{oms}.
Estima-se que 17,7 milhões de pessoas morreram por DCV em 2015,
representando 31\% de todas as mortes no mundo. Aproximadamente
40\% das mortes por DCV ocorreram devido às doenças na artéria coronária (DAC).
A principal causa da DAC é a aterosclerose que consiste no acúmulo 
de placas de gordura no interior da parede da artéria ocasionando
uma diminuição do diâmetro do lúmen. A aterosclerose pode ser 
prevenida com uma mudança de hábitos nocivos tais como:
o uso de tabaco, o uso de álcool, falta de atividade física
e dietas não saudáveis.
Para uma abordagem corretiva, porém, dois tratamentos podem ser
realizados: o \textit{bypass coronário} (também conhecido
como ponte de safena) e \textit{a angioplastia coronária
transluminal percutânea} (PTCA). O PTCA é um procedimento
minimamente invasivo onde um tubo aramado, chamado \textit{stents},
é colocado. Os principais objetivos deste trabalho são 
o desenvolvimento de um código em Elementos Finitos para a formulação
corrente-vorticidade com o transporte de espécie química
e conhecer a dinâmica do escoamento sanguíneo em uma artéria coronária com
aterosclerose e com \textit{stents farmacológico} implantado.

\medskip
As equações que governam a dinâmica do escoamento sanguíneo em uma
artéria coronária foram desenvolvidas segundo a hipótese do meio
contínuo. Dessa forma, os príncipios de conservação de massa,
de quantidade de movimento linear e de espécie química foram utilizados.
O sangue foi considerado como um fluido incompressível, newtoniano e monofásico,
como também o coeficiente difusivo foi aproximado como constante.
A equação de Navier-Stokes é apresentada segundo a formulação
corrente-vorticidade com a equação de transporte de espécie
química sem geração interna.

\medskip
As equações de Navier-Stokes e de transporte de espécie química
foram discretizadas sobre uma malha triangular não estruturada
através do Método dos Elementos Finitos. Devido o desacoplamento
entre o campo de velocidade e de pressão possibilitado pela formulação
corrente-vorticidade, o elemento \textit{triangular linear} foi utilizado.
As equações foram discretizadas no tempo utilizando a expansão da série de Taylor
mantendo os termos de segunda ordem com o intuito de reduzir
as oscilações espúrias que são características das equações 
do tipo convecção-difusão. Em seguida, a formulação de Galerkin foi utilizada
para discretizarmos as equações no espaço. Dessa forma,
o esquema \textit{Taylor-Galerkin} foi utilizado como proposto por Donea (1984) \cite{donea1984},
Zienkiewicz e Taylor (2000) \cite{zienkiewiczvol3}.

\medskip
O desenvolvimento computacional foi feito em linguagem Python utilizando
o paradigma de orientação a objetos e no Capítulo 4 é apresentado
os principais \textit{scripts} do processo de simulação além
do algoritmo de solução. A validação do código numérico foi
realizada comparando a solução numérica com a solução analítica em três escoamentos:
o \textit{Escoamento de Couette}, o \textit{Escoamento de Poiseuille} e
o \textit{Escoamento de Poiseuille em meio domínio}. 
Também foi feito a comparação da solução do escoamento em uma cavidade
com tampa deslizante (\textit{lid-driven cavity flow}) com àquelas
apresentadas por Ghia et al. (1982) \cite{ghia1982} e Marchi et al. (2009) \cite{marchi2009}.
Em seguida, é apresentado a comparação dos esquemas Galerkin e 
Taylor-Galerkin quando as oscilações espúrias estão presente em um escoamento
puramente convectivo.

\medskip
A hidrodinâmica do escoamento sanguíneo e o transporte
de espécie química na artéria coronária foi simulada em 4 diferentes tipos de geometrias
como sugerido por Wang et al. (2017) \cite{wang2017}, porém com algumas modificações
para as coordenadas cartesianas. Esses diferentes tipos de geometrias
consistem em uma: 
(1) artéria coronária com aterosclerose com 40\% de obstrução do lúmen; 
(2) artéria coronária com aterosclerose e com stents farmacológico implantado; 
(3) artéria coronária real com aterosclerose com 40\% de obstrução do lúmen; 
(4) artéria coronária real com aterosclerose e com stents farmacológico implantado. 
A visualização da simulação numérica foi realizada
utilizando o software livre \textit{Paraview} proposto por Henderson (2007) \cite{paraview}.

\medskip
\noindent
Este trabalho foi organizado da seguinte forma:

\begin{itemize}
 \item Introdução\\[-1cm] 
 \item Capítulo 1: Revisão Bibliográfica\\[-1cm]
 \item Capítulo 2: Equações de Governo\\[-1cm]
 \item Capítulo 3: Método dos Elementos Finitos\\[-1cm]
 \item Capítulo 4: Código Numérico\\[-1cm]
 \item Capítulo 5: Validação do Código Numérico\\[-1cm]
 \item Capítulo 6: Resultados\\[-1cm]
 \item Conclusão
\end{itemize}
