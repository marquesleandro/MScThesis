\noindent\textbf{INTRODUCTION}
\\

According to the World Health Organization (WHO), more people die each year from cardiovascular disease (CVD) than any other cause in the world in the last 15 years \cite{oms2018}.
It is estimated that 15.2 million people died from CVD in 2016,
representing 26.7\% of all deaths in the world. 
In Brazil, about 38\% of deaths due to CVD is in the procutive age range
(18 to 65 years) and the estimated
costs of CVD were R\$ 37.1 billion
in 2015, that is, 0.7\% GDP \cite{siqueira2017}.
About 60\% of CVD deaths
occurred due to coronary artery disease (CAD).
The main cause of CAD is the atherosclerosis which consists of
the accumulation of fatty plaques inside the artery wall causing
a decrease in lumen diameter.
The Atherosclerosis can be prevented with a change in harmful habits
such as: cigarette smoking, physical inactivity/low fitness and poor dietary habits \cite{spring2013}.
For a corrective approach, however, two treatments can be performed:
the \textit{Coronary Artery Bypass Sugery} (CABG) and
\textit{Percutaneous Transluminal Coronary Angioplasty} (PTCA).
The PTCA a procedure minimally invasive where a wire tube,
called \textit{stents} is placed \cite{sigwart1987}.
The main objectives of this work plan are the development of a
Finite Element code using the Lagragian-Eulerian Arbitrary (ALE) description for the linear momentum conservation and for species transport equationin an incompressible, one-phase and newtonian fluid, in addtion to know the dynamics of blood flow in a coronary artery with atherosclerosis and drug-eluting stent placed.

\medskip
The equations that govern the dynamics of blood flow in a coronary artery were developed according to the continuum hypothesis.
In this way, the principles of mass conservation, linear momentum conservation and chemical species were used.
The blood was considered as an incompressible, newtonian and one-phase fluid, as well as the diffusive coefficient was approximated as constante.
The Navier-Stokes equation is shown according to the vorticity-streamfunction formulation with species transport equation without internal source term.

\medskip
The domain were discretized over an unstructured triangular mesh generated by GMSH open source \cite{gmsh} and the Finite Element Method (FEM) were used. 
Due to decoupling between the velocity and pressure fields provided by vorticity-streamfunction formulation, the linear triangular element can be used without breaking the Babuska-Brezzi restriction \cite{babuska1971}\cite{brezzi1974}.
The Navier-Stokes and Species Transport equations were discretized in time using the Taylor series and the semi-Lagragian Method \cite{pironneau1982} was used in order to reduce spurious oscillations that are usually seen in the diffusion-convection equations. Finally, Galerkin Method was used to discretize the equations in space. 

\medskip
The computational development was done in Python \cite{python} using Object-Oriented Paradigm (OOP) and the Chapter 4 presents the simulation process scripts in addition to the solution algorithm.
The numerical code validation was performed by comparison between numerical and analytical solutions in three benchmark problems:
\textit{Couette Flow}, \textit{Poiseuille Flow} and \textit{Half Poiseuille Flow}. 
The horizontal and vertical velocities in \textit{Lid-Driven Cavity} was compared with those presented by \textit{Ghia et al.} \cite{ghia1982} and \textit{Marchi et al.} \cite{marchi2009}.
Then, the comparison of the Taylor-Galerkin and semi-Lagrangian Methods is presented when spurious oscillations are present in a pure convective flow.

\medskip
The blood flow hydrodynamics and the species chemical transport in coronary artery were simulated in 4 geometries types as suggested by \textit{Wang et al.} \cite{wang2017}, but with some modications to the cartesian coordinates.
These geometries types consist of one:
(1) coronary artery with atherosclerosis with 40\% lumen obstruction;
(2) coronary artery with atherosclerosis and with drug-eluting stent placed;
(3) real coronary artery with atherosclerosis with 40\% lumen obstruction;
(4) real coronary artery with atherosclerosis and with drug-eluting stent placed.
The numerical simulation visualization was performed using \textit{Paraview} open source as proposed by \textit{Henderson (2007)} \cite{paraview}.


\medskip
\noindent
This work was organized as follows:

\begin{itemize}
 \item Introduction\\[-1cm] 
 \item Chapter 1: Literature Review\\[-1cm]
 \item Chapter 2: Governing Equations\\[-1cm]
 \item Chapter 3: Finite Element Method\\[-1cm]
 \item Chapter 4: Numerical Code\\[-1cm]
 \item Chapter 5: Validation\\[-1cm]
 \item Chapter 6: Results\\[-1cm]
 \item Conclusion
\end{itemize}
