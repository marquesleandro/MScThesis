	\clearpage
	\noindent\textbf{CONCLUSION}
	$\!$\\

	In this work, the Navier-Stokes equation according to the 
	vorticity-streamfunction formulation with the species transport 
	equation was presented in a Finite Element Method approach
	using the Arbitrary Lagrangian-Eulerian description.
	The governing equations were discretized in spatial domain by
	Galerkin Method and in time domain by semi-Lagrangian Method. 
	As the vorticity-streamfuntion formulation does not have the 
	coupling between the velocity and pressure fields, we can use the 
linear triangular element. In addition, the unknowns are scalar,
in contrast to the primitive variables that are vectorial fields.
In this way, a smooth implementation 
the numerical code is possible.

\medskip
A computational code was developed in a high-level programming language 
using the object orientation paradigm (OOP), 
providing a platform for numerical
simulations of the drug transport in bloodstream. The simulator 
is also able to describe in detail problems involving the flow 
of Newtonian fluids with scalar transport 
(such that the temperature) due to the generalized 
construction of the code. The average computational cost of 
each simulation process was measured for further improvement.
 

\medskip
The numerical code showed satisfactory results compared to the 
analytical solutions of \textit{Couette Flow}, 
\textit{Poiseuille Flow} and \textit{Half Poiseuille Flow} 
where the free-slip condition on the symmetry axis was applied. 
Moreover, the mesh convergence test was performed where
it was concluded that the numerical code error decreases linearly
as the mesh is refined.
The \textit{lid-driven cavity flow} was also simulated where the 
results were compared with those presented by Ghia et al. (1982) 
\cite{ghia1982} and Marchi et al. (2009) \cite{marchi2009} 
for several Reynolds number. Finally, the comparison between the
semi-Lagrangian Method in an unstructured linear and a quadratic
triangular mesh was
presented for a purely advection flow of a parabolic function 
where it was possible to observe the effectiveness of the
quadratic triangular element compared to linear element
for decrease the numerical diffusion in high Reynolds number.
Thus, it was possible to validate the numerical code
for two-dimensional simulations in cartesian coordinates,
using the Vorticity-Streamfunction Formulation
with the Species Transport Equation in an ALE-FE context
and sumitted to the Dirichlet boundary condition.


\medskip
The other goal of this work was to understand the dynamics of blood 
flow in a coronary artery with 
a drug-eluting stent placed for several Schmidt number. Thus, the simulation for two geometries 
modeled in a two-dimensional cartesian coordinates was presented. 
The profile of the velocity field was shown for the two proposed 
geometries where it was possible to observe the horizontal
velocity increase close to symmetric axis. In addition,
the curved channel model shown an acceptable approach because
the maximum horizontal velocity had a difference less than $1\%$ 
from the real channel.  
The simulation was done using several \textit{Schmidt} numbers, 
such as $1$, $10$, $100$ and $1000$. It was possible to verify in the 
simulation that the number of \textit{Schmidt} directly influences 
the transport of the drug in the bloodstream. For high values of 
the \textit{Schmidt} number, the transport of chemical species 
becomes purely convective and the diffusion in blood flow is decreased. 
In addition, the comparison between linear and quadratic triangular
element was performed to curved channel with stent. It is observed that
the concentration field had less diffusion when the quadratic element
was used. This effect may be due to improvement in the accuracy
of the simulation, however more detailed analyzes are necessary.


\vspace{0.7cm}
\noindent
The following futher developments are proposed:

\begin{itemize}
 \vspace{-0.3cm}
 \item Increase assembly performance

 \item Use of primitive variables in the Navier-Stokes equation in a 3D approach
 
 \item Blood flow model as a multiphase problem

 \item Blood model as a non-Newtonian fluid
\end{itemize}








