\clearpage
\noindent\textbf{CONCLUSION}
$\!$\\

In this work, the Navier-Stokes equation according to the 
vorticity-streamfunction formulation with the species transport 
equation was presented in a Finite Element Method approach 
and the Taylor-Galerkin method was applied to the governing equations. 
As the vorticity-streamfuntion formulation does not present the 
coupling between velocity and pressure fields, we can use the 
linear triangular element. In addition to the unknowns are scalar,
in contrast to the primitive variables that are vectorial fields.
In this way, a smooth implementation 
the numerical code is possible.

\medskip
A complete code was developed in a high-level programming language 
using the object orientation paradigm, providing a platform for numerical
simulations of the drug transport in bloodstream. The simulator 
is also able to describe in detail problems involving the flow 
of Newtonian fluids with scalar transport 
(as in concentration and temperature) due to the generalized 
construction of the code.

\medskip
The numerical code showed satisfactory results compared to the 
analytical solutions of \textit{Couette Flow}, 
\textit{Poiseuille Flow} and \textit{Half Poiseuille Flow} 
where the free-slip condition on the symmetry axis was applied. 
The \textit{lid-driven cavity flow} was also simulated where the 
results were compared with those presented by Ghia et al. (1982) 
\cite{ghia1982} and Marchi et al. (2009) \cite{marchi2009} 
for several Reynolds number. Finally, the comparison between the 
\textit{Galerkin} and \textit{Taylor-Galerkin} methods was 
presented for a purely advection flow of a parabolic function 
where it was possible to observe the effectiveness of the 
\textit{Taylor-Galerkin} method compared to \textit{Galerkin} method 
for decrease spurious oscillations. Thus, the validation of the 
numerical code was performed for two-dimensional 
convective-diffusive problems in cartesian coordinates 
and submitted to the boundary condition of \textit{Dirichlet}.

\medskip
The objective of this work was to understand the dynamics of blood 
flow in a coronary artery with atherosclerosis and with 
a drug-eluting stent. Thus, the simulation for four geometries 
modeled as two-dimensional and in cartesian coordinates was presented. 
The profile of the velocity field was shown for the four proposed 
geometries where it was possible to observe the increase 
in maximum velocity when the drug-eluting stent was implanted. 
The simulation was done using several \textit{Schmidt} numbers, 
such as $Sc=1$ and $10$. It was possible to verify in the 
simulation that the number of \textit{Schmidt} directly influences 
the transport of the drug in the bloodstream. For high values of 
the number of \textit{Schmidt}, the transport of chemical species 
becomes purely convective and its influence on the artery 
wall must be verified. In addition to the curved channel shown an
acceptable model for the real case, due to the $2$\% deviation
from the maximum non-dimensional velocity.

\vspace{0.7cm}
\noindent
The following futher developments is proposed:

\begin{itemize}
 \vspace{-0.3cm}
 \item Utilização do esquema \textit{Semi-Lagrangeano} para as derivadas materiais
       em substituição do esquema \textit{Taylor-Galerkin}
       para a redução das oscilações espúrias  

 \item Use of primitive variables in the Navier-Stokes equation in a 3D approach
 
 \item Blood flow model as a multiphase problem

 \item Blood model as a non-Newtonian fluid
\end{itemize}








