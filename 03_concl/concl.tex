\clearpage
\noindent\textbf{CONCLUSÃO}
$\!$\\

Neste trabalho foi apresentado a equação de Navier-Stokes
utilizando a formulação corrente-vorticidade
com a equação de transporte de espécie química em uma abordagem
do Método dos Elementos Finitos em que o esquema Taylor-Galerkin
foi aplicado às equações de governo. Como a formulação corrente-vorticidade
não apresenta o acoplamento entre a velocidade e pressão,
podemos utilizar o elemento triangular linear sem restrição
possibilitando assim uma facilidade na implementação do código
númerico além das variáveis envolvidas serem escalares e não vetoriais
como no caso das variáveis primitivas.

\medskip
Foi construído um código completo em linguagem de programação de alto nível 
usando o paradigma de orientação de objetos e a partir do presente momento,
possuímos uma plataforma de estudos de problemas de escoamento de fármacos
em artérias. O simulador é capaz também de descrever em detalhes problemas
envolvendo escoamento de fluidos newtonianos com transporte de natureza
escalar (como na concentração e na temperatura) devido a construção 
generalizada do código.

\medskip
O código numérico apresentou
resultados satisfatórios comparados às soluções analíticas dos
\textit{Escoamento de Couette}, \textit{Escoamento de Poiseuille}
e \textit{Escoamento de Poiseuille em Meio Domínio} onde a condição
de superfície livre escorregamento no eixo de simetria foi aplicada.
Foi simulado, também, o escoamento em uma cavidade com tampa móvel (\textit{lid-driven cavity flow})
onde os resultados foram comparados com aqueles apresentados por Ghia et al. (1982) \cite{ghia1982} e Marchi et al. (2009) \cite{marchi2009}
para vários números de Reynolds.
Por fim, foi apresentado a comparação
entre os esquemas \textit{Galerkin} e \textit{Taylor-Galerkin} para um
escoamento puramente convectivo de uma função parabólica onde foi
possível observar a eficácia do esquema \textit{Taylor-Galerkin}
em comparação ao esquema \textit{Galerkin} para a redução das
oscilações espúrias. 
Dessa forma, a validação do código numérico foi realizada
para problemas convectivos-difusivos bidimensionais em coordenadas cartesianas e submetido à condição de contorno de
\textit{Dirichlet}.

\medskip
O objetivo desse trabalho era conhecer a dinâmica do escoamento
sanguíneo em uma artéria coronária com aterosclerose e com 
stent farmacológico implantado. Dessa forma, foi apresentado a simulação
para quatro geometrias modeladas como bidimensionais e
em coordenadas cartesianas. 
Foi apresentado o perfil do campo de velocidade para as quatro geometrias propostas onde foi
possível observar o aumento da velocidade máxima quando o stent
farmacológico estava implantado.  
A simulação foi feita utilizando diversos números de \textit{Schmidt},
tais como $Sc = 1$ e $10$. Foi possível verificar na simulação que o número de
\textit{Schmidt} influencia diretamente no transporte do fármaco
na corrente sanguínea. Para elevados valores do número de \textit{Schmidt},
o transporte de espécie química torna-se puramente convectivo
e sua influência na parede da artéria deve ser verificada.

\vspace{0.7cm}
\noindent
Para trabalhos futuros, destacamos quatro desenvolvimentos:

\begin{itemize}
 \vspace{-0.3cm}
 \item Utilização do esquema \textit{Semi-Lagrangeano} para as derivadas materiais
       em substituição do esquema \textit{Taylor-Galerkin}
       para a redução das oscilações espúrias  

 \item Utilização das variáveis primitivas na equação de Navier-Stokes em uma abordagem 3D
 
 \item Modelar o escoamento sanguíneo como um problema multifásico

 \item Simular a transferência de espécie química na parede da artéria acoplado ao escoamento no lúmen.
\end{itemize}








